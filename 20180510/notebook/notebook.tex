
% Default to the notebook output style

    


% Inherit from the specified cell style.




    
\documentclass[11pt]{article}

    
    
    \usepackage[T1]{fontenc}
    % Nicer default font (+ math font) than Computer Modern for most use cases
    \usepackage{mathpazo}

    % Basic figure setup, for now with no caption control since it's done
    % automatically by Pandoc (which extracts ![](path) syntax from Markdown).
    \usepackage{graphicx}
    % We will generate all images so they have a width \maxwidth. This means
    % that they will get their normal width if they fit onto the page, but
    % are scaled down if they would overflow the margins.
    \makeatletter
    \def\maxwidth{\ifdim\Gin@nat@width>\linewidth\linewidth
    \else\Gin@nat@width\fi}
    \makeatother
    \let\Oldincludegraphics\includegraphics
    % Set max figure width to be 80% of text width, for now hardcoded.
    \renewcommand{\includegraphics}[1]{\Oldincludegraphics[width=.8\maxwidth]{#1}}
    % Ensure that by default, figures have no caption (until we provide a
    % proper Figure object with a Caption API and a way to capture that
    % in the conversion process - todo).
    \usepackage{caption}
    \DeclareCaptionLabelFormat{nolabel}{}
    \captionsetup{labelformat=nolabel}

    \usepackage{adjustbox} % Used to constrain images to a maximum size 
    \usepackage{xcolor} % Allow colors to be defined
    \usepackage{enumerate} % Needed for markdown enumerations to work
    \usepackage{geometry} % Used to adjust the document margins
    \usepackage{amsmath} % Equations
    \usepackage{amssymb} % Equations
    \usepackage{textcomp} % defines textquotesingle
    % Hack from http://tex.stackexchange.com/a/47451/13684:
    \AtBeginDocument{%
        \def\PYZsq{\textquotesingle}% Upright quotes in Pygmentized code
    }
    \usepackage{upquote} % Upright quotes for verbatim code
    \usepackage{eurosym} % defines \euro
    \usepackage[mathletters]{ucs} % Extended unicode (utf-8) support
    \usepackage[utf8x]{inputenc} % Allow utf-8 characters in the tex document
    \usepackage{fancyvrb} % verbatim replacement that allows latex
    \usepackage{grffile} % extends the file name processing of package graphics 
                         % to support a larger range 
    % The hyperref package gives us a pdf with properly built
    % internal navigation ('pdf bookmarks' for the table of contents,
    % internal cross-reference links, web links for URLs, etc.)
    \usepackage{hyperref}
    \usepackage{longtable} % longtable support required by pandoc >1.10
    \usepackage{booktabs}  % table support for pandoc > 1.12.2
    \usepackage[inline]{enumitem} % IRkernel/repr support (it uses the enumerate* environment)
    \usepackage[normalem]{ulem} % ulem is needed to support strikethroughs (\sout)
                                % normalem makes italics be italics, not underlines
    

    
    
    % Colors for the hyperref package
    \definecolor{urlcolor}{rgb}{0,.145,.698}
    \definecolor{linkcolor}{rgb}{.71,0.21,0.01}
    \definecolor{citecolor}{rgb}{.12,.54,.11}

    % ANSI colors
    \definecolor{ansi-black}{HTML}{3E424D}
    \definecolor{ansi-black-intense}{HTML}{282C36}
    \definecolor{ansi-red}{HTML}{E75C58}
    \definecolor{ansi-red-intense}{HTML}{B22B31}
    \definecolor{ansi-green}{HTML}{00A250}
    \definecolor{ansi-green-intense}{HTML}{007427}
    \definecolor{ansi-yellow}{HTML}{DDB62B}
    \definecolor{ansi-yellow-intense}{HTML}{B27D12}
    \definecolor{ansi-blue}{HTML}{208FFB}
    \definecolor{ansi-blue-intense}{HTML}{0065CA}
    \definecolor{ansi-magenta}{HTML}{D160C4}
    \definecolor{ansi-magenta-intense}{HTML}{A03196}
    \definecolor{ansi-cyan}{HTML}{60C6C8}
    \definecolor{ansi-cyan-intense}{HTML}{258F8F}
    \definecolor{ansi-white}{HTML}{C5C1B4}
    \definecolor{ansi-white-intense}{HTML}{A1A6B2}

    % commands and environments needed by pandoc snippets
    % extracted from the output of `pandoc -s`
    \providecommand{\tightlist}{%
      \setlength{\itemsep}{0pt}\setlength{\parskip}{0pt}}
    \DefineVerbatimEnvironment{Highlighting}{Verbatim}{commandchars=\\\{\}}
    % Add ',fontsize=\small' for more characters per line
    \newenvironment{Shaded}{}{}
    \newcommand{\KeywordTok}[1]{\textcolor[rgb]{0.00,0.44,0.13}{\textbf{{#1}}}}
    \newcommand{\DataTypeTok}[1]{\textcolor[rgb]{0.56,0.13,0.00}{{#1}}}
    \newcommand{\DecValTok}[1]{\textcolor[rgb]{0.25,0.63,0.44}{{#1}}}
    \newcommand{\BaseNTok}[1]{\textcolor[rgb]{0.25,0.63,0.44}{{#1}}}
    \newcommand{\FloatTok}[1]{\textcolor[rgb]{0.25,0.63,0.44}{{#1}}}
    \newcommand{\CharTok}[1]{\textcolor[rgb]{0.25,0.44,0.63}{{#1}}}
    \newcommand{\StringTok}[1]{\textcolor[rgb]{0.25,0.44,0.63}{{#1}}}
    \newcommand{\CommentTok}[1]{\textcolor[rgb]{0.38,0.63,0.69}{\textit{{#1}}}}
    \newcommand{\OtherTok}[1]{\textcolor[rgb]{0.00,0.44,0.13}{{#1}}}
    \newcommand{\AlertTok}[1]{\textcolor[rgb]{1.00,0.00,0.00}{\textbf{{#1}}}}
    \newcommand{\FunctionTok}[1]{\textcolor[rgb]{0.02,0.16,0.49}{{#1}}}
    \newcommand{\RegionMarkerTok}[1]{{#1}}
    \newcommand{\ErrorTok}[1]{\textcolor[rgb]{1.00,0.00,0.00}{\textbf{{#1}}}}
    \newcommand{\NormalTok}[1]{{#1}}
    
    % Additional commands for more recent versions of Pandoc
    \newcommand{\ConstantTok}[1]{\textcolor[rgb]{0.53,0.00,0.00}{{#1}}}
    \newcommand{\SpecialCharTok}[1]{\textcolor[rgb]{0.25,0.44,0.63}{{#1}}}
    \newcommand{\VerbatimStringTok}[1]{\textcolor[rgb]{0.25,0.44,0.63}{{#1}}}
    \newcommand{\SpecialStringTok}[1]{\textcolor[rgb]{0.73,0.40,0.53}{{#1}}}
    \newcommand{\ImportTok}[1]{{#1}}
    \newcommand{\DocumentationTok}[1]{\textcolor[rgb]{0.73,0.13,0.13}{\textit{{#1}}}}
    \newcommand{\AnnotationTok}[1]{\textcolor[rgb]{0.38,0.63,0.69}{\textbf{\textit{{#1}}}}}
    \newcommand{\CommentVarTok}[1]{\textcolor[rgb]{0.38,0.63,0.69}{\textbf{\textit{{#1}}}}}
    \newcommand{\VariableTok}[1]{\textcolor[rgb]{0.10,0.09,0.49}{{#1}}}
    \newcommand{\ControlFlowTok}[1]{\textcolor[rgb]{0.00,0.44,0.13}{\textbf{{#1}}}}
    \newcommand{\OperatorTok}[1]{\textcolor[rgb]{0.40,0.40,0.40}{{#1}}}
    \newcommand{\BuiltInTok}[1]{{#1}}
    \newcommand{\ExtensionTok}[1]{{#1}}
    \newcommand{\PreprocessorTok}[1]{\textcolor[rgb]{0.74,0.48,0.00}{{#1}}}
    \newcommand{\AttributeTok}[1]{\textcolor[rgb]{0.49,0.56,0.16}{{#1}}}
    \newcommand{\InformationTok}[1]{\textcolor[rgb]{0.38,0.63,0.69}{\textbf{\textit{{#1}}}}}
    \newcommand{\WarningTok}[1]{\textcolor[rgb]{0.38,0.63,0.69}{\textbf{\textit{{#1}}}}}
    
    
    % Define a nice break command that doesn't care if a line doesn't already
    % exist.
    \def\br{\hspace*{\fill} \\* }
    % Math Jax compatability definitions
    \def\gt{>}
    \def\lt{<}
    % Document parameters
    \title{Kaggle\_data\_science}
    
    
    

    % Pygments definitions
    
\makeatletter
\def\PY@reset{\let\PY@it=\relax \let\PY@bf=\relax%
    \let\PY@ul=\relax \let\PY@tc=\relax%
    \let\PY@bc=\relax \let\PY@ff=\relax}
\def\PY@tok#1{\csname PY@tok@#1\endcsname}
\def\PY@toks#1+{\ifx\relax#1\empty\else%
    \PY@tok{#1}\expandafter\PY@toks\fi}
\def\PY@do#1{\PY@bc{\PY@tc{\PY@ul{%
    \PY@it{\PY@bf{\PY@ff{#1}}}}}}}
\def\PY#1#2{\PY@reset\PY@toks#1+\relax+\PY@do{#2}}

\expandafter\def\csname PY@tok@w\endcsname{\def\PY@tc##1{\textcolor[rgb]{0.73,0.73,0.73}{##1}}}
\expandafter\def\csname PY@tok@c\endcsname{\let\PY@it=\textit\def\PY@tc##1{\textcolor[rgb]{0.25,0.50,0.50}{##1}}}
\expandafter\def\csname PY@tok@cp\endcsname{\def\PY@tc##1{\textcolor[rgb]{0.74,0.48,0.00}{##1}}}
\expandafter\def\csname PY@tok@k\endcsname{\let\PY@bf=\textbf\def\PY@tc##1{\textcolor[rgb]{0.00,0.50,0.00}{##1}}}
\expandafter\def\csname PY@tok@kp\endcsname{\def\PY@tc##1{\textcolor[rgb]{0.00,0.50,0.00}{##1}}}
\expandafter\def\csname PY@tok@kt\endcsname{\def\PY@tc##1{\textcolor[rgb]{0.69,0.00,0.25}{##1}}}
\expandafter\def\csname PY@tok@o\endcsname{\def\PY@tc##1{\textcolor[rgb]{0.40,0.40,0.40}{##1}}}
\expandafter\def\csname PY@tok@ow\endcsname{\let\PY@bf=\textbf\def\PY@tc##1{\textcolor[rgb]{0.67,0.13,1.00}{##1}}}
\expandafter\def\csname PY@tok@nb\endcsname{\def\PY@tc##1{\textcolor[rgb]{0.00,0.50,0.00}{##1}}}
\expandafter\def\csname PY@tok@nf\endcsname{\def\PY@tc##1{\textcolor[rgb]{0.00,0.00,1.00}{##1}}}
\expandafter\def\csname PY@tok@nc\endcsname{\let\PY@bf=\textbf\def\PY@tc##1{\textcolor[rgb]{0.00,0.00,1.00}{##1}}}
\expandafter\def\csname PY@tok@nn\endcsname{\let\PY@bf=\textbf\def\PY@tc##1{\textcolor[rgb]{0.00,0.00,1.00}{##1}}}
\expandafter\def\csname PY@tok@ne\endcsname{\let\PY@bf=\textbf\def\PY@tc##1{\textcolor[rgb]{0.82,0.25,0.23}{##1}}}
\expandafter\def\csname PY@tok@nv\endcsname{\def\PY@tc##1{\textcolor[rgb]{0.10,0.09,0.49}{##1}}}
\expandafter\def\csname PY@tok@no\endcsname{\def\PY@tc##1{\textcolor[rgb]{0.53,0.00,0.00}{##1}}}
\expandafter\def\csname PY@tok@nl\endcsname{\def\PY@tc##1{\textcolor[rgb]{0.63,0.63,0.00}{##1}}}
\expandafter\def\csname PY@tok@ni\endcsname{\let\PY@bf=\textbf\def\PY@tc##1{\textcolor[rgb]{0.60,0.60,0.60}{##1}}}
\expandafter\def\csname PY@tok@na\endcsname{\def\PY@tc##1{\textcolor[rgb]{0.49,0.56,0.16}{##1}}}
\expandafter\def\csname PY@tok@nt\endcsname{\let\PY@bf=\textbf\def\PY@tc##1{\textcolor[rgb]{0.00,0.50,0.00}{##1}}}
\expandafter\def\csname PY@tok@nd\endcsname{\def\PY@tc##1{\textcolor[rgb]{0.67,0.13,1.00}{##1}}}
\expandafter\def\csname PY@tok@s\endcsname{\def\PY@tc##1{\textcolor[rgb]{0.73,0.13,0.13}{##1}}}
\expandafter\def\csname PY@tok@sd\endcsname{\let\PY@it=\textit\def\PY@tc##1{\textcolor[rgb]{0.73,0.13,0.13}{##1}}}
\expandafter\def\csname PY@tok@si\endcsname{\let\PY@bf=\textbf\def\PY@tc##1{\textcolor[rgb]{0.73,0.40,0.53}{##1}}}
\expandafter\def\csname PY@tok@se\endcsname{\let\PY@bf=\textbf\def\PY@tc##1{\textcolor[rgb]{0.73,0.40,0.13}{##1}}}
\expandafter\def\csname PY@tok@sr\endcsname{\def\PY@tc##1{\textcolor[rgb]{0.73,0.40,0.53}{##1}}}
\expandafter\def\csname PY@tok@ss\endcsname{\def\PY@tc##1{\textcolor[rgb]{0.10,0.09,0.49}{##1}}}
\expandafter\def\csname PY@tok@sx\endcsname{\def\PY@tc##1{\textcolor[rgb]{0.00,0.50,0.00}{##1}}}
\expandafter\def\csname PY@tok@m\endcsname{\def\PY@tc##1{\textcolor[rgb]{0.40,0.40,0.40}{##1}}}
\expandafter\def\csname PY@tok@gh\endcsname{\let\PY@bf=\textbf\def\PY@tc##1{\textcolor[rgb]{0.00,0.00,0.50}{##1}}}
\expandafter\def\csname PY@tok@gu\endcsname{\let\PY@bf=\textbf\def\PY@tc##1{\textcolor[rgb]{0.50,0.00,0.50}{##1}}}
\expandafter\def\csname PY@tok@gd\endcsname{\def\PY@tc##1{\textcolor[rgb]{0.63,0.00,0.00}{##1}}}
\expandafter\def\csname PY@tok@gi\endcsname{\def\PY@tc##1{\textcolor[rgb]{0.00,0.63,0.00}{##1}}}
\expandafter\def\csname PY@tok@gr\endcsname{\def\PY@tc##1{\textcolor[rgb]{1.00,0.00,0.00}{##1}}}
\expandafter\def\csname PY@tok@ge\endcsname{\let\PY@it=\textit}
\expandafter\def\csname PY@tok@gs\endcsname{\let\PY@bf=\textbf}
\expandafter\def\csname PY@tok@gp\endcsname{\let\PY@bf=\textbf\def\PY@tc##1{\textcolor[rgb]{0.00,0.00,0.50}{##1}}}
\expandafter\def\csname PY@tok@go\endcsname{\def\PY@tc##1{\textcolor[rgb]{0.53,0.53,0.53}{##1}}}
\expandafter\def\csname PY@tok@gt\endcsname{\def\PY@tc##1{\textcolor[rgb]{0.00,0.27,0.87}{##1}}}
\expandafter\def\csname PY@tok@err\endcsname{\def\PY@bc##1{\setlength{\fboxsep}{0pt}\fcolorbox[rgb]{1.00,0.00,0.00}{1,1,1}{\strut ##1}}}
\expandafter\def\csname PY@tok@kc\endcsname{\let\PY@bf=\textbf\def\PY@tc##1{\textcolor[rgb]{0.00,0.50,0.00}{##1}}}
\expandafter\def\csname PY@tok@kd\endcsname{\let\PY@bf=\textbf\def\PY@tc##1{\textcolor[rgb]{0.00,0.50,0.00}{##1}}}
\expandafter\def\csname PY@tok@kn\endcsname{\let\PY@bf=\textbf\def\PY@tc##1{\textcolor[rgb]{0.00,0.50,0.00}{##1}}}
\expandafter\def\csname PY@tok@kr\endcsname{\let\PY@bf=\textbf\def\PY@tc##1{\textcolor[rgb]{0.00,0.50,0.00}{##1}}}
\expandafter\def\csname PY@tok@bp\endcsname{\def\PY@tc##1{\textcolor[rgb]{0.00,0.50,0.00}{##1}}}
\expandafter\def\csname PY@tok@fm\endcsname{\def\PY@tc##1{\textcolor[rgb]{0.00,0.00,1.00}{##1}}}
\expandafter\def\csname PY@tok@vc\endcsname{\def\PY@tc##1{\textcolor[rgb]{0.10,0.09,0.49}{##1}}}
\expandafter\def\csname PY@tok@vg\endcsname{\def\PY@tc##1{\textcolor[rgb]{0.10,0.09,0.49}{##1}}}
\expandafter\def\csname PY@tok@vi\endcsname{\def\PY@tc##1{\textcolor[rgb]{0.10,0.09,0.49}{##1}}}
\expandafter\def\csname PY@tok@vm\endcsname{\def\PY@tc##1{\textcolor[rgb]{0.10,0.09,0.49}{##1}}}
\expandafter\def\csname PY@tok@sa\endcsname{\def\PY@tc##1{\textcolor[rgb]{0.73,0.13,0.13}{##1}}}
\expandafter\def\csname PY@tok@sb\endcsname{\def\PY@tc##1{\textcolor[rgb]{0.73,0.13,0.13}{##1}}}
\expandafter\def\csname PY@tok@sc\endcsname{\def\PY@tc##1{\textcolor[rgb]{0.73,0.13,0.13}{##1}}}
\expandafter\def\csname PY@tok@dl\endcsname{\def\PY@tc##1{\textcolor[rgb]{0.73,0.13,0.13}{##1}}}
\expandafter\def\csname PY@tok@s2\endcsname{\def\PY@tc##1{\textcolor[rgb]{0.73,0.13,0.13}{##1}}}
\expandafter\def\csname PY@tok@sh\endcsname{\def\PY@tc##1{\textcolor[rgb]{0.73,0.13,0.13}{##1}}}
\expandafter\def\csname PY@tok@s1\endcsname{\def\PY@tc##1{\textcolor[rgb]{0.73,0.13,0.13}{##1}}}
\expandafter\def\csname PY@tok@mb\endcsname{\def\PY@tc##1{\textcolor[rgb]{0.40,0.40,0.40}{##1}}}
\expandafter\def\csname PY@tok@mf\endcsname{\def\PY@tc##1{\textcolor[rgb]{0.40,0.40,0.40}{##1}}}
\expandafter\def\csname PY@tok@mh\endcsname{\def\PY@tc##1{\textcolor[rgb]{0.40,0.40,0.40}{##1}}}
\expandafter\def\csname PY@tok@mi\endcsname{\def\PY@tc##1{\textcolor[rgb]{0.40,0.40,0.40}{##1}}}
\expandafter\def\csname PY@tok@il\endcsname{\def\PY@tc##1{\textcolor[rgb]{0.40,0.40,0.40}{##1}}}
\expandafter\def\csname PY@tok@mo\endcsname{\def\PY@tc##1{\textcolor[rgb]{0.40,0.40,0.40}{##1}}}
\expandafter\def\csname PY@tok@ch\endcsname{\let\PY@it=\textit\def\PY@tc##1{\textcolor[rgb]{0.25,0.50,0.50}{##1}}}
\expandafter\def\csname PY@tok@cm\endcsname{\let\PY@it=\textit\def\PY@tc##1{\textcolor[rgb]{0.25,0.50,0.50}{##1}}}
\expandafter\def\csname PY@tok@cpf\endcsname{\let\PY@it=\textit\def\PY@tc##1{\textcolor[rgb]{0.25,0.50,0.50}{##1}}}
\expandafter\def\csname PY@tok@c1\endcsname{\let\PY@it=\textit\def\PY@tc##1{\textcolor[rgb]{0.25,0.50,0.50}{##1}}}
\expandafter\def\csname PY@tok@cs\endcsname{\let\PY@it=\textit\def\PY@tc##1{\textcolor[rgb]{0.25,0.50,0.50}{##1}}}

\def\PYZbs{\char`\\}
\def\PYZus{\char`\_}
\def\PYZob{\char`\{}
\def\PYZcb{\char`\}}
\def\PYZca{\char`\^}
\def\PYZam{\char`\&}
\def\PYZlt{\char`\<}
\def\PYZgt{\char`\>}
\def\PYZsh{\char`\#}
\def\PYZpc{\char`\%}
\def\PYZdl{\char`\$}
\def\PYZhy{\char`\-}
\def\PYZsq{\char`\'}
\def\PYZdq{\char`\"}
\def\PYZti{\char`\~}
% for compatibility with earlier versions
\def\PYZat{@}
\def\PYZlb{[}
\def\PYZrb{]}
\makeatother


    % Exact colors from NB
    \definecolor{incolor}{rgb}{0.0, 0.0, 0.5}
    \definecolor{outcolor}{rgb}{0.545, 0.0, 0.0}



    
    % Prevent overflowing lines due to hard-to-break entities
    \sloppy 
    % Setup hyperref package
    \hypersetup{
      breaklinks=true,  % so long urls are correctly broken across lines
      colorlinks=true,
      urlcolor=urlcolor,
      linkcolor=linkcolor,
      citecolor=citecolor,
      }
    % Slightly bigger margins than the latex defaults
    
    \geometry{verbose,tmargin=1in,bmargin=1in,lmargin=1in,rmargin=1in}
    
    

    \begin{document}
    
    
    \maketitle
    
    

    
    \hypertarget{data-scientist}{%
\section{DATA SCIENTIST}\label{data-scientist}}

\textbf{In this tutorial, I only explain you what you need to be a data
scientist neither more nor less.}

Data scientist need to have these skills:

\begin{enumerate}
\def\labelenumi{\arabic{enumi}.}
\tightlist
\item
  Basic Tools: Like python, R or SQL. You do not need to know
  everything. What you only need is to learn how to use \textbf{python}
\item
  Basic Statistics: Like mean, median or standart deviation. If you know
  basic statistics, you can use \textbf{python} easily.
\item
  Data Munging: Working with messy and difficult data. Like a
  inconsistent date and string formatting. As you guess, \textbf{python}
  helps us.
\item
  Data Visualization: Title is actually explanatory. We will visualize
  the data with \textbf{python} like matplot and seaborn libraries.
\item
  Machine Learning: You do not need to understand math behind the
  machine learning technique. You only need is understanding basics of
  machine learning and learning how to implement it while using
  \textbf{python}.
\end{enumerate}

\hypertarget{as-a-summary-we-will-learn-python-to-be-data-scientist}{%
\subsubsection{As a summary we will learn python to be data scientist
!!!}\label{as-a-summary-we-will-learn-python-to-be-data-scientist}}

\textbf{Content:} 1. Introduction to Python: 1. Matplotlib 1.
Dictionaries 1. Pandas 1. Logic, control flow and filtering 1. Loop data
structures 1. Python Data Science Toolbox: 1. User defined function 1.
Scope 1. Nested function 1. Default and flexible arguments 1. Lambda
function 1. Anonymous function 1. Iterators 1. List comprehension 1.
Cleaning Data 1. Diagnose data for cleaning 1. Explotary data analysis
1. Visual exploratory data analysis 1. Tidy data 1. Pivoting data 1.
Concatenating data 1. Data types 1. Missing data and testing with assert
1. Pandas Foundation 1. Review of pandas 1. Building data frames from
scratch 1. Visual exploratory data analysis 1. Statistical explatory
data analysis 1. Indexing pandas time series 1. Resampling pandas time
series 1. Manipulating Data Frames with Pandas 1. Indexing data frames
1. Slicing data frames 1. Filtering data frames 1. Transforming data
frames 1. Index objects and labeled data 1. Hierarchical indexing 1.
Pivoting data frames 1. Stacking and unstacking data frames 1. Melting
data frames 1. Categoricals and groupby 1. Data Visualization 1.
Seaborn: https://www.kaggle.com/kanncaa1/seaborn-for-beginners 1. Bokeh:
https://www.kaggle.com/kanncaa1/interactive-bokeh-tutorial-part-1 1.
Bokeh: https://www.kaggle.com/kanncaa1/interactive-bokeh-tutorial-part-2
1. Machine Learning 1.
https://www.kaggle.com/kanncaa1/machine-learning-tutorial-for-beginners/
1. Deep Learning 1.
https://www.kaggle.com/kanncaa1/deep-learning-tutorial-for-beginners

    \begin{Verbatim}[commandchars=\\\{\}]
{\color{incolor}In [{\color{incolor}2}]:} \PY{c+c1}{\PYZsh{} This Python 3 environment comes with many helpful analytics libraries installed}
        \PY{c+c1}{\PYZsh{} It is defined by the kaggle/python docker image: https://github.com/kaggle/docker\PYZhy{}python}
        \PY{c+c1}{\PYZsh{} For example, here\PYZsq{}s several helpful packages to load in }
        
        \PY{k+kn}{import} \PY{n+nn}{numpy} \PY{k}{as} \PY{n+nn}{np} \PY{c+c1}{\PYZsh{} linear algebra}
        \PY{k+kn}{import} \PY{n+nn}{pandas} \PY{k}{as} \PY{n+nn}{pd} \PY{c+c1}{\PYZsh{} data processing, CSV file I/O (e.g. pd.read\PYZus{}csv)}
        \PY{k+kn}{import} \PY{n+nn}{matplotlib}\PY{n+nn}{.}\PY{n+nn}{pyplot} \PY{k}{as} \PY{n+nn}{plt}
        \PY{k+kn}{import} \PY{n+nn}{seaborn} \PY{k}{as} \PY{n+nn}{sns}
        
        \PY{c+c1}{\PYZsh{} Input data files are available in the \PYZdq{}../input/\PYZdq{} directory.}
        \PY{c+c1}{\PYZsh{} For example, running this (by clicking run or pressing Shift+Enter) will list the files in the input directory}
        
        \PY{k+kn}{from} \PY{n+nn}{subprocess} \PY{k}{import} \PY{n}{check\PYZus{}output}
        \PY{n+nb}{print}\PY{p}{(}\PY{n}{check\PYZus{}output}\PY{p}{(}\PY{p}{[}\PY{l+s+s2}{\PYZdq{}}\PY{l+s+s2}{ls}\PY{l+s+s2}{\PYZdq{}}\PY{p}{,} \PY{l+s+s2}{\PYZdq{}}\PY{l+s+s2}{../input}\PY{l+s+s2}{\PYZdq{}}\PY{p}{]}\PY{p}{)}\PY{o}{.}\PY{n}{decode}\PY{p}{(}\PY{l+s+s2}{\PYZdq{}}\PY{l+s+s2}{utf8}\PY{l+s+s2}{\PYZdq{}}\PY{p}{)}\PY{p}{)}
        
        \PY{c+c1}{\PYZsh{} Any results you write to the current directory are saved as output.}
\end{Verbatim}


    \begin{Verbatim}[commandchars=\\\{\}]
combats.csv
pokemon.csv
tests.csv


    \end{Verbatim}

    \begin{Verbatim}[commandchars=\\\{\}]
{\color{incolor}In [{\color{incolor}66}]:} \PY{n}{data} \PY{o}{=} \PY{n}{pd}\PY{o}{.}\PY{n}{read\PYZus{}csv}\PY{p}{(}\PY{l+s+s1}{\PYZsq{}}\PY{l+s+s1}{../input/pokemon.csv}\PY{l+s+s1}{\PYZsq{}}\PY{p}{)}
\end{Verbatim}


    \begin{Verbatim}[commandchars=\\\{\}]
{\color{incolor}In [{\color{incolor}3}]:} \PY{n}{data}\PY{o}{.}\PY{n}{head}\PY{p}{(}\PY{p}{)}
\end{Verbatim}


\begin{Verbatim}[commandchars=\\\{\}]
{\color{outcolor}Out[{\color{outcolor}3}]:}    \#           Name Type 1  Type 2  HP  Attack  Defense  Sp. Atk  Sp. Def  \textbackslash{}
        0  1      Bulbasaur  Grass  Poison  45      49       49       65       65   
        1  2        Ivysaur  Grass  Poison  60      62       63       80       80   
        2  3       Venusaur  Grass  Poison  80      82       83      100      100   
        3  4  Mega Venusaur  Grass  Poison  80     100      123      122      120   
        4  5     Charmander   Fire     NaN  39      52       43       60       50   
        
           Speed  Generation  Legendary  
        0     45           1      False  
        1     60           1      False  
        2     80           1      False  
        3     80           1      False  
        4     65           1      False  
\end{Verbatim}
            
    \begin{Verbatim}[commandchars=\\\{\}]
{\color{incolor}In [{\color{incolor}4}]:} \PY{n}{data1} \PY{o}{=} \PY{n}{data}\PY{p}{[}\PY{n}{data}\PY{p}{[}\PY{l+s+s1}{\PYZsq{}}\PY{l+s+s1}{Legendary}\PY{l+s+s1}{\PYZsq{}}\PY{p}{]}\PY{o}{==}\PY{k+kc}{True}\PY{p}{]}\PY{p}{[}\PY{l+s+s1}{\PYZsq{}}\PY{l+s+s1}{Attack}\PY{l+s+s1}{\PYZsq{}}\PY{p}{]}
        \PY{n}{data2} \PY{o}{=} \PY{n}{data}\PY{p}{[}\PY{n}{data}\PY{p}{[}\PY{l+s+s1}{\PYZsq{}}\PY{l+s+s1}{Legendary}\PY{l+s+s1}{\PYZsq{}}\PY{p}{]}\PY{o}{==}\PY{k+kc}{False}\PY{p}{]}\PY{p}{[}\PY{l+s+s1}{\PYZsq{}}\PY{l+s+s1}{Attack}\PY{l+s+s1}{\PYZsq{}}\PY{p}{]}
        \PY{n}{fig} \PY{o}{=} \PY{n}{plt}\PY{o}{.}\PY{n}{figure}\PY{p}{(}\PY{l+m+mi}{1}\PY{p}{,} \PY{n}{figsize}\PY{o}{=}\PY{p}{(}\PY{l+m+mi}{18}\PY{p}{,}\PY{l+m+mi}{6}\PY{p}{)}\PY{p}{)}
        \PY{n}{ax1} \PY{o}{=} \PY{n}{fig}\PY{o}{.}\PY{n}{add\PYZus{}subplot}\PY{p}{(}\PY{l+m+mi}{1}\PY{p}{,}\PY{l+m+mi}{2}\PY{p}{,}\PY{l+m+mi}{1}\PY{p}{)}
        \PY{n}{ax2} \PY{o}{=} \PY{n}{fig}\PY{o}{.}\PY{n}{add\PYZus{}subplot}\PY{p}{(}\PY{l+m+mi}{1}\PY{p}{,}\PY{l+m+mi}{2}\PY{p}{,}\PY{l+m+mi}{2}\PY{p}{)}
        \PY{c+c1}{\PYZsh{} ax1.hist(data1)}
        \PY{c+c1}{\PYZsh{} ax2.hist(data2)}
        
        \PY{c+c1}{\PYZsh{} plt.hist([data1,data2], stacked=True)}
        \PY{c+c1}{\PYZsh{} plt.legend([\PYZsq{}True\PYZsq{},\PYZsq{}False\PYZsq{}])}
        \PY{c+c1}{\PYZsh{} plt.show()}
        
        \PY{c+c1}{\PYZsh{}data.groupby(\PYZdq{}Legendary\PYZdq{})[\PYZdq{}Attack\PYZdq{}].plot(kind=\PYZdq{}kde\PYZdq{}, legend=True)}
        
        \PY{n}{data1}\PY{o}{.}\PY{n}{plot}\PY{p}{(}\PY{n}{kind}\PY{o}{=}\PY{l+s+s2}{\PYZdq{}}\PY{l+s+s2}{hist}\PY{l+s+s2}{\PYZdq{}}\PY{p}{,} \PY{n}{label}\PY{o}{=}\PY{l+s+s2}{\PYZdq{}}\PY{l+s+s2}{True}\PY{l+s+s2}{\PYZdq{}}\PY{p}{,}\PY{n}{ax}\PY{o}{=}\PY{n}{ax1}\PY{p}{)}
        \PY{n}{data2}\PY{o}{.}\PY{n}{plot}\PY{p}{(}\PY{n}{kind}\PY{o}{=}\PY{l+s+s2}{\PYZdq{}}\PY{l+s+s2}{kde}\PY{l+s+s2}{\PYZdq{}}\PY{p}{,} \PY{n}{label}\PY{o}{=}\PY{l+s+s2}{\PYZdq{}}\PY{l+s+s2}{False}\PY{l+s+s2}{\PYZdq{}}\PY{p}{,}\PY{n}{ax}\PY{o}{=}\PY{n}{ax2}\PY{p}{)}
        \PY{n}{ax1}\PY{o}{.}\PY{n}{legend}\PY{p}{(}\PY{n}{loc}\PY{o}{=}\PY{l+s+s1}{\PYZsq{}}\PY{l+s+s1}{upper right}\PY{l+s+s1}{\PYZsq{}}\PY{p}{)} 
        \PY{n}{ax2}\PY{o}{.}\PY{n}{legend}\PY{p}{(}\PY{n}{loc}\PY{o}{=}\PY{l+s+s1}{\PYZsq{}}\PY{l+s+s1}{upper right}\PY{l+s+s1}{\PYZsq{}}\PY{p}{)}
\end{Verbatim}


\begin{Verbatim}[commandchars=\\\{\}]
{\color{outcolor}Out[{\color{outcolor}4}]:} <matplotlib.legend.Legend at 0x109f4beb8>
\end{Verbatim}
            
    \begin{center}
    \adjustimage{max size={0.9\linewidth}{0.9\paperheight}}{output_4_1.png}
    \end{center}
    { \hspace*{\fill} \\}
    
    \begin{Verbatim}[commandchars=\\\{\}]
{\color{incolor}In [{\color{incolor}5}]:} \PY{c+c1}{\PYZsh{}correlation map}
        \PY{n}{f}\PY{p}{,}\PY{n}{ax} \PY{o}{=} \PY{n}{plt}\PY{o}{.}\PY{n}{subplots}\PY{p}{(}\PY{n}{figsize}\PY{o}{=}\PY{p}{(}\PY{l+m+mi}{18}\PY{p}{,} \PY{l+m+mi}{18}\PY{p}{)}\PY{p}{)}
        \PY{n}{sns}\PY{o}{.}\PY{n}{heatmap}\PY{p}{(}\PY{n}{data}\PY{o}{.}\PY{n}{corr}\PY{p}{(}\PY{p}{)}\PY{p}{,} \PY{n}{annot}\PY{o}{=}\PY{k+kc}{True}\PY{p}{,} \PY{n}{linewidths}\PY{o}{=}\PY{o}{.}\PY{l+m+mi}{5}\PY{p}{,} \PY{n}{fmt}\PY{o}{=} \PY{l+s+s1}{\PYZsq{}}\PY{l+s+s1}{.1f}\PY{l+s+s1}{\PYZsq{}}\PY{p}{,}\PY{n}{ax}\PY{o}{=}\PY{n}{ax}\PY{p}{)}
\end{Verbatim}


\begin{Verbatim}[commandchars=\\\{\}]
{\color{outcolor}Out[{\color{outcolor}5}]:} <matplotlib.axes.\_subplots.AxesSubplot at 0x103521eb8>
\end{Verbatim}
            
    \begin{center}
    \adjustimage{max size={0.9\linewidth}{0.9\paperheight}}{output_5_1.png}
    \end{center}
    { \hspace*{\fill} \\}
    
    \begin{Verbatim}[commandchars=\\\{\}]
{\color{incolor}In [{\color{incolor}6}]:} \PY{n}{data}\PY{o}{.}\PY{n}{head}\PY{p}{(}\PY{l+m+mi}{10}\PY{p}{)}
\end{Verbatim}


\begin{Verbatim}[commandchars=\\\{\}]
{\color{outcolor}Out[{\color{outcolor}6}]:}     \#              Name Type 1  Type 2  HP  Attack  Defense  Sp. Atk  Sp. Def  \textbackslash{}
        0   1         Bulbasaur  Grass  Poison  45      49       49       65       65   
        1   2           Ivysaur  Grass  Poison  60      62       63       80       80   
        2   3          Venusaur  Grass  Poison  80      82       83      100      100   
        3   4     Mega Venusaur  Grass  Poison  80     100      123      122      120   
        4   5        Charmander   Fire     NaN  39      52       43       60       50   
        5   6        Charmeleon   Fire     NaN  58      64       58       80       65   
        6   7         Charizard   Fire  Flying  78      84       78      109       85   
        7   8  Mega Charizard X   Fire  Dragon  78     130      111      130       85   
        8   9  Mega Charizard Y   Fire  Flying  78     104       78      159      115   
        9  10          Squirtle  Water     NaN  44      48       65       50       64   
        
           Speed  Generation  Legendary  
        0     45           1      False  
        1     60           1      False  
        2     80           1      False  
        3     80           1      False  
        4     65           1      False  
        5     80           1      False  
        6    100           1      False  
        7    100           1      False  
        8    100           1      False  
        9     43           1      False  
\end{Verbatim}
            
    \begin{Verbatim}[commandchars=\\\{\}]
{\color{incolor}In [{\color{incolor}7}]:} \PY{n}{data}\PY{o}{.}\PY{n}{info}\PY{p}{(}\PY{p}{)}
\end{Verbatim}


    \begin{Verbatim}[commandchars=\\\{\}]
<class 'pandas.core.frame.DataFrame'>
RangeIndex: 800 entries, 0 to 799
Data columns (total 12 columns):
\#             800 non-null int64
Name          799 non-null object
Type 1        800 non-null object
Type 2        414 non-null object
HP            800 non-null int64
Attack        800 non-null int64
Defense       800 non-null int64
Sp. Atk       800 non-null int64
Sp. Def       800 non-null int64
Speed         800 non-null int64
Generation    800 non-null int64
Legendary     800 non-null bool
dtypes: bool(1), int64(8), object(3)
memory usage: 69.6+ KB

    \end{Verbatim}

    \begin{Verbatim}[commandchars=\\\{\}]
{\color{incolor}In [{\color{incolor}8}]:} \PY{n}{data}\PY{o}{.}\PY{n}{columns}
\end{Verbatim}


\begin{Verbatim}[commandchars=\\\{\}]
{\color{outcolor}Out[{\color{outcolor}8}]:} Index(['\#', 'Name', 'Type 1', 'Type 2', 'HP', 'Attack', 'Defense', 'Sp. Atk',
               'Sp. Def', 'Speed', 'Generation', 'Legendary'],
              dtype='object')
\end{Verbatim}
            
    \hypertarget{introduction-to-python}{%
\section{1. INTRODUCTION TO PYTHON}\label{introduction-to-python}}

    \hypertarget{matplotlib}{%
\subsubsection{MATPLOTLIB}\label{matplotlib}}

Matplot is a python library that help us to plot data. The easiest and
basic plots are line, scatter and histogram plots. * Line plot is better
when x axis is time. * Scatter is better when there is correlation
between two variables * Histogram is better when we need to see
distribution of numerical data. * Customization: Colors,labels,thickness
of line, title, opacity, grid, figsize, ticks of axis and linestyle

    \begin{Verbatim}[commandchars=\\\{\}]
{\color{incolor}In [{\color{incolor}9}]:} \PY{c+c1}{\PYZsh{} Line Plot}
        \PY{c+c1}{\PYZsh{} color = color, label = label, linewidth = width of line, alpha = opacity, grid = grid, linestyle = sytle of line}
        \PY{n}{data}\PY{o}{.}\PY{n}{Speed}\PY{o}{.}\PY{n}{plot}\PY{p}{(}\PY{n}{kind} \PY{o}{=} \PY{l+s+s1}{\PYZsq{}}\PY{l+s+s1}{line}\PY{l+s+s1}{\PYZsq{}}\PY{p}{,} \PY{n}{color} \PY{o}{=} \PY{l+s+s1}{\PYZsq{}}\PY{l+s+s1}{g}\PY{l+s+s1}{\PYZsq{}}\PY{p}{,}\PY{n}{label} \PY{o}{=} \PY{l+s+s1}{\PYZsq{}}\PY{l+s+s1}{Speed}\PY{l+s+s1}{\PYZsq{}}\PY{p}{,}\PY{n}{linewidth}\PY{o}{=}\PY{l+m+mi}{1}\PY{p}{,}\PY{n}{alpha} \PY{o}{=} \PY{l+m+mf}{0.5}\PY{p}{,}\PY{n}{grid} \PY{o}{=} \PY{k+kc}{True}\PY{p}{,}\PY{n}{linestyle} \PY{o}{=} \PY{l+s+s1}{\PYZsq{}}\PY{l+s+s1}{:}\PY{l+s+s1}{\PYZsq{}}\PY{p}{)}
        \PY{n}{data}\PY{o}{.}\PY{n}{Defense}\PY{o}{.}\PY{n}{plot}\PY{p}{(}\PY{n}{kind}\PY{o}{=}\PY{l+s+s2}{\PYZdq{}}\PY{l+s+s2}{line}\PY{l+s+s2}{\PYZdq{}}\PY{p}{,} \PY{n}{color} \PY{o}{=} \PY{l+s+s1}{\PYZsq{}}\PY{l+s+s1}{r}\PY{l+s+s1}{\PYZsq{}}\PY{p}{,}\PY{n}{label} \PY{o}{=} \PY{l+s+s1}{\PYZsq{}}\PY{l+s+s1}{Defense}\PY{l+s+s1}{\PYZsq{}}\PY{p}{,}\PY{n}{linewidth}\PY{o}{=}\PY{l+m+mi}{1}\PY{p}{,} \PY{n}{alpha} \PY{o}{=} \PY{l+m+mf}{0.5}\PY{p}{,}\PY{n}{grid} \PY{o}{=} \PY{k+kc}{True}\PY{p}{,}\PY{n}{linestyle} \PY{o}{=} \PY{l+s+s1}{\PYZsq{}}\PY{l+s+s1}{\PYZhy{}.}\PY{l+s+s1}{\PYZsq{}}\PY{p}{)}
        \PY{n}{plt}\PY{o}{.}\PY{n}{legend}\PY{p}{(}\PY{n}{loc}\PY{o}{=}\PY{l+s+s1}{\PYZsq{}}\PY{l+s+s1}{upper right}\PY{l+s+s1}{\PYZsq{}}\PY{p}{)}     \PY{c+c1}{\PYZsh{} legend = puts label into plot}
        \PY{n}{plt}\PY{o}{.}\PY{n}{xlabel}\PY{p}{(}\PY{l+s+s1}{\PYZsq{}}\PY{l+s+s1}{x axis}\PY{l+s+s1}{\PYZsq{}}\PY{p}{)}              \PY{c+c1}{\PYZsh{} label = name of label}
        \PY{n}{plt}\PY{o}{.}\PY{n}{ylabel}\PY{p}{(}\PY{l+s+s1}{\PYZsq{}}\PY{l+s+s1}{y axis}\PY{l+s+s1}{\PYZsq{}}\PY{p}{)}
        \PY{n}{plt}\PY{o}{.}\PY{n}{title}\PY{p}{(}\PY{l+s+s1}{\PYZsq{}}\PY{l+s+s1}{Line Plot}\PY{l+s+s1}{\PYZsq{}}\PY{p}{)}            \PY{c+c1}{\PYZsh{} title = title of plot}
\end{Verbatim}


\begin{Verbatim}[commandchars=\\\{\}]
{\color{outcolor}Out[{\color{outcolor}9}]:} Text(0.5,1,'Line Plot')
\end{Verbatim}
            
    \begin{center}
    \adjustimage{max size={0.9\linewidth}{0.9\paperheight}}{output_11_1.png}
    \end{center}
    { \hspace*{\fill} \\}
    
    \begin{Verbatim}[commandchars=\\\{\}]
{\color{incolor}In [{\color{incolor}10}]:} \PY{c+c1}{\PYZsh{} Scatter Plot }
         \PY{c+c1}{\PYZsh{} x = attack, y = defense}
         \PY{n}{data}\PY{o}{.}\PY{n}{plot}\PY{p}{(}\PY{n}{kind}\PY{o}{=}\PY{l+s+s1}{\PYZsq{}}\PY{l+s+s1}{scatter}\PY{l+s+s1}{\PYZsq{}}\PY{p}{,} \PY{n}{x}\PY{o}{=}\PY{l+s+s1}{\PYZsq{}}\PY{l+s+s1}{Attack}\PY{l+s+s1}{\PYZsq{}}\PY{p}{,} \PY{n}{y}\PY{o}{=}\PY{l+s+s1}{\PYZsq{}}\PY{l+s+s1}{Defense}\PY{l+s+s1}{\PYZsq{}}\PY{p}{,}\PY{n}{alpha} \PY{o}{=} \PY{l+m+mf}{0.5}\PY{p}{,}\PY{n}{color} \PY{o}{=} \PY{l+s+s1}{\PYZsq{}}\PY{l+s+s1}{red}\PY{l+s+s1}{\PYZsq{}}\PY{p}{)}
         \PY{n}{plt}\PY{o}{.}\PY{n}{xlabel}\PY{p}{(}\PY{l+s+s1}{\PYZsq{}}\PY{l+s+s1}{Attack}\PY{l+s+s1}{\PYZsq{}}\PY{p}{)}              \PY{c+c1}{\PYZsh{} label = name of label}
         \PY{n}{plt}\PY{o}{.}\PY{n}{ylabel}\PY{p}{(}\PY{l+s+s1}{\PYZsq{}}\PY{l+s+s1}{Defence}\PY{l+s+s1}{\PYZsq{}}\PY{p}{)}
         \PY{n}{plt}\PY{o}{.}\PY{n}{title}\PY{p}{(}\PY{l+s+s1}{\PYZsq{}}\PY{l+s+s1}{Attack Defense Scatter Plot}\PY{l+s+s1}{\PYZsq{}}\PY{p}{)}            \PY{c+c1}{\PYZsh{} title = title of plot}
\end{Verbatim}


\begin{Verbatim}[commandchars=\\\{\}]
{\color{outcolor}Out[{\color{outcolor}10}]:} Text(0.5,1,'Attack Defense Scatter Plot')
\end{Verbatim}
            
    \begin{center}
    \adjustimage{max size={0.9\linewidth}{0.9\paperheight}}{output_12_1.png}
    \end{center}
    { \hspace*{\fill} \\}
    
    \begin{Verbatim}[commandchars=\\\{\}]
{\color{incolor}In [{\color{incolor}11}]:} \PY{c+c1}{\PYZsh{} Histogram}
         \PY{c+c1}{\PYZsh{} bins = number of bar in figure}
         \PY{n}{data}\PY{o}{.}\PY{n}{Speed}\PY{o}{.}\PY{n}{plot}\PY{p}{(}\PY{n}{kind} \PY{o}{=} \PY{l+s+s1}{\PYZsq{}}\PY{l+s+s1}{hist}\PY{l+s+s1}{\PYZsq{}}\PY{p}{,}\PY{n}{bins} \PY{o}{=} \PY{l+m+mi}{50}\PY{p}{,}\PY{n}{figsize} \PY{o}{=} \PY{p}{(}\PY{l+m+mi}{18}\PY{p}{,}\PY{l+m+mi}{6}\PY{p}{)}\PY{p}{)}
\end{Verbatim}


\begin{Verbatim}[commandchars=\\\{\}]
{\color{outcolor}Out[{\color{outcolor}11}]:} <matplotlib.axes.\_subplots.AxesSubplot at 0x10b871128>
\end{Verbatim}
            
    \begin{center}
    \adjustimage{max size={0.9\linewidth}{0.9\paperheight}}{output_13_1.png}
    \end{center}
    { \hspace*{\fill} \\}
    
    \begin{Verbatim}[commandchars=\\\{\}]
{\color{incolor}In [{\color{incolor}12}]:} \PY{c+c1}{\PYZsh{} clf() = cleans it up again you can start a fresh}
         \PY{n}{data}\PY{o}{.}\PY{n}{Speed}\PY{o}{.}\PY{n}{plot}\PY{p}{(}\PY{n}{kind} \PY{o}{=} \PY{l+s+s1}{\PYZsq{}}\PY{l+s+s1}{hist}\PY{l+s+s1}{\PYZsq{}}\PY{p}{,}\PY{n}{bins} \PY{o}{=} \PY{l+m+mi}{50}\PY{p}{)}
         \PY{n}{plt}\PY{o}{.}\PY{n}{clf}\PY{p}{(}\PY{p}{)}
         \PY{c+c1}{\PYZsh{} We cannot see plot due to clf()}
\end{Verbatim}


    
    \begin{verbatim}
<Figure size 432x288 with 0 Axes>
    \end{verbatim}

    
    \hypertarget{dictionary}{%
\subsubsection{DICTIONARY}\label{dictionary}}

Why we need dictionary? * It has `key' and `value' * Faster than lists
What is key and value. Example: * dictionary = \{`spain' : `madrid'\} *
Key is spain. * Values is madrid. \textbf{It's that easy.} Lets practice
some other properties like keys(), values(), update, add, check, remove
key, remove all entries and remove dicrionary.

    \begin{Verbatim}[commandchars=\\\{\}]
{\color{incolor}In [{\color{incolor}13}]:} \PY{c+c1}{\PYZsh{}create dictionary and look its keys and values}
         \PY{n}{dictionary} \PY{o}{=} \PY{p}{\PYZob{}}\PY{l+s+s1}{\PYZsq{}}\PY{l+s+s1}{spain}\PY{l+s+s1}{\PYZsq{}} \PY{p}{:} \PY{l+s+s1}{\PYZsq{}}\PY{l+s+s1}{madrid}\PY{l+s+s1}{\PYZsq{}}\PY{p}{,}\PY{l+s+s1}{\PYZsq{}}\PY{l+s+s1}{usa}\PY{l+s+s1}{\PYZsq{}} \PY{p}{:} \PY{l+s+s1}{\PYZsq{}}\PY{l+s+s1}{vegas}\PY{l+s+s1}{\PYZsq{}}\PY{p}{\PYZcb{}}
         \PY{n+nb}{print}\PY{p}{(}\PY{n}{dictionary}\PY{o}{.}\PY{n}{keys}\PY{p}{(}\PY{p}{)}\PY{p}{)}
         \PY{n+nb}{print}\PY{p}{(}\PY{n}{dictionary}\PY{o}{.}\PY{n}{values}\PY{p}{(}\PY{p}{)}\PY{p}{)}
\end{Verbatim}


    \begin{Verbatim}[commandchars=\\\{\}]
dict\_keys(['spain', 'usa'])
dict\_values(['madrid', 'vegas'])

    \end{Verbatim}

    \begin{Verbatim}[commandchars=\\\{\}]
{\color{incolor}In [{\color{incolor}14}]:} \PY{c+c1}{\PYZsh{} Keys have to be immutable objects like string, boolean, float, integer or tubles}
         \PY{c+c1}{\PYZsh{} List is not immutable}
         \PY{c+c1}{\PYZsh{} Keys are unique}
         \PY{n}{dictionary}\PY{p}{[}\PY{l+s+s1}{\PYZsq{}}\PY{l+s+s1}{spain}\PY{l+s+s1}{\PYZsq{}}\PY{p}{]} \PY{o}{=} \PY{l+s+s2}{\PYZdq{}}\PY{l+s+s2}{barcelona}\PY{l+s+s2}{\PYZdq{}}    \PY{c+c1}{\PYZsh{} update existing entry}
         \PY{n+nb}{print}\PY{p}{(}\PY{n}{dictionary}\PY{p}{)}
         \PY{n}{dictionary}\PY{p}{[}\PY{l+s+s1}{\PYZsq{}}\PY{l+s+s1}{france}\PY{l+s+s1}{\PYZsq{}}\PY{p}{]} \PY{o}{=} \PY{l+s+s2}{\PYZdq{}}\PY{l+s+s2}{paris}\PY{l+s+s2}{\PYZdq{}}       \PY{c+c1}{\PYZsh{} Add new entry}
         \PY{n+nb}{print}\PY{p}{(}\PY{n}{dictionary}\PY{p}{)}
         \PY{k}{del} \PY{n}{dictionary}\PY{p}{[}\PY{l+s+s1}{\PYZsq{}}\PY{l+s+s1}{spain}\PY{l+s+s1}{\PYZsq{}}\PY{p}{]}              \PY{c+c1}{\PYZsh{} remove entry with key \PYZsq{}spain\PYZsq{}}
         \PY{n+nb}{print}\PY{p}{(}\PY{n}{dictionary}\PY{p}{)}
         \PY{n+nb}{print}\PY{p}{(}\PY{l+s+s1}{\PYZsq{}}\PY{l+s+s1}{france}\PY{l+s+s1}{\PYZsq{}} \PY{o+ow}{in} \PY{n}{dictionary}\PY{p}{)}        \PY{c+c1}{\PYZsh{} check include or not}
         \PY{n}{dictionary}\PY{o}{.}\PY{n}{clear}\PY{p}{(}\PY{p}{)}                   \PY{c+c1}{\PYZsh{} remove all entries in dict}
         \PY{n+nb}{print}\PY{p}{(}\PY{n}{dictionary}\PY{p}{)}
\end{Verbatim}


    \begin{Verbatim}[commandchars=\\\{\}]
\{'spain': 'barcelona', 'usa': 'vegas'\}
\{'spain': 'barcelona', 'usa': 'vegas', 'france': 'paris'\}
\{'usa': 'vegas', 'france': 'paris'\}
True
\{\}

    \end{Verbatim}

    \begin{Verbatim}[commandchars=\\\{\}]
{\color{incolor}In [{\color{incolor}15}]:} \PY{c+c1}{\PYZsh{} In order to run all code you need to take comment this line}
         \PY{c+c1}{\PYZsh{} del dictionary ;        \PYZsh{} delete entire dictionary     }
         \PY{n+nb}{print}\PY{p}{(}\PY{n}{dictionary}\PY{p}{)}       \PY{c+c1}{\PYZsh{} it gives error because dictionary is deleted}
\end{Verbatim}


    \begin{Verbatim}[commandchars=\\\{\}]
\{\}

    \end{Verbatim}

    \hypertarget{pandas}{%
\subsubsection{PANDAS}\label{pandas}}

What we need to know about pandas? * CSV: comma - separated values

    \begin{Verbatim}[commandchars=\\\{\}]
{\color{incolor}In [{\color{incolor}16}]:} \PY{n}{data} \PY{o}{=} \PY{n}{pd}\PY{o}{.}\PY{n}{read\PYZus{}csv}\PY{p}{(}\PY{l+s+s1}{\PYZsq{}}\PY{l+s+s1}{../input/pokemon.csv}\PY{l+s+s1}{\PYZsq{}}\PY{p}{)}
\end{Verbatim}


    \begin{Verbatim}[commandchars=\\\{\}]
{\color{incolor}In [{\color{incolor}17}]:} \PY{n}{series} \PY{o}{=} \PY{n}{data}\PY{p}{[}\PY{l+s+s1}{\PYZsq{}}\PY{l+s+s1}{Defense}\PY{l+s+s1}{\PYZsq{}}\PY{p}{]}        \PY{c+c1}{\PYZsh{} data[\PYZsq{}Defense\PYZsq{}] = series}
         \PY{n+nb}{print}\PY{p}{(}\PY{n+nb}{type}\PY{p}{(}\PY{n}{series}\PY{p}{)}\PY{p}{)}
         \PY{n}{data\PYZus{}frame} \PY{o}{=} \PY{n}{data}\PY{p}{[}\PY{p}{[}\PY{l+s+s1}{\PYZsq{}}\PY{l+s+s1}{Defense}\PY{l+s+s1}{\PYZsq{}}\PY{p}{]}\PY{p}{]}  \PY{c+c1}{\PYZsh{} data[[\PYZsq{}Defense\PYZsq{}]] = data frame}
         \PY{n+nb}{print}\PY{p}{(}\PY{n+nb}{type}\PY{p}{(}\PY{n}{data\PYZus{}frame}\PY{p}{)}\PY{p}{)}
\end{Verbatim}


    \begin{Verbatim}[commandchars=\\\{\}]
<class 'pandas.core.series.Series'>
<class 'pandas.core.frame.DataFrame'>

    \end{Verbatim}

    Before continue with pandas, we need to learn \textbf{logic, control
flow} and \textbf{filtering.} Comparison operator: ==, \textless{},
\textgreater{}, \textless{}= Boolean operators: and, or ,not Filtering
pandas

    \begin{Verbatim}[commandchars=\\\{\}]
{\color{incolor}In [{\color{incolor}18}]:} \PY{c+c1}{\PYZsh{} Comparison operator}
         \PY{n+nb}{print}\PY{p}{(}\PY{l+m+mi}{3} \PY{o}{\PYZgt{}} \PY{l+m+mi}{2}\PY{p}{)}
         \PY{n+nb}{print}\PY{p}{(}\PY{l+m+mi}{3}\PY{o}{!=}\PY{l+m+mi}{2}\PY{p}{)}
         \PY{c+c1}{\PYZsh{} Boolean operators}
         \PY{n+nb}{print}\PY{p}{(}\PY{k+kc}{True} \PY{o+ow}{and} \PY{k+kc}{False}\PY{p}{)}
         \PY{n+nb}{print}\PY{p}{(}\PY{k+kc}{True} \PY{o+ow}{or} \PY{k+kc}{False}\PY{p}{)}
\end{Verbatim}


    \begin{Verbatim}[commandchars=\\\{\}]
True
True
False
True

    \end{Verbatim}

    \begin{Verbatim}[commandchars=\\\{\}]
{\color{incolor}In [{\color{incolor}19}]:} \PY{c+c1}{\PYZsh{} 1 \PYZhy{} Filtering Pandas data frame}
         \PY{n}{x} \PY{o}{=} \PY{n}{data}\PY{p}{[}\PY{l+s+s1}{\PYZsq{}}\PY{l+s+s1}{Defense}\PY{l+s+s1}{\PYZsq{}}\PY{p}{]}\PY{o}{\PYZgt{}}\PY{l+m+mi}{200}     \PY{c+c1}{\PYZsh{} There are only 3 pokemons who have higher defense value than 200}
         \PY{n}{data}\PY{p}{[}\PY{n}{x}\PY{p}{]}
\end{Verbatim}


\begin{Verbatim}[commandchars=\\\{\}]
{\color{outcolor}Out[{\color{outcolor}19}]:}        \#          Name Type 1  Type 2  HP  Attack  Defense  Sp. Atk  Sp. Def  \textbackslash{}
         224  225  Mega Steelix  Steel  Ground  75     125      230       55       95   
         230  231       Shuckle    Bug    Rock  20      10      230       10      230   
         333  334   Mega Aggron  Steel     NaN  70     140      230       60       80   
         
              Speed  Generation  Legendary  
         224     30           2      False  
         230      5           2      False  
         333     50           3      False  
\end{Verbatim}
            
    \begin{Verbatim}[commandchars=\\\{\}]
{\color{incolor}In [{\color{incolor}20}]:} \PY{c+c1}{\PYZsh{} 2 \PYZhy{} Filtering pandas with logical\PYZus{}and}
         \PY{c+c1}{\PYZsh{} There are only 2 pokemons who have higher defence value than 2oo and higher attack value than 100}
         \PY{n}{data}\PY{p}{[}\PY{n}{np}\PY{o}{.}\PY{n}{logical\PYZus{}and}\PY{p}{(}\PY{n}{data}\PY{p}{[}\PY{l+s+s1}{\PYZsq{}}\PY{l+s+s1}{Defense}\PY{l+s+s1}{\PYZsq{}}\PY{p}{]}\PY{o}{\PYZgt{}}\PY{l+m+mi}{200}\PY{p}{,} \PY{n}{data}\PY{p}{[}\PY{l+s+s1}{\PYZsq{}}\PY{l+s+s1}{Attack}\PY{l+s+s1}{\PYZsq{}}\PY{p}{]}\PY{o}{\PYZgt{}}\PY{l+m+mi}{100} \PY{p}{)}\PY{p}{]}
\end{Verbatim}


\begin{Verbatim}[commandchars=\\\{\}]
{\color{outcolor}Out[{\color{outcolor}20}]:}        \#          Name Type 1  Type 2  HP  Attack  Defense  Sp. Atk  Sp. Def  \textbackslash{}
         224  225  Mega Steelix  Steel  Ground  75     125      230       55       95   
         333  334   Mega Aggron  Steel     NaN  70     140      230       60       80   
         
              Speed  Generation  Legendary  
         224     30           2      False  
         333     50           3      False  
\end{Verbatim}
            
    \begin{Verbatim}[commandchars=\\\{\}]
{\color{incolor}In [{\color{incolor}21}]:} \PY{c+c1}{\PYZsh{} This is also same with previous code line. Therefore we can also use \PYZsq{}\PYZam{}\PYZsq{} for filtering.}
         \PY{n}{data}\PY{p}{[}\PY{p}{(}\PY{n}{data}\PY{p}{[}\PY{l+s+s1}{\PYZsq{}}\PY{l+s+s1}{Defense}\PY{l+s+s1}{\PYZsq{}}\PY{p}{]}\PY{o}{\PYZgt{}}\PY{l+m+mi}{200}\PY{p}{)} \PY{o}{\PYZam{}} \PY{p}{(}\PY{n}{data}\PY{p}{[}\PY{l+s+s1}{\PYZsq{}}\PY{l+s+s1}{Attack}\PY{l+s+s1}{\PYZsq{}}\PY{p}{]}\PY{o}{\PYZgt{}}\PY{l+m+mi}{100}\PY{p}{)}\PY{p}{]}
\end{Verbatim}


\begin{Verbatim}[commandchars=\\\{\}]
{\color{outcolor}Out[{\color{outcolor}21}]:}        \#          Name Type 1  Type 2  HP  Attack  Defense  Sp. Atk  Sp. Def  \textbackslash{}
         224  225  Mega Steelix  Steel  Ground  75     125      230       55       95   
         333  334   Mega Aggron  Steel     NaN  70     140      230       60       80   
         
              Speed  Generation  Legendary  
         224     30           2      False  
         333     50           3      False  
\end{Verbatim}
            
    \hypertarget{while-and-for-loops}{%
\subsubsection{WHILE and FOR LOOPS}\label{while-and-for-loops}}

We will learn most basic while and for loops

    \begin{Verbatim}[commandchars=\\\{\}]
{\color{incolor}In [{\color{incolor}22}]:} \PY{c+c1}{\PYZsh{} Stay in loop if condition( i is not equal 5) is true}
         \PY{n}{i} \PY{o}{=} \PY{l+m+mi}{0}
         \PY{k}{while} \PY{n}{i} \PY{o}{!=} \PY{l+m+mi}{5} \PY{p}{:}
             \PY{n+nb}{print}\PY{p}{(}\PY{l+s+s1}{\PYZsq{}}\PY{l+s+s1}{i is: }\PY{l+s+s1}{\PYZsq{}}\PY{p}{,}\PY{n}{i}\PY{p}{)}
             \PY{n}{i} \PY{o}{+}\PY{o}{=}\PY{l+m+mi}{1} 
         \PY{n+nb}{print}\PY{p}{(}\PY{n}{i}\PY{p}{,}\PY{l+s+s1}{\PYZsq{}}\PY{l+s+s1}{ is equal to 5}\PY{l+s+s1}{\PYZsq{}}\PY{p}{)}
\end{Verbatim}


    \begin{Verbatim}[commandchars=\\\{\}]
i is:  0
i is:  1
i is:  2
i is:  3
i is:  4
5  is equal to 5

    \end{Verbatim}

    \begin{Verbatim}[commandchars=\\\{\}]
{\color{incolor}In [{\color{incolor}23}]:} \PY{c+c1}{\PYZsh{} Stay in loop if condition( i is not equal 5) is true}
         \PY{n}{lis} \PY{o}{=} \PY{p}{[}\PY{l+m+mi}{1}\PY{p}{,}\PY{l+m+mi}{2}\PY{p}{,}\PY{l+m+mi}{3}\PY{p}{,}\PY{l+m+mi}{4}\PY{p}{,}\PY{l+m+mi}{5}\PY{p}{]}
         \PY{k}{for} \PY{n}{i} \PY{o+ow}{in} \PY{n}{lis}\PY{p}{:}
             \PY{n+nb}{print}\PY{p}{(}\PY{l+s+s1}{\PYZsq{}}\PY{l+s+s1}{i is: }\PY{l+s+s1}{\PYZsq{}}\PY{p}{,}\PY{n}{i}\PY{p}{)}
         \PY{n+nb}{print}\PY{p}{(}\PY{l+s+s1}{\PYZsq{}}\PY{l+s+s1}{\PYZsq{}}\PY{p}{)}
         
         \PY{c+c1}{\PYZsh{} Enumerate index and value of list}
         \PY{c+c1}{\PYZsh{} index : value = 0:1, 1:2, 2:3, 3:4, 4:5}
         \PY{k}{for} \PY{n}{index}\PY{p}{,} \PY{n}{value} \PY{o+ow}{in} \PY{n+nb}{enumerate}\PY{p}{(}\PY{n}{lis}\PY{p}{)}\PY{p}{:}
             \PY{n+nb}{print}\PY{p}{(}\PY{n}{index}\PY{p}{,}\PY{l+s+s2}{\PYZdq{}}\PY{l+s+s2}{ : }\PY{l+s+s2}{\PYZdq{}}\PY{p}{,}\PY{n}{value}\PY{p}{)}
         \PY{n+nb}{print}\PY{p}{(}\PY{l+s+s1}{\PYZsq{}}\PY{l+s+s1}{\PYZsq{}}\PY{p}{)}   
         
         \PY{c+c1}{\PYZsh{} For dictionaries}
         \PY{c+c1}{\PYZsh{} We can use for loop to achive key and value of dictionary. We learnt key and value at dictionary part.}
         \PY{n}{dictionary} \PY{o}{=} \PY{p}{\PYZob{}}\PY{l+s+s1}{\PYZsq{}}\PY{l+s+s1}{spain}\PY{l+s+s1}{\PYZsq{}}\PY{p}{:}\PY{l+s+s1}{\PYZsq{}}\PY{l+s+s1}{madrid}\PY{l+s+s1}{\PYZsq{}}\PY{p}{,}\PY{l+s+s1}{\PYZsq{}}\PY{l+s+s1}{france}\PY{l+s+s1}{\PYZsq{}}\PY{p}{:}\PY{l+s+s1}{\PYZsq{}}\PY{l+s+s1}{paris}\PY{l+s+s1}{\PYZsq{}}\PY{p}{\PYZcb{}}
         \PY{k}{for} \PY{n}{key}\PY{p}{,}\PY{n}{value} \PY{o+ow}{in} \PY{n}{dictionary}\PY{o}{.}\PY{n}{items}\PY{p}{(}\PY{p}{)}\PY{p}{:}
             \PY{n+nb}{print}\PY{p}{(}\PY{n}{key}\PY{p}{,}\PY{l+s+s2}{\PYZdq{}}\PY{l+s+s2}{ : }\PY{l+s+s2}{\PYZdq{}}\PY{p}{,}\PY{n}{value}\PY{p}{)}
         \PY{n+nb}{print}\PY{p}{(}\PY{l+s+s1}{\PYZsq{}}\PY{l+s+s1}{\PYZsq{}}\PY{p}{)}
         
         \PY{c+c1}{\PYZsh{} For pandas we can achieve index and value}
         \PY{k}{for} \PY{n}{index}\PY{p}{,}\PY{n}{value} \PY{o+ow}{in} \PY{n}{data}\PY{p}{[}\PY{p}{[}\PY{l+s+s1}{\PYZsq{}}\PY{l+s+s1}{Attack}\PY{l+s+s1}{\PYZsq{}}\PY{p}{]}\PY{p}{]}\PY{p}{[}\PY{l+m+mi}{0}\PY{p}{:}\PY{l+m+mi}{5}\PY{p}{]}\PY{o}{.}\PY{n}{iterrows}\PY{p}{(}\PY{p}{)}\PY{p}{:}
             \PY{n+nb}{print}\PY{p}{(}\PY{n}{index}\PY{p}{,}\PY{l+s+s2}{\PYZdq{}}\PY{l+s+s2}{ : }\PY{l+s+s2}{\PYZdq{}}\PY{p}{,}\PY{n}{value}\PY{p}{)}
\end{Verbatim}


    \begin{Verbatim}[commandchars=\\\{\}]
i is:  1
i is:  2
i is:  3
i is:  4
i is:  5

0  :  1
1  :  2
2  :  3
3  :  4
4  :  5

spain  :  madrid
france  :  paris

0  :  Attack    49
Name: 0, dtype: int64
1  :  Attack    62
Name: 1, dtype: int64
2  :  Attack    82
Name: 2, dtype: int64
3  :  Attack    100
Name: 3, dtype: int64
4  :  Attack    52
Name: 4, dtype: int64

    \end{Verbatim}

    In this part, you learn: * how to import csv file * plotting
line,scatter and histogram * basic dictionary features * basic pandas
features like filtering that is actually something always used and main
for being data scientist * While and for loops

    \hypertarget{python-data-science-toolbox}{%
\section{2. PYTHON DATA SCIENCE
TOOLBOX}\label{python-data-science-toolbox}}

    \hypertarget{user-defined-function}{%
\subsubsection{USER DEFINED FUNCTION}\label{user-defined-function}}

What we need to know about functions: * docstrings: documentation for
functions. Example: for f(): ""``This is docstring for documentation of
function f''"" * tuble: sequence of immutable python objects. cant
modify values tuble uses paranthesis like tuble = (1,2,3) unpack tuble
into several variables like a,b,c = tuble

    \begin{Verbatim}[commandchars=\\\{\}]
{\color{incolor}In [{\color{incolor}24}]:} \PY{c+c1}{\PYZsh{} example of what we learn above}
         \PY{k}{def} \PY{n+nf}{tuble\PYZus{}ex}\PY{p}{(}\PY{p}{)}\PY{p}{:}
             \PY{l+s+sd}{\PYZdq{}\PYZdq{}\PYZdq{} return defined t tuble\PYZdq{}\PYZdq{}\PYZdq{}}
             \PY{n}{t} \PY{o}{=} \PY{p}{(}\PY{l+m+mi}{1}\PY{p}{,}\PY{l+m+mi}{2}\PY{p}{,}\PY{l+m+mi}{3}\PY{p}{)}
             \PY{k}{return} \PY{n}{t}
         \PY{n}{a}\PY{p}{,}\PY{n}{b}\PY{p}{,}\PY{n}{c} \PY{o}{=} \PY{n}{tuble\PYZus{}ex}\PY{p}{(}\PY{p}{)}
         \PY{n+nb}{print}\PY{p}{(}\PY{n}{a}\PY{p}{,}\PY{n}{b}\PY{p}{,}\PY{n}{c}\PY{p}{)}
\end{Verbatim}


    \begin{Verbatim}[commandchars=\\\{\}]
1 2 3

    \end{Verbatim}

    \hypertarget{scope}{%
\subsubsection{SCOPE}\label{scope}}

What we need to know about scope: * global: defined main body in script
* local: defined in a function * built in scope: names in predefined
built in scope module such as print, len Lets make some basic examples

    \begin{Verbatim}[commandchars=\\\{\}]
{\color{incolor}In [{\color{incolor}25}]:} \PY{c+c1}{\PYZsh{} guess print what}
         \PY{n}{x} \PY{o}{=} \PY{l+m+mi}{2}
         \PY{k}{def} \PY{n+nf}{f}\PY{p}{(}\PY{p}{)}\PY{p}{:}
             \PY{n}{x} \PY{o}{=} \PY{l+m+mi}{3}
             \PY{k}{return} \PY{n}{x}
         \PY{n+nb}{print}\PY{p}{(}\PY{n}{x}\PY{p}{)}      \PY{c+c1}{\PYZsh{} x = 2 global scope}
         \PY{n+nb}{print}\PY{p}{(}\PY{n}{f}\PY{p}{(}\PY{p}{)}\PY{p}{)}    \PY{c+c1}{\PYZsh{} x = 3 local scope}
\end{Verbatim}


    \begin{Verbatim}[commandchars=\\\{\}]
2
3

    \end{Verbatim}

    \begin{Verbatim}[commandchars=\\\{\}]
{\color{incolor}In [{\color{incolor}26}]:} \PY{c+c1}{\PYZsh{} What if there is no local scope}
         \PY{n}{x} \PY{o}{=} \PY{l+m+mi}{5}
         \PY{k}{def} \PY{n+nf}{f}\PY{p}{(}\PY{p}{)}\PY{p}{:}
             \PY{n}{y} \PY{o}{=} \PY{l+m+mi}{2}\PY{o}{*}\PY{n}{x}        \PY{c+c1}{\PYZsh{} there is no local scope x}
             \PY{k}{return} \PY{n}{y}
         \PY{n+nb}{print}\PY{p}{(}\PY{n}{f}\PY{p}{(}\PY{p}{)}\PY{p}{)}         \PY{c+c1}{\PYZsh{} it uses global scope x}
         \PY{c+c1}{\PYZsh{} First local scopesearched, then global scope searched, if two of them cannot be found lastly built in scope searched.}
\end{Verbatim}


    \begin{Verbatim}[commandchars=\\\{\}]
10

    \end{Verbatim}

    \begin{Verbatim}[commandchars=\\\{\}]
{\color{incolor}In [{\color{incolor}27}]:} \PY{c+c1}{\PYZsh{} How can we learn what is built in scope}
         \PY{k+kn}{import} \PY{n+nn}{builtins}
         \PY{n+nb}{dir}\PY{p}{(}\PY{n}{builtins}\PY{p}{)}
\end{Verbatim}


\begin{Verbatim}[commandchars=\\\{\}]
{\color{outcolor}Out[{\color{outcolor}27}]:} ['ArithmeticError',
          'AssertionError',
          'AttributeError',
          'BaseException',
          'BlockingIOError',
          'BrokenPipeError',
          'BufferError',
          'BytesWarning',
          'ChildProcessError',
          'ConnectionAbortedError',
          'ConnectionError',
          'ConnectionRefusedError',
          'ConnectionResetError',
          'DeprecationWarning',
          'EOFError',
          'Ellipsis',
          'EnvironmentError',
          'Exception',
          'False',
          'FileExistsError',
          'FileNotFoundError',
          'FloatingPointError',
          'FutureWarning',
          'GeneratorExit',
          'IOError',
          'ImportError',
          'ImportWarning',
          'IndentationError',
          'IndexError',
          'InterruptedError',
          'IsADirectoryError',
          'KeyError',
          'KeyboardInterrupt',
          'LookupError',
          'MemoryError',
          'ModuleNotFoundError',
          'NameError',
          'None',
          'NotADirectoryError',
          'NotImplemented',
          'NotImplementedError',
          'OSError',
          'OverflowError',
          'PendingDeprecationWarning',
          'PermissionError',
          'ProcessLookupError',
          'RecursionError',
          'ReferenceError',
          'ResourceWarning',
          'RuntimeError',
          'RuntimeWarning',
          'StopAsyncIteration',
          'StopIteration',
          'SyntaxError',
          'SyntaxWarning',
          'SystemError',
          'SystemExit',
          'TabError',
          'TimeoutError',
          'True',
          'TypeError',
          'UnboundLocalError',
          'UnicodeDecodeError',
          'UnicodeEncodeError',
          'UnicodeError',
          'UnicodeTranslateError',
          'UnicodeWarning',
          'UserWarning',
          'ValueError',
          'Warning',
          'ZeroDivisionError',
          '\_\_IPYTHON\_\_',
          '\_\_build\_class\_\_',
          '\_\_debug\_\_',
          '\_\_doc\_\_',
          '\_\_import\_\_',
          '\_\_loader\_\_',
          '\_\_name\_\_',
          '\_\_package\_\_',
          '\_\_spec\_\_',
          'abs',
          'all',
          'any',
          'ascii',
          'bin',
          'bool',
          'bytearray',
          'bytes',
          'callable',
          'chr',
          'classmethod',
          'compile',
          'complex',
          'copyright',
          'credits',
          'delattr',
          'dict',
          'dir',
          'display',
          'divmod',
          'enumerate',
          'eval',
          'exec',
          'filter',
          'float',
          'format',
          'frozenset',
          'get\_ipython',
          'getattr',
          'globals',
          'hasattr',
          'hash',
          'help',
          'hex',
          'id',
          'input',
          'int',
          'isinstance',
          'issubclass',
          'iter',
          'len',
          'license',
          'list',
          'locals',
          'map',
          'max',
          'memoryview',
          'min',
          'next',
          'object',
          'oct',
          'open',
          'ord',
          'pow',
          'print',
          'property',
          'range',
          'repr',
          'reversed',
          'round',
          'set',
          'setattr',
          'slice',
          'sorted',
          'staticmethod',
          'str',
          'sum',
          'super',
          'tuple',
          'type',
          'vars',
          'zip']
\end{Verbatim}
            
    \hypertarget{nested-function}{%
\subsubsection{NESTED FUNCTION}\label{nested-function}}

\begin{itemize}
\tightlist
\item
  function inside function.
\item
  There is a LEGB rule that is search local scope, enclosing function,
  global and built in scopes, respectively.
\end{itemize}

    \begin{Verbatim}[commandchars=\\\{\}]
{\color{incolor}In [{\color{incolor}28}]:} \PY{c+c1}{\PYZsh{}nested function}
         \PY{k}{def} \PY{n+nf}{square}\PY{p}{(}\PY{p}{)}\PY{p}{:}
             \PY{l+s+sd}{\PYZdq{}\PYZdq{}\PYZdq{} return square of value \PYZdq{}\PYZdq{}\PYZdq{}}
             \PY{k}{def} \PY{n+nf}{add}\PY{p}{(}\PY{p}{)}\PY{p}{:}
                 \PY{l+s+sd}{\PYZdq{}\PYZdq{}\PYZdq{} add two local variable \PYZdq{}\PYZdq{}\PYZdq{}}
                 \PY{n}{x} \PY{o}{=} \PY{l+m+mi}{2}
                 \PY{n}{y} \PY{o}{=} \PY{l+m+mi}{3}
                 \PY{n}{z} \PY{o}{=} \PY{n}{x} \PY{o}{+} \PY{n}{y}
                 \PY{k}{return} \PY{n}{z}
             \PY{k}{return} \PY{n}{add}\PY{p}{(}\PY{p}{)}\PY{o}{*}\PY{o}{*}\PY{l+m+mi}{2}
         \PY{n+nb}{print}\PY{p}{(}\PY{n}{square}\PY{p}{(}\PY{p}{)}\PY{p}{)}    
\end{Verbatim}


    \begin{Verbatim}[commandchars=\\\{\}]
25

    \end{Verbatim}

    \hypertarget{default-and-flexible-arguments}{%
\subsubsection{DEFAULT and FLEXIBLE
ARGUMENTS}\label{default-and-flexible-arguments}}

\begin{itemize}
\tightlist
\item
  Default argument example: def f(a, b=1): """ b = 1 is default
  argument"""
\item
  Flexible argument example: def f(\emph{args): """ }args can be one or
  more""" def f(** kwargs) """ **kwargs is a dictionary"""
\end{itemize}

 lets write some code to practice

    \begin{Verbatim}[commandchars=\\\{\}]
{\color{incolor}In [{\color{incolor}89}]:} \PY{c+c1}{\PYZsh{} default arguments}
         \PY{k}{def} \PY{n+nf}{f}\PY{p}{(}\PY{n}{a}\PY{p}{,} \PY{n}{b} \PY{o}{=} \PY{l+m+mi}{1}\PY{p}{,} \PY{n}{c} \PY{o}{=} \PY{l+m+mi}{2}\PY{p}{)}\PY{p}{:}
             \PY{n}{y} \PY{o}{=} \PY{n}{a} \PY{o}{+} \PY{n}{b} \PY{o}{\PYZhy{}} \PY{n}{c}
             \PY{k}{return} \PY{n}{y}
         \PY{n+nb}{print}\PY{p}{(}\PY{n}{f}\PY{p}{(}\PY{l+m+mi}{5}\PY{p}{)}\PY{p}{)}
         \PY{c+c1}{\PYZsh{} what if we want to change default arguments}
         \PY{n+nb}{print}\PY{p}{(}\PY{n}{f}\PY{p}{(}\PY{n}{b}\PY{o}{=}\PY{l+m+mi}{5}\PY{p}{,}\PY{n}{c}\PY{o}{=}\PY{l+m+mi}{4}\PY{p}{,}\PY{n}{a}\PY{o}{=}\PY{l+m+mi}{3}\PY{p}{)}\PY{p}{)}
\end{Verbatim}


    \begin{Verbatim}[commandchars=\\\{\}]
4
4

    \end{Verbatim}

    \begin{Verbatim}[commandchars=\\\{\}]
{\color{incolor}In [{\color{incolor}87}]:} \PY{c+c1}{\PYZsh{} flexible arguments *args}
         \PY{k}{def} \PY{n+nf}{f}\PY{p}{(}\PY{o}{*}\PY{n}{args}\PY{p}{)}\PY{p}{:}
             \PY{k}{for} \PY{n}{i} \PY{o+ow}{in} \PY{n}{args}\PY{p}{:}
                 \PY{n+nb}{print}\PY{p}{(}\PY{n}{i}\PY{p}{)}
         \PY{n}{f}\PY{p}{(}\PY{l+m+mi}{1}\PY{p}{)}
         \PY{n+nb}{print}\PY{p}{(}\PY{l+s+s2}{\PYZdq{}}\PY{l+s+s2}{\PYZdq{}}\PY{p}{)}
         \PY{n}{f}\PY{p}{(}\PY{l+m+mi}{1}\PY{p}{,}\PY{l+m+mi}{2}\PY{p}{,}\PY{l+m+mi}{3}\PY{p}{,}\PY{l+m+mi}{4}\PY{p}{)}
         \PY{c+c1}{\PYZsh{} flexible arguments **kwargs that is dictionary}
         \PY{k}{def} \PY{n+nf}{f}\PY{p}{(}\PY{o}{*}\PY{o}{*}\PY{n}{kwargs}\PY{p}{)}\PY{p}{:}
             \PY{l+s+sd}{\PYZdq{}\PYZdq{}\PYZdq{} print key and value of dictionary\PYZdq{}\PYZdq{}\PYZdq{}}
             \PY{k}{for} \PY{n}{key}\PY{p}{,} \PY{n}{value} \PY{o+ow}{in} \PY{n}{kwargs}\PY{o}{.}\PY{n}{items}\PY{p}{(}\PY{p}{)}\PY{p}{:}               \PY{c+c1}{\PYZsh{} If you do not understand this part turn for loop part and look at dictionary in for loop}
                 \PY{n+nb}{print}\PY{p}{(}\PY{n}{key}\PY{p}{,} \PY{l+s+s2}{\PYZdq{}}\PY{l+s+s2}{ }\PY{l+s+s2}{\PYZdq{}}\PY{p}{,} \PY{n}{value}\PY{p}{)}
         \PY{n}{f}\PY{p}{(}\PY{n}{country} \PY{o}{=} \PY{l+s+s1}{\PYZsq{}}\PY{l+s+s1}{spain}\PY{l+s+s1}{\PYZsq{}}\PY{p}{,} \PY{n}{capital} \PY{o}{=} \PY{l+s+s1}{\PYZsq{}}\PY{l+s+s1}{madrid}\PY{l+s+s1}{\PYZsq{}}\PY{p}{,} \PY{n}{population} \PY{o}{=} \PY{l+m+mi}{123456}\PY{p}{)}
\end{Verbatim}


    \begin{Verbatim}[commandchars=\\\{\}]

          File "<ipython-input-87-f2b132cdf72c>", line 13
        f(\{ountry = 'spain', capital = 'madrid', population = 123456\})
                  \^{}
    SyntaxError: invalid syntax


    \end{Verbatim}

    \hypertarget{lambda-function}{%
\subsubsection{LAMBDA FUNCTION}\label{lambda-function}}

Faster way of writing function

    \begin{Verbatim}[commandchars=\\\{\}]
{\color{incolor}In [{\color{incolor}90}]:} \PY{c+c1}{\PYZsh{} lambda function}
         \PY{n}{square} \PY{o}{=} \PY{k}{lambda} \PY{n}{x}\PY{p}{:} \PY{n}{x}\PY{o}{*}\PY{o}{*}\PY{l+m+mi}{2}     \PY{c+c1}{\PYZsh{} where x is name of argument}
         \PY{n+nb}{print}\PY{p}{(}\PY{n}{square}\PY{p}{(}\PY{l+m+mi}{4}\PY{p}{)}\PY{p}{)}
         \PY{n}{tot} \PY{o}{=} \PY{k}{lambda} \PY{n}{x}\PY{p}{,}\PY{n}{y}\PY{p}{,}\PY{n}{z}\PY{p}{:} \PY{n}{x}\PY{o}{+}\PY{n}{y}\PY{o}{+}\PY{n}{z}   \PY{c+c1}{\PYZsh{} where x,y,z are names of arguments}
         \PY{n+nb}{print}\PY{p}{(}\PY{n}{tot}\PY{p}{(}\PY{l+m+mi}{1}\PY{p}{,}\PY{l+m+mi}{2}\PY{p}{,}\PY{l+m+mi}{3}\PY{p}{)}\PY{p}{)}
\end{Verbatim}


    \begin{Verbatim}[commandchars=\\\{\}]
16
6

    \end{Verbatim}

    \hypertarget{anonymous-function}{%
\subsubsection{ANONYMOUS FUNCTİON}\label{anonymous-function}}

Like lambda function but it can take more than one arguments. *
map(func,seq) : applies a function to all the items in a list

    \begin{Verbatim}[commandchars=\\\{\}]
{\color{incolor}In [{\color{incolor}100}]:} \PY{n}{number\PYZus{}list} \PY{o}{=} \PY{p}{[}\PY{l+m+mi}{1}\PY{p}{,}\PY{l+m+mi}{2}\PY{p}{,}\PY{l+m+mi}{3}\PY{p}{]}
          \PY{n}{y} \PY{o}{=} \PY{p}{[}\PY{n}{yb} \PY{k}{for} \PY{n}{yb} \PY{o+ow}{in} \PY{n+nb}{map}\PY{p}{(}\PY{k}{lambda} \PY{n}{x}\PY{p}{:}\PY{n}{x}\PY{o}{*}\PY{o}{*}\PY{l+m+mi}{2}\PY{p}{,}\PY{n}{number\PYZus{}list}\PY{p}{)}\PY{p}{]}
          \PY{n+nb}{print}\PY{p}{(}\PY{n}{y}\PY{p}{)}
          
          \PY{n}{x} \PY{o}{=} \PY{p}{[}\PY{n}{xb}\PY{o}{*}\PY{o}{*}\PY{l+m+mi}{2} \PY{k}{for} \PY{n}{xb} \PY{o+ow}{in} \PY{n}{number\PYZus{}list}\PY{p}{]}
          \PY{n+nb}{print}\PY{p}{(}\PY{n}{x}\PY{p}{)}
\end{Verbatim}


    \begin{Verbatim}[commandchars=\\\{\}]
[1, 4, 9]
[1, 4, 9]

    \end{Verbatim}

    \hypertarget{iterators}{%
\subsubsection{ITERATORS}\label{iterators}}

\begin{itemize}
\tightlist
\item
  iterable is an object that can return an iterator
\item
  iterable: an object with an associated iter() method example: list,
  strings and dictionaries
\item
  iterator: produces next value with next() method
\end{itemize}

    \begin{Verbatim}[commandchars=\\\{\}]
{\color{incolor}In [{\color{incolor}96}]:} \PY{c+c1}{\PYZsh{} iteration example}
         \PY{n}{name} \PY{o}{=} \PY{l+s+s2}{\PYZdq{}}\PY{l+s+s2}{ronaldo}\PY{l+s+s2}{\PYZdq{}}
         \PY{n}{it} \PY{o}{=} \PY{n+nb}{iter}\PY{p}{(}\PY{n}{name}\PY{p}{)}
         \PY{n+nb}{print}\PY{p}{(}\PY{n+nb}{next}\PY{p}{(}\PY{n}{it}\PY{p}{)}\PY{p}{)}    \PY{c+c1}{\PYZsh{} print next iteration}
         \PY{n+nb}{print}\PY{p}{(}\PY{o}{*}\PY{n}{it}\PY{p}{)}         \PY{c+c1}{\PYZsh{} print remaining iteration}
\end{Verbatim}


    \begin{Verbatim}[commandchars=\\\{\}]
r
o n a l d o

    \end{Verbatim}

    zip(): zip lists

    \begin{Verbatim}[commandchars=\\\{\}]
{\color{incolor}In [{\color{incolor}97}]:} \PY{c+c1}{\PYZsh{} zip example}
         \PY{n}{list1} \PY{o}{=} \PY{p}{[}\PY{l+m+mi}{1}\PY{p}{,}\PY{l+m+mi}{2}\PY{p}{,}\PY{l+m+mi}{3}\PY{p}{,}\PY{l+m+mi}{4}\PY{p}{]}
         \PY{n}{list2} \PY{o}{=} \PY{p}{[}\PY{l+m+mi}{5}\PY{p}{,}\PY{l+m+mi}{6}\PY{p}{,}\PY{l+m+mi}{7}\PY{p}{,}\PY{l+m+mi}{8}\PY{p}{]}
         \PY{n}{z} \PY{o}{=} \PY{n+nb}{zip}\PY{p}{(}\PY{n}{list1}\PY{p}{,}\PY{n}{list2}\PY{p}{)}
         \PY{n+nb}{print}\PY{p}{(}\PY{n}{z}\PY{p}{)}
         \PY{n}{z\PYZus{}list} \PY{o}{=} \PY{n+nb}{list}\PY{p}{(}\PY{n}{z}\PY{p}{)}
         \PY{n+nb}{print}\PY{p}{(}\PY{n}{z\PYZus{}list}\PY{p}{)}
\end{Verbatim}


    \begin{Verbatim}[commandchars=\\\{\}]
<zip object at 0x11b552948>
[(1, 5), (2, 6), (3, 7), (4, 8)]

    \end{Verbatim}

    \begin{Verbatim}[commandchars=\\\{\}]
{\color{incolor}In [{\color{incolor}98}]:} \PY{n}{un\PYZus{}zip} \PY{o}{=} \PY{n+nb}{zip}\PY{p}{(}\PY{o}{*}\PY{n}{z\PYZus{}list}\PY{p}{)}
         \PY{n}{un\PYZus{}list1}\PY{p}{,}\PY{n}{un\PYZus{}list2} \PY{o}{=} \PY{n+nb}{list}\PY{p}{(}\PY{n}{un\PYZus{}zip}\PY{p}{)} \PY{c+c1}{\PYZsh{} unzip returns tuble}
         \PY{n+nb}{print}\PY{p}{(}\PY{n}{un\PYZus{}list1}\PY{p}{)}
         \PY{n+nb}{print}\PY{p}{(}\PY{n}{un\PYZus{}list2}\PY{p}{)}
         \PY{n+nb}{print}\PY{p}{(}\PY{n+nb}{type}\PY{p}{(}\PY{n}{un\PYZus{}list2}\PY{p}{)}\PY{p}{)}
\end{Verbatim}


    \begin{Verbatim}[commandchars=\\\{\}]
(1, 2, 3, 4)
(5, 6, 7, 8)
<class 'tuple'>

    \end{Verbatim}

    \hypertarget{list-comprehension}{%
\subsubsection{LIST COMPREHENSİON}\label{list-comprehension}}

\textbf{One of the most important topic of this kernel} We use list
comprehension for data analysis often. list comprehension: collapse for
loops for building lists into a single line Ex: num1 = {[}1,2,3{]} and
we want to make it num2 = {[}2,3,4{]}. This can be done with for loop.
However it is unnecessarily long. We can make it one line code that is
list comprehension.

    \begin{Verbatim}[commandchars=\\\{\}]
{\color{incolor}In [{\color{incolor}99}]:} \PY{c+c1}{\PYZsh{} Example of list comprehension}
         \PY{n}{num1} \PY{o}{=} \PY{p}{[}\PY{l+m+mi}{1}\PY{p}{,}\PY{l+m+mi}{2}\PY{p}{,}\PY{l+m+mi}{3}\PY{p}{]}
         \PY{n}{num2} \PY{o}{=} \PY{p}{[}\PY{n}{i} \PY{o}{+} \PY{l+m+mi}{1} \PY{k}{for} \PY{n}{i} \PY{o+ow}{in} \PY{n}{num1} \PY{p}{]}
         \PY{n+nb}{print}\PY{p}{(}\PY{n}{num2}\PY{p}{)}
\end{Verbatim}


    \begin{Verbatim}[commandchars=\\\{\}]
[2, 3, 4]

    \end{Verbatim}

     i +1: list comprehension syntax for i in num1: for loop syntax i:
iterator num1: iterable object

    \begin{Verbatim}[commandchars=\\\{\}]
{\color{incolor}In [{\color{incolor} }]:} \PY{c+c1}{\PYZsh{} Conditionals on iterable}
        \PY{n}{num1} \PY{o}{=} \PY{p}{[}\PY{l+m+mi}{5}\PY{p}{,}\PY{l+m+mi}{10}\PY{p}{,}\PY{l+m+mi}{15}\PY{p}{]}
        \PY{n}{num2} \PY{o}{=} \PY{p}{[}\PY{n}{i}\PY{o}{*}\PY{o}{*}\PY{l+m+mi}{2} \PY{k}{if} \PY{n}{i} \PY{o}{==} \PY{l+m+mi}{10} \PY{k}{else} \PY{n}{i}\PY{o}{\PYZhy{}}\PY{l+m+mi}{5} \PY{k}{if} \PY{n}{i} \PY{o}{\PYZlt{}} \PY{l+m+mi}{7} \PY{k}{else} \PY{n}{i}\PY{o}{+}\PY{l+m+mi}{5} \PY{k}{for} \PY{n}{i} \PY{o+ow}{in} \PY{n}{num1}\PY{p}{]}
        \PY{n+nb}{print}\PY{p}{(}\PY{n}{num2}\PY{p}{)}
\end{Verbatim}


    \begin{Verbatim}[commandchars=\\\{\}]
{\color{incolor}In [{\color{incolor}101}]:} \PY{c+c1}{\PYZsh{} lets return pokemon csv and make one more list comprehension example}
          \PY{c+c1}{\PYZsh{} lets classify pokemons whether they have high or low speed. Our threshold is average speed.}
          \PY{n}{threshold} \PY{o}{=} \PY{n+nb}{sum}\PY{p}{(}\PY{n}{data}\PY{o}{.}\PY{n}{Speed}\PY{p}{)}\PY{o}{/}\PY{n+nb}{len}\PY{p}{(}\PY{n}{data}\PY{o}{.}\PY{n}{Speed}\PY{p}{)}
          \PY{n}{data}\PY{p}{[}\PY{l+s+s2}{\PYZdq{}}\PY{l+s+s2}{speed\PYZus{}level}\PY{l+s+s2}{\PYZdq{}}\PY{p}{]} \PY{o}{=} \PY{p}{[}\PY{l+s+s2}{\PYZdq{}}\PY{l+s+s2}{high}\PY{l+s+s2}{\PYZdq{}} \PY{k}{if} \PY{n}{i} \PY{o}{\PYZgt{}} \PY{n}{threshold} \PY{k}{else} \PY{l+s+s2}{\PYZdq{}}\PY{l+s+s2}{low}\PY{l+s+s2}{\PYZdq{}} \PY{k}{for} \PY{n}{i} \PY{o+ow}{in} \PY{n}{data}\PY{o}{.}\PY{n}{Speed}\PY{p}{]}
          \PY{n}{data}\PY{o}{.}\PY{n}{loc}\PY{p}{[}\PY{p}{:}\PY{l+m+mi}{10}\PY{p}{,}\PY{p}{[}\PY{l+s+s2}{\PYZdq{}}\PY{l+s+s2}{speed\PYZus{}level}\PY{l+s+s2}{\PYZdq{}}\PY{p}{,}\PY{l+s+s2}{\PYZdq{}}\PY{l+s+s2}{Speed}\PY{l+s+s2}{\PYZdq{}}\PY{p}{]}\PY{p}{]} \PY{c+c1}{\PYZsh{} we will learn loc more detailed later}
\end{Verbatim}


\begin{Verbatim}[commandchars=\\\{\}]
{\color{outcolor}Out[{\color{outcolor}101}]:}    speed\_level  Speed
          0          low     45
          1          low     60
          2         high     80
          3         high     80
          4          low     65
          5         high     80
          6         high    100
          7         high    100
          8         high    100
          9          low     43
          10         low     58
\end{Verbatim}
            
    Up to now, you learn * User defined function * Scope * Nested function *
Default and flexible arguments * Lambda function * Anonymous function *
Iterators * List comprehension

    \hypertarget{cleaning-data}{%
\section{3.CLEANING DATA}\label{cleaning-data}}

    \hypertarget{diagnose-data-for-cleaning}{%
\subsubsection{DIAGNOSE DATA for
CLEANING}\label{diagnose-data-for-cleaning}}

We need to diagnose and clean data before exploring. Unclean data: *
Column name inconsistency like upper-lower case letter or space between
words * missing data * different language

 We will use head, tail, columns, shape and info methods to diagnose
data

    \begin{Verbatim}[commandchars=\\\{\}]
{\color{incolor}In [{\color{incolor}102}]:} \PY{n}{data} \PY{o}{=} \PY{n}{pd}\PY{o}{.}\PY{n}{read\PYZus{}csv}\PY{p}{(}\PY{l+s+s1}{\PYZsq{}}\PY{l+s+s1}{../input/pokemon.csv}\PY{l+s+s1}{\PYZsq{}}\PY{p}{)}
          \PY{n}{data}\PY{o}{.}\PY{n}{head}\PY{p}{(}\PY{p}{)}  \PY{c+c1}{\PYZsh{} head shows first 5 rows}
\end{Verbatim}


\begin{Verbatim}[commandchars=\\\{\}]
{\color{outcolor}Out[{\color{outcolor}102}]:}    \#           Name Type 1  Type 2  HP  Attack  Defense  Sp. Atk  Sp. Def  \textbackslash{}
          0  1      Bulbasaur  Grass  Poison  45      49       49       65       65   
          1  2        Ivysaur  Grass  Poison  60      62       63       80       80   
          2  3       Venusaur  Grass  Poison  80      82       83      100      100   
          3  4  Mega Venusaur  Grass  Poison  80     100      123      122      120   
          4  5     Charmander   Fire     NaN  39      52       43       60       50   
          
             Speed  Generation  Legendary  
          0     45           1      False  
          1     60           1      False  
          2     80           1      False  
          3     80           1      False  
          4     65           1      False  
\end{Verbatim}
            
    \begin{Verbatim}[commandchars=\\\{\}]
{\color{incolor}In [{\color{incolor}103}]:} \PY{c+c1}{\PYZsh{} tail shows last 5 rows}
          \PY{n}{data}\PY{o}{.}\PY{n}{tail}\PY{p}{(}\PY{p}{)}
\end{Verbatim}


\begin{Verbatim}[commandchars=\\\{\}]
{\color{outcolor}Out[{\color{outcolor}103}]:}        \#            Name   Type 1 Type 2  HP  Attack  Defense  Sp. Atk  \textbackslash{}
          795  796         Diancie     Rock  Fairy  50     100      150      100   
          796  797    Mega Diancie     Rock  Fairy  50     160      110      160   
          797  798  Hoopa Confined  Psychic  Ghost  80     110       60      150   
          798  799   Hoopa Unbound  Psychic   Dark  80     160       60      170   
          799  800       Volcanion     Fire  Water  80     110      120      130   
          
               Sp. Def  Speed  Generation  Legendary  
          795      150     50           6       True  
          796      110    110           6       True  
          797      130     70           6       True  
          798      130     80           6       True  
          799       90     70           6       True  
\end{Verbatim}
            
    \begin{Verbatim}[commandchars=\\\{\}]
{\color{incolor}In [{\color{incolor}104}]:} \PY{c+c1}{\PYZsh{} columns gives column names of features}
          \PY{n}{data}\PY{o}{.}\PY{n}{columns}
\end{Verbatim}


\begin{Verbatim}[commandchars=\\\{\}]
{\color{outcolor}Out[{\color{outcolor}104}]:} Index(['\#', 'Name', 'Type 1', 'Type 2', 'HP', 'Attack', 'Defense', 'Sp. Atk',
                 'Sp. Def', 'Speed', 'Generation', 'Legendary'],
                dtype='object')
\end{Verbatim}
            
    \begin{Verbatim}[commandchars=\\\{\}]
{\color{incolor}In [{\color{incolor}105}]:} \PY{c+c1}{\PYZsh{} shape gives number of rows and columns in a tuble}
          \PY{n}{data}\PY{o}{.}\PY{n}{shape}
\end{Verbatim}


\begin{Verbatim}[commandchars=\\\{\}]
{\color{outcolor}Out[{\color{outcolor}105}]:} (800, 12)
\end{Verbatim}
            
    \begin{Verbatim}[commandchars=\\\{\}]
{\color{incolor}In [{\color{incolor}109}]:} \PY{c+c1}{\PYZsh{} info gives data type like dataframe, number of sample or row, number of feature or column, feature types and memory usage}
          \PY{n}{data}\PY{o}{.}\PY{n}{info}\PY{p}{(}\PY{p}{)}
\end{Verbatim}


    \begin{Verbatim}[commandchars=\\\{\}]
<class 'pandas.core.frame.DataFrame'>
RangeIndex: 800 entries, 0 to 799
Data columns (total 12 columns):
\#             800 non-null int64
Name          799 non-null object
Type 1        800 non-null object
Type 2        414 non-null object
HP            800 non-null int64
Attack        800 non-null int64
Defense       800 non-null int64
Sp. Atk       800 non-null int64
Sp. Def       800 non-null int64
Speed         800 non-null int64
Generation    800 non-null int64
Legendary     800 non-null bool
dtypes: bool(1), int64(8), object(3)
memory usage: 69.6+ KB

    \end{Verbatim}

    \hypertarget{explotary-data-analysis}{%
\subsubsection{EXPLOTARY DATA ANALYSIS}\label{explotary-data-analysis}}

value\_counts(): Frequency counts outliers: the value that is
considerably higher or lower from rest of the data * Lets say value at
75\% is Q3 and value at 25\% is Q1. * Outlier are smaller than Q1 -
1.5(Q3-Q1) and bigger than Q3 + 1.5(Q3-Q1) We will use describe()
method. Describe method includes: * count: number of entries * mean:
average of entries * std: standart deviation * min: minimum entry *
25\%: first quantile * 50\%: median or second quantile * 75\%: third
quantile * max: maximum entry

 What is quantile?

\begin{itemize}
\item
  1,4,5,6,8,9,11,12,13,14,15,16,17
\item
  The median is the number that is in \textbf{middle} of the sequence.
  In this case it would be 11.
\item
  The lower quartile is the median in between the smallest number and
  the median i.e.~in between 1 and 11, which is 6.
\item
  The upper quartile, you find the median between the median and the
  largest number i.e.~between 11 and 17, which will be 14 according to
  the question above.
\end{itemize}

    \begin{Verbatim}[commandchars=\\\{\}]
{\color{incolor}In [{\color{incolor}112}]:} \PY{c+c1}{\PYZsh{} For example lets look frequency of pokemom types}
          \PY{n+nb}{print}\PY{p}{(}\PY{n}{data}\PY{p}{[}\PY{l+s+s1}{\PYZsq{}}\PY{l+s+s1}{Type 1}\PY{l+s+s1}{\PYZsq{}}\PY{p}{]}\PY{o}{.}\PY{n}{value\PYZus{}counts}\PY{p}{(}\PY{n}{dropna} \PY{o}{=}\PY{k+kc}{False}\PY{p}{)}\PY{p}{)}  \PY{c+c1}{\PYZsh{} if there are nan values that also be counted}
          \PY{c+c1}{\PYZsh{} As it can be seen below there are 112 water pokemon or 70 grass pokemon}
\end{Verbatim}


    \begin{Verbatim}[commandchars=\\\{\}]
Water       112
Normal       98
Grass        70
Bug          69
Psychic      57
Fire         52
Electric     44
Rock         44
Ground       32
Ghost        32
Dragon       32
Dark         31
Poison       28
Fighting     27
Steel        27
Ice          24
Fairy        17
Flying        4
Name: Type 1, dtype: int64

    \end{Verbatim}

    \begin{Verbatim}[commandchars=\\\{\}]
{\color{incolor}In [{\color{incolor}29}]:} \PY{n}{datatmp} \PY{o}{=} \PY{n}{data}\PY{p}{[}\PY{p}{[}\PY{l+s+s2}{\PYZdq{}}\PY{l+s+s2}{\PYZsh{}}\PY{l+s+s2}{\PYZdq{}}\PY{p}{,}\PY{l+s+s2}{\PYZdq{}}\PY{l+s+s2}{Name}\PY{l+s+s2}{\PYZdq{}}\PY{p}{,}\PY{l+s+s2}{\PYZdq{}}\PY{l+s+s2}{Type 1}\PY{l+s+s2}{\PYZdq{}}\PY{p}{,}\PY{l+s+s2}{\PYZdq{}}\PY{l+s+s2}{Type 2}\PY{l+s+s2}{\PYZdq{}}\PY{p}{]}\PY{p}{]}
         \PY{n}{pmtype} \PY{o}{=} \PY{n+nb}{set}\PY{p}{(}\PY{n}{datatmp}\PY{p}{[}\PY{l+s+s1}{\PYZsq{}}\PY{l+s+s1}{Type 1}\PY{l+s+s1}{\PYZsq{}}\PY{p}{]}\PY{p}{)}
         \PY{n}{pmtype} \PY{o}{=} \PY{n}{pmtype}\PY{o}{.}\PY{n}{union}\PY{p}{(}\PY{n+nb}{set}\PY{p}{(}\PY{n}{datatmp}\PY{p}{[}\PY{l+s+s1}{\PYZsq{}}\PY{l+s+s1}{Type 2}\PY{l+s+s1}{\PYZsq{}}\PY{p}{]}\PY{o}{.}\PY{n}{dropna}\PY{p}{(}\PY{p}{)}\PY{p}{)}\PY{p}{)}
         \PY{n+nb}{print}\PY{p}{(}\PY{n}{pmtype}\PY{p}{)}
\end{Verbatim}


    \begin{Verbatim}[commandchars=\\\{\}]
\{'Water', 'Fighting', 'Ghost', 'Steel', 'Psychic', 'Fairy', 'Flying', 'Ground', 'Dragon', 'Grass', 'Poison', 'Bug', 'Normal', 'Electric', 'Fire', 'Dark', 'Rock', 'Ice'\}

    \end{Verbatim}

    \begin{Verbatim}[commandchars=\\\{\}]
{\color{incolor}In [{\color{incolor}30}]:} \PY{k}{for} \PY{n}{s} \PY{o+ow}{in} \PY{n+nb}{list}\PY{p}{(}\PY{n}{pmtype}\PY{p}{)}\PY{p}{:} \PY{n}{datatmp}\PY{p}{[}\PY{n}{s}\PY{p}{]} \PY{o}{=} \PY{n}{pd}\PY{o}{.}\PY{n}{Series}\PY{p}{(}\PY{p}{[}\PY{l+m+mi}{0} \PY{k}{for} \PY{n}{\PYZus{}} \PY{o+ow}{in} \PY{n+nb}{range}\PY{p}{(}\PY{n+nb}{len}\PY{p}{(}\PY{n}{datatmp}\PY{p}{)}\PY{p}{)}\PY{p}{]}\PY{p}{)}
         \PY{n}{datatmp}\PY{o}{.}\PY{n}{head}\PY{p}{(}\PY{p}{)}
\end{Verbatim}


    \begin{Verbatim}[commandchars=\\\{\}]
/Users/ayim/Desktop/virenv/lib/python3.6/site-packages/ipykernel\_launcher.py:1: SettingWithCopyWarning: 
A value is trying to be set on a copy of a slice from a DataFrame.
Try using .loc[row\_indexer,col\_indexer] = value instead

See the caveats in the documentation: http://pandas.pydata.org/pandas-docs/stable/indexing.html\#indexing-view-versus-copy
  """Entry point for launching an IPython kernel.

    \end{Verbatim}

\begin{Verbatim}[commandchars=\\\{\}]
{\color{outcolor}Out[{\color{outcolor}30}]:}    \#           Name Type 1  Type 2  Water  Fighting  Ghost  Steel  Psychic  \textbackslash{}
         0  1      Bulbasaur  Grass  Poison      0         0      0      0        0   
         1  2        Ivysaur  Grass  Poison      0         0      0      0        0   
         2  3       Venusaur  Grass  Poison      0         0      0      0        0   
         3  4  Mega Venusaur  Grass  Poison      0         0      0      0        0   
         4  5     Charmander   Fire     NaN      0         0      0      0        0   
         
            Fairy {\ldots}   Dragon  Grass  Poison  Bug  Normal  Electric  Fire  Dark  Rock  \textbackslash{}
         0      0 {\ldots}        0      0       0    0       0         0     0     0     0   
         1      0 {\ldots}        0      0       0    0       0         0     0     0     0   
         2      0 {\ldots}        0      0       0    0       0         0     0     0     0   
         3      0 {\ldots}        0      0       0    0       0         0     0     0     0   
         4      0 {\ldots}        0      0       0    0       0         0     0     0     0   
         
            Ice  
         0    0  
         1    0  
         2    0  
         3    0  
         4    0  
         
         [5 rows x 22 columns]
\end{Verbatim}
            
    \begin{Verbatim}[commandchars=\\\{\}]
{\color{incolor}In [{\color{incolor}33}]:} \PY{k}{for} \PY{n}{index}\PY{p}{,} \PY{n}{row} \PY{o+ow}{in} \PY{n}{datatmp}\PY{o}{.}\PY{n}{iterrows}\PY{p}{(}\PY{p}{)}\PY{p}{:}
             \PY{k}{if} \PY{n}{pd}\PY{o}{.}\PY{n}{isnull}\PY{p}{(}\PY{n}{row}\PY{p}{[}\PY{l+s+s1}{\PYZsq{}}\PY{l+s+s1}{Type 1}\PY{l+s+s1}{\PYZsq{}}\PY{p}{]}\PY{p}{)} \PY{p}{:} \PY{k}{continue}
             \PY{k}{else}\PY{p}{:} \PY{n}{datatmp}\PY{o}{.}\PY{n}{set\PYZus{}value}\PY{p}{(}\PY{n}{index}\PY{p}{,} \PY{n}{row}\PY{p}{[}\PY{l+s+s1}{\PYZsq{}}\PY{l+s+s1}{Type 1}\PY{l+s+s1}{\PYZsq{}}\PY{p}{]}\PY{p}{,} \PY{l+m+mi}{1}\PY{p}{)}
         
         \PY{k}{for} \PY{n}{index}\PY{p}{,} \PY{n}{row} \PY{o+ow}{in} \PY{n}{datatmp}\PY{o}{.}\PY{n}{iterrows}\PY{p}{(}\PY{p}{)}\PY{p}{:}
             \PY{k}{if} \PY{n}{pd}\PY{o}{.}\PY{n}{isnull}\PY{p}{(}\PY{n}{row}\PY{p}{[}\PY{l+s+s1}{\PYZsq{}}\PY{l+s+s1}{Type 2}\PY{l+s+s1}{\PYZsq{}}\PY{p}{]}\PY{p}{)} \PY{p}{:} \PY{k}{continue}
             \PY{k}{else}\PY{p}{:} \PY{n}{datatmp}\PY{o}{.}\PY{n}{set\PYZus{}value}\PY{p}{(}\PY{n}{index}\PY{p}{,} \PY{n}{row}\PY{p}{[}\PY{l+s+s1}{\PYZsq{}}\PY{l+s+s1}{Type 2}\PY{l+s+s1}{\PYZsq{}}\PY{p}{]}\PY{p}{,} \PY{l+m+mi}{1}\PY{p}{)}
                 
         \PY{n}{datatmp}\PY{o}{.}\PY{n}{head}\PY{p}{(}\PY{p}{)}  
\end{Verbatim}


    \begin{Verbatim}[commandchars=\\\{\}]
/Users/ayim/Desktop/virenv/lib/python3.6/site-packages/ipykernel\_launcher.py:3: FutureWarning: set\_value is deprecated and will be removed in a future release. Please use .at[] or .iat[] accessors instead
  This is separate from the ipykernel package so we can avoid doing imports until
/Users/ayim/Desktop/virenv/lib/python3.6/site-packages/ipykernel\_launcher.py:7: FutureWarning: set\_value is deprecated and will be removed in a future release. Please use .at[] or .iat[] accessors instead
  import sys

    \end{Verbatim}

\begin{Verbatim}[commandchars=\\\{\}]
{\color{outcolor}Out[{\color{outcolor}33}]:}    \#           Name Type 1  Type 2  Water  Fighting  Ghost  Steel  Psychic  \textbackslash{}
         0  1      Bulbasaur  Grass  Poison      0         0      0      0        0   
         1  2        Ivysaur  Grass  Poison      0         0      0      0        0   
         2  3       Venusaur  Grass  Poison      0         0      0      0        0   
         3  4  Mega Venusaur  Grass  Poison      0         0      0      0        0   
         4  5     Charmander   Fire     NaN      0         0      0      0        0   
         
            Fairy {\ldots}   Dragon  Grass  Poison  Bug  Normal  Electric  Fire  Dark  Rock  \textbackslash{}
         0      0 {\ldots}        0      1       1    0       0         0     0     0     0   
         1      0 {\ldots}        0      1       1    0       0         0     0     0     0   
         2      0 {\ldots}        0      1       1    0       0         0     0     0     0   
         3      0 {\ldots}        0      1       1    0       0         0     0     0     0   
         4      0 {\ldots}        0      0       0    0       0         0     1     0     0   
         
            Ice  
         0    0  
         1    0  
         2    0  
         3    0  
         4    0  
         
         [5 rows x 22 columns]
\end{Verbatim}
            
    \begin{Verbatim}[commandchars=\\\{\}]
{\color{incolor}In [{\color{incolor}34}]:} \PY{c+c1}{\PYZsh{} For example max HP is 255 or min defense is 5}
         \PY{n}{data}\PY{o}{.}\PY{n}{describe}\PY{p}{(}\PY{p}{)} \PY{c+c1}{\PYZsh{}ignore null entries}
\end{Verbatim}


\begin{Verbatim}[commandchars=\\\{\}]
{\color{outcolor}Out[{\color{outcolor}34}]:}               \#          HP      Attack     Defense     Sp. Atk     Sp. Def  \textbackslash{}
         count  800.0000  800.000000  800.000000  800.000000  800.000000  800.000000   
         mean   400.5000   69.258750   79.001250   73.842500   72.820000   71.902500   
         std    231.0844   25.534669   32.457366   31.183501   32.722294   27.828916   
         min      1.0000    1.000000    5.000000    5.000000   10.000000   20.000000   
         25\%    200.7500   50.000000   55.000000   50.000000   49.750000   50.000000   
         50\%    400.5000   65.000000   75.000000   70.000000   65.000000   70.000000   
         75\%    600.2500   80.000000  100.000000   90.000000   95.000000   90.000000   
         max    800.0000  255.000000  190.000000  230.000000  194.000000  230.000000   
         
                     Speed  Generation  
         count  800.000000   800.00000  
         mean    68.277500     3.32375  
         std     29.060474     1.66129  
         min      5.000000     1.00000  
         25\%     45.000000     2.00000  
         50\%     65.000000     3.00000  
         75\%     90.000000     5.00000  
         max    180.000000     6.00000  
\end{Verbatim}
            
    \#\#\#VISUAL EXPLORATORY DATA ANALYSIS * Box plots: visualize basic
statistics like outliers, min/max or quantiles

    \begin{Verbatim}[commandchars=\\\{\}]
{\color{incolor}In [{\color{incolor}35}]:} \PY{c+c1}{\PYZsh{} For example: compare attack of pokemons that are legendary  or not}
         \PY{c+c1}{\PYZsh{} Black line at top is max}
         \PY{c+c1}{\PYZsh{} Blue line at top is 75\PYZpc{}}
         \PY{c+c1}{\PYZsh{} Red line is median (50\PYZpc{})}
         \PY{c+c1}{\PYZsh{} Blue line at bottom is 25\PYZpc{}}
         \PY{c+c1}{\PYZsh{} Black line at bottom is min}
         \PY{c+c1}{\PYZsh{} There are no outliers}
         \PY{n}{data}\PY{o}{.}\PY{n}{boxplot}\PY{p}{(}\PY{n}{column}\PY{o}{=}\PY{l+s+s1}{\PYZsq{}}\PY{l+s+s1}{Attack}\PY{l+s+s1}{\PYZsq{}}\PY{p}{,}\PY{n}{by} \PY{o}{=} \PY{l+s+s1}{\PYZsq{}}\PY{l+s+s1}{Legendary}\PY{l+s+s1}{\PYZsq{}}\PY{p}{)}
\end{Verbatim}


\begin{Verbatim}[commandchars=\\\{\}]
{\color{outcolor}Out[{\color{outcolor}35}]:} <matplotlib.axes.\_subplots.AxesSubplot at 0x10bc4c2b0>
\end{Verbatim}
            
    \begin{center}
    \adjustimage{max size={0.9\linewidth}{0.9\paperheight}}{output_72_1.png}
    \end{center}
    { \hspace*{\fill} \\}
    
    \hypertarget{tidy-data}{%
\subsubsection{TIDY DATA}\label{tidy-data}}

We tidy data with melt(). Describing melt is confusing. Therefore lets
make example to understand it.

    \begin{Verbatim}[commandchars=\\\{\}]
{\color{incolor}In [{\color{incolor}36}]:} \PY{c+c1}{\PYZsh{} Firstly I create new data from pokemons data to explain melt nore easily.}
         \PY{n}{data\PYZus{}new} \PY{o}{=} \PY{n}{data}\PY{o}{.}\PY{n}{head}\PY{p}{(}\PY{p}{)}    \PY{c+c1}{\PYZsh{} I only take 5 rows into new data}
         \PY{n}{data\PYZus{}new}
\end{Verbatim}


\begin{Verbatim}[commandchars=\\\{\}]
{\color{outcolor}Out[{\color{outcolor}36}]:}    \#           Name Type 1  Type 2  HP  Attack  Defense  Sp. Atk  Sp. Def  \textbackslash{}
         0  1      Bulbasaur  Grass  Poison  45      49       49       65       65   
         1  2        Ivysaur  Grass  Poison  60      62       63       80       80   
         2  3       Venusaur  Grass  Poison  80      82       83      100      100   
         3  4  Mega Venusaur  Grass  Poison  80     100      123      122      120   
         4  5     Charmander   Fire     NaN  39      52       43       60       50   
         
            Speed  Generation  Legendary  
         0     45           1      False  
         1     60           1      False  
         2     80           1      False  
         3     80           1      False  
         4     65           1      False  
\end{Verbatim}
            
    \begin{Verbatim}[commandchars=\\\{\}]
{\color{incolor}In [{\color{incolor}37}]:} \PY{c+c1}{\PYZsh{} lets melt}
         \PY{c+c1}{\PYZsh{} id\PYZus{}vars = what we do not wish to melt}
         \PY{c+c1}{\PYZsh{} value\PYZus{}vars = what we want to melt}
         \PY{n}{melted} \PY{o}{=} \PY{n}{pd}\PY{o}{.}\PY{n}{melt}\PY{p}{(}\PY{n}{frame}\PY{o}{=}\PY{n}{data\PYZus{}new}\PY{p}{,}\PY{n}{id\PYZus{}vars} \PY{o}{=} \PY{l+s+s1}{\PYZsq{}}\PY{l+s+s1}{Name}\PY{l+s+s1}{\PYZsq{}}\PY{p}{,} \PY{n}{value\PYZus{}vars}\PY{o}{=} \PY{p}{[}\PY{l+s+s1}{\PYZsq{}}\PY{l+s+s1}{Attack}\PY{l+s+s1}{\PYZsq{}}\PY{p}{,}\PY{l+s+s1}{\PYZsq{}}\PY{l+s+s1}{Defense}\PY{l+s+s1}{\PYZsq{}}\PY{p}{]}\PY{p}{)}
         \PY{n}{melted}
\end{Verbatim}


\begin{Verbatim}[commandchars=\\\{\}]
{\color{outcolor}Out[{\color{outcolor}37}]:}             Name variable  value
         0      Bulbasaur   Attack     49
         1        Ivysaur   Attack     62
         2       Venusaur   Attack     82
         3  Mega Venusaur   Attack    100
         4     Charmander   Attack     52
         5      Bulbasaur  Defense     49
         6        Ivysaur  Defense     63
         7       Venusaur  Defense     83
         8  Mega Venusaur  Defense    123
         9     Charmander  Defense     43
\end{Verbatim}
            
    \hypertarget{pivoting-data}{%
\subsubsection{PIVOTING DATA}\label{pivoting-data}}

Reverse of melting.

    \begin{Verbatim}[commandchars=\\\{\}]
{\color{incolor}In [{\color{incolor}38}]:} \PY{c+c1}{\PYZsh{} Index is name}
         \PY{c+c1}{\PYZsh{} I want to make that columns are variable}
         \PY{c+c1}{\PYZsh{} Finally values in columns are value}
         \PY{n}{melted}\PY{o}{.}\PY{n}{pivot}\PY{p}{(}\PY{n}{index} \PY{o}{=} \PY{l+s+s1}{\PYZsq{}}\PY{l+s+s1}{Name}\PY{l+s+s1}{\PYZsq{}}\PY{p}{,} \PY{n}{columns} \PY{o}{=} \PY{l+s+s1}{\PYZsq{}}\PY{l+s+s1}{variable}\PY{l+s+s1}{\PYZsq{}}\PY{p}{,}\PY{n}{values}\PY{o}{=}\PY{l+s+s1}{\PYZsq{}}\PY{l+s+s1}{value}\PY{l+s+s1}{\PYZsq{}}\PY{p}{)}
\end{Verbatim}


\begin{Verbatim}[commandchars=\\\{\}]
{\color{outcolor}Out[{\color{outcolor}38}]:} variable       Attack  Defense
         Name                          
         Bulbasaur          49       49
         Charmander         52       43
         Ivysaur            62       63
         Mega Venusaur     100      123
         Venusaur           82       83
\end{Verbatim}
            
    \hypertarget{concatenating-data}{%
\subsubsection{CONCATENATING DATA}\label{concatenating-data}}

We can concatenate two dataframe

    \begin{Verbatim}[commandchars=\\\{\}]
{\color{incolor}In [{\color{incolor}39}]:} \PY{c+c1}{\PYZsh{} Firstly lets create 2 data frame}
         \PY{n}{data1} \PY{o}{=} \PY{n}{data}\PY{o}{.}\PY{n}{head}\PY{p}{(}\PY{p}{)}
         \PY{n}{data2}\PY{o}{=} \PY{n}{data}\PY{o}{.}\PY{n}{tail}\PY{p}{(}\PY{p}{)}
         \PY{n}{conc\PYZus{}data\PYZus{}row} \PY{o}{=} \PY{n}{pd}\PY{o}{.}\PY{n}{concat}\PY{p}{(}\PY{p}{[}\PY{n}{data1}\PY{p}{,}\PY{n}{data2}\PY{p}{]}\PY{p}{,}\PY{n}{axis} \PY{o}{=}\PY{l+m+mi}{0}\PY{p}{,}\PY{n}{ignore\PYZus{}index} \PY{o}{=}\PY{k+kc}{True}\PY{p}{)} \PY{c+c1}{\PYZsh{} axis = 0 : adds dataframes in row}
         \PY{n}{conc\PYZus{}data\PYZus{}row}
\end{Verbatim}


\begin{Verbatim}[commandchars=\\\{\}]
{\color{outcolor}Out[{\color{outcolor}39}]:}      \#            Name   Type 1  Type 2  HP  Attack  Defense  Sp. Atk  \textbackslash{}
         0    1       Bulbasaur    Grass  Poison  45      49       49       65   
         1    2         Ivysaur    Grass  Poison  60      62       63       80   
         2    3        Venusaur    Grass  Poison  80      82       83      100   
         3    4   Mega Venusaur    Grass  Poison  80     100      123      122   
         4    5      Charmander     Fire     NaN  39      52       43       60   
         5  796         Diancie     Rock   Fairy  50     100      150      100   
         6  797    Mega Diancie     Rock   Fairy  50     160      110      160   
         7  798  Hoopa Confined  Psychic   Ghost  80     110       60      150   
         8  799   Hoopa Unbound  Psychic    Dark  80     160       60      170   
         9  800       Volcanion     Fire   Water  80     110      120      130   
         
            Sp. Def  Speed  Generation  Legendary  
         0       65     45           1      False  
         1       80     60           1      False  
         2      100     80           1      False  
         3      120     80           1      False  
         4       50     65           1      False  
         5      150     50           6       True  
         6      110    110           6       True  
         7      130     70           6       True  
         8      130     80           6       True  
         9       90     70           6       True  
\end{Verbatim}
            
    \begin{Verbatim}[commandchars=\\\{\}]
{\color{incolor}In [{\color{incolor}40}]:} \PY{n}{data1} \PY{o}{=} \PY{n}{data}\PY{p}{[}\PY{l+s+s1}{\PYZsq{}}\PY{l+s+s1}{Attack}\PY{l+s+s1}{\PYZsq{}}\PY{p}{]}\PY{o}{.}\PY{n}{head}\PY{p}{(}\PY{p}{)}
         \PY{n}{data2}\PY{o}{=} \PY{n}{data}\PY{p}{[}\PY{l+s+s1}{\PYZsq{}}\PY{l+s+s1}{Defense}\PY{l+s+s1}{\PYZsq{}}\PY{p}{]}\PY{o}{.}\PY{n}{head}\PY{p}{(}\PY{p}{)}
         \PY{n}{conc\PYZus{}data\PYZus{}col} \PY{o}{=} \PY{n}{pd}\PY{o}{.}\PY{n}{concat}\PY{p}{(}\PY{p}{[}\PY{n}{data1}\PY{p}{,}\PY{n}{data2}\PY{p}{]}\PY{p}{,}\PY{n}{axis} \PY{o}{=}\PY{l+m+mi}{1}\PY{p}{)} \PY{c+c1}{\PYZsh{} axis = 0 : adds dataframes in row}
         \PY{n}{conc\PYZus{}data\PYZus{}col}
\end{Verbatim}


\begin{Verbatim}[commandchars=\\\{\}]
{\color{outcolor}Out[{\color{outcolor}40}]:}    Attack  Defense
         0      49       49
         1      62       63
         2      82       83
         3     100      123
         4      52       43
\end{Verbatim}
            
    \hypertarget{data-types}{%
\subsubsection{DATA TYPES}\label{data-types}}

There are 5 basic data types: object(string),booleab, integer, float and
categorical. We can make conversion data types like from str to
categorical or from int to float Why is category important: * make
dataframe smaller in memory * can be utilized for anlaysis especially
for sklear(we will learn later)

    \begin{Verbatim}[commandchars=\\\{\}]
{\color{incolor}In [{\color{incolor}41}]:} \PY{n}{data}\PY{o}{.}\PY{n}{dtypes}
\end{Verbatim}


\begin{Verbatim}[commandchars=\\\{\}]
{\color{outcolor}Out[{\color{outcolor}41}]:} \#              int64
         Name          object
         Type 1        object
         Type 2        object
         HP             int64
         Attack         int64
         Defense        int64
         Sp. Atk        int64
         Sp. Def        int64
         Speed          int64
         Generation     int64
         Legendary       bool
         dtype: object
\end{Verbatim}
            
    \begin{Verbatim}[commandchars=\\\{\}]
{\color{incolor}In [{\color{incolor}42}]:} \PY{c+c1}{\PYZsh{} lets convert object(str) to categorical and int to float.}
         \PY{n}{data}\PY{p}{[}\PY{l+s+s1}{\PYZsq{}}\PY{l+s+s1}{Type 1}\PY{l+s+s1}{\PYZsq{}}\PY{p}{]} \PY{o}{=} \PY{n}{data}\PY{p}{[}\PY{l+s+s1}{\PYZsq{}}\PY{l+s+s1}{Type 1}\PY{l+s+s1}{\PYZsq{}}\PY{p}{]}\PY{o}{.}\PY{n}{astype}\PY{p}{(}\PY{l+s+s1}{\PYZsq{}}\PY{l+s+s1}{category}\PY{l+s+s1}{\PYZsq{}}\PY{p}{)}
         \PY{n}{data}\PY{p}{[}\PY{l+s+s1}{\PYZsq{}}\PY{l+s+s1}{Speed}\PY{l+s+s1}{\PYZsq{}}\PY{p}{]} \PY{o}{=} \PY{n}{data}\PY{p}{[}\PY{l+s+s1}{\PYZsq{}}\PY{l+s+s1}{Speed}\PY{l+s+s1}{\PYZsq{}}\PY{p}{]}\PY{o}{.}\PY{n}{astype}\PY{p}{(}\PY{l+s+s1}{\PYZsq{}}\PY{l+s+s1}{float}\PY{l+s+s1}{\PYZsq{}}\PY{p}{)}
\end{Verbatim}


    \begin{Verbatim}[commandchars=\\\{\}]
{\color{incolor}In [{\color{incolor}65}]:} \PY{c+c1}{\PYZsh{} As you can see Type 1 is converted from object to categorical}
         \PY{c+c1}{\PYZsh{} And Speed ,s converted from int to float}
         \PY{n}{data}\PY{o}{.}\PY{n}{dtypes}
\end{Verbatim}


\begin{Verbatim}[commandchars=\\\{\}]
{\color{outcolor}Out[{\color{outcolor}65}]:} \#                int64
         Name            object
         Type 1        category
         Type 2          object
         HP               int64
         Attack           int64
         Defense          int64
         Sp. Atk          int64
         Sp. Def          int64
         Speed          float64
         Generation       int64
         Legendary         bool
         dtype: object
\end{Verbatim}
            
    \hypertarget{missing-data-and-testing-with-assert}{%
\subsubsection{MISSING DATA and TESTING WITH
ASSERT}\label{missing-data-and-testing-with-assert}}

If we encounter with missing data, what we can do: * leave as is * drop
them with dropna() * fill missing value with fillna() * fill missing
values with test statistics like mean Assert statement: check that you
can turn on or turn off when you are done with your testing of the
program

    \begin{Verbatim}[commandchars=\\\{\}]
{\color{incolor}In [{\color{incolor}67}]:} \PY{c+c1}{\PYZsh{} Lets look at does pokemon data have nan value}
         \PY{c+c1}{\PYZsh{} As you can see there are 800 entries. However Type 2 has 414 non\PYZhy{}null object so it has 386 null object.}
         \PY{n}{data}\PY{o}{.}\PY{n}{info}\PY{p}{(}\PY{p}{)}
\end{Verbatim}


    \begin{Verbatim}[commandchars=\\\{\}]
<class 'pandas.core.frame.DataFrame'>
RangeIndex: 800 entries, 0 to 799
Data columns (total 12 columns):
\#             800 non-null int64
Name          799 non-null object
Type 1        800 non-null object
Type 2        414 non-null object
HP            800 non-null int64
Attack        800 non-null int64
Defense       800 non-null int64
Sp. Atk       800 non-null int64
Sp. Def       800 non-null int64
Speed         800 non-null int64
Generation    800 non-null int64
Legendary     800 non-null bool
dtypes: bool(1), int64(8), object(3)
memory usage: 69.6+ KB

    \end{Verbatim}

    \begin{Verbatim}[commandchars=\\\{\}]
{\color{incolor}In [{\color{incolor}85}]:} \PY{c+c1}{\PYZsh{} Lets chech Type 2}
         \PY{n}{data}\PY{p}{[}\PY{l+s+s2}{\PYZdq{}}\PY{l+s+s2}{Type 2}\PY{l+s+s2}{\PYZdq{}}\PY{p}{]}\PY{o}{.}\PY{n}{value\PYZus{}counts}\PY{p}{(}\PY{n}{dropna} \PY{o}{=}\PY{k+kc}{False}\PY{p}{)}
         \PY{c+c1}{\PYZsh{} As you can see, there are 386 NAN value}
\end{Verbatim}


\begin{Verbatim}[commandchars=\\\{\}]
{\color{outcolor}Out[{\color{outcolor}85}]:} NaN         386
         Flying       97
         Ground       35
         Poison       34
         Psychic      33
         Fighting     26
         Grass        25
         Fairy        23
         Steel        22
         Dark         20
         Dragon       18
         Ice          14
         Water        14
         Ghost        14
         Rock         14
         Fire         12
         Electric      6
         Normal        4
         Bug           3
         Name: Type 2, dtype: int64
\end{Verbatim}
            
    \begin{Verbatim}[commandchars=\\\{\}]
{\color{incolor}In [{\color{incolor}77}]:} \PY{c+c1}{\PYZsh{} Lets drop nan values}
         \PY{n}{data1}\PY{o}{=}\PY{n}{data}\PY{o}{.}\PY{n}{copy}\PY{p}{(}\PY{p}{)}   \PY{c+c1}{\PYZsh{} also we will use data to fill missing value so I assign it to data1 variable}
         \PY{n}{data1}\PY{o}{.}\PY{n}{dropna}\PY{p}{(}\PY{n}{subset}\PY{o}{=}\PY{p}{[}\PY{l+s+s1}{\PYZsq{}}\PY{l+s+s1}{Type 2}\PY{l+s+s1}{\PYZsq{}}\PY{p}{]}\PY{p}{,}\PY{n}{inplace} \PY{o}{=} \PY{k+kc}{True}\PY{p}{)}  \PY{c+c1}{\PYZsh{} inplace = True means we do not assign it to new variable. Changes automatically assigned to data}
         \PY{c+c1}{\PYZsh{} So does it work ?}
         \PY{n}{data1}\PY{o}{.}\PY{n}{info}\PY{p}{(}\PY{p}{)}
\end{Verbatim}


    \begin{Verbatim}[commandchars=\\\{\}]
<class 'pandas.core.frame.DataFrame'>
Int64Index: 414 entries, 0 to 799
Data columns (total 12 columns):
\#             414 non-null int64
Name          414 non-null object
Type 1        414 non-null object
Type 2        414 non-null object
HP            414 non-null int64
Attack        414 non-null int64
Defense       414 non-null int64
Sp. Atk       414 non-null int64
Sp. Def       414 non-null int64
Speed         414 non-null int64
Generation    414 non-null int64
Legendary     414 non-null bool
dtypes: bool(1), int64(8), object(3)
memory usage: 39.2+ KB

    \end{Verbatim}

    \begin{Verbatim}[commandchars=\\\{\}]
{\color{incolor}In [{\color{incolor}48}]:} \PY{c+c1}{\PYZsh{}  Lets check with assert statement}
         \PY{c+c1}{\PYZsh{} Assert statement:}
         \PY{k}{assert} \PY{l+m+mi}{1}\PY{o}{==}\PY{l+m+mi}{1} \PY{c+c1}{\PYZsh{} return nothing because it is true}
\end{Verbatim}


    \begin{Verbatim}[commandchars=\\\{\}]
{\color{incolor}In [{\color{incolor}81}]:} \PY{c+c1}{\PYZsh{} In order to run all code, we need to make this line comment}
         \PY{c+c1}{\PYZsh{} assert 1==2 \PYZsh{} return error because it is false}
         \PY{k}{assert} \PY{n}{data1}\PY{p}{[}\PY{l+s+s1}{\PYZsq{}}\PY{l+s+s1}{Type 2}\PY{l+s+s1}{\PYZsq{}}\PY{p}{]}\PY{o}{.}\PY{n}{notnull}\PY{p}{(}\PY{p}{)}\PY{o}{.}\PY{n}{all}\PY{p}{(}\PY{p}{)}
\end{Verbatim}


\begin{Verbatim}[commandchars=\\\{\}]
{\color{outcolor}Out[{\color{outcolor}81}]:} True
\end{Verbatim}
            
    \begin{Verbatim}[commandchars=\\\{\}]
{\color{incolor}In [{\color{incolor}89}]:} \PY{n}{data2} \PY{o}{=} \PY{n}{data}\PY{o}{.}\PY{n}{copy}\PY{p}{(}\PY{p}{)}
         \PY{n}{data2}\PY{p}{[}\PY{l+s+s2}{\PYZdq{}}\PY{l+s+s2}{Type 2}\PY{l+s+s2}{\PYZdq{}}\PY{p}{]}\PY{o}{.}\PY{n}{fillna}\PY{p}{(}\PY{l+s+s1}{\PYZsq{}}\PY{l+s+s1}{empty}\PY{l+s+s1}{\PYZsq{}}\PY{p}{,}\PY{n}{inplace} \PY{o}{=} \PY{k+kc}{True}\PY{p}{)}
         \PY{n}{data2}\PY{o}{.}\PY{n}{head}\PY{p}{(}\PY{p}{)}
\end{Verbatim}


\begin{Verbatim}[commandchars=\\\{\}]
{\color{outcolor}Out[{\color{outcolor}89}]:}    \#           Name Type 1  Type 2  HP  Attack  Defense  Sp. Atk  Sp. Def  \textbackslash{}
         0  1      Bulbasaur  Grass  Poison  45      49       49       65       65   
         1  2        Ivysaur  Grass  Poison  60      62       63       80       80   
         2  3       Venusaur  Grass  Poison  80      82       83      100      100   
         3  4  Mega Venusaur  Grass  Poison  80     100      123      122      120   
         4  5     Charmander   Fire   empty  39      52       43       60       50   
         
            Speed  Generation  Legendary  
         0     45           1      False  
         1     60           1      False  
         2     80           1      False  
         3     80           1      False  
         4     65           1      False  
\end{Verbatim}
            
    \begin{Verbatim}[commandchars=\\\{\}]
{\color{incolor}In [{\color{incolor}91}]:} \PY{k}{assert} \PY{n}{data2}\PY{p}{[}\PY{l+s+s1}{\PYZsq{}}\PY{l+s+s1}{Type 2}\PY{l+s+s1}{\PYZsq{}}\PY{p}{]}\PY{o}{.}\PY{n}{notnull}\PY{p}{(}\PY{p}{)}\PY{o}{.}\PY{n}{all}\PY{p}{(}\PY{p}{)} \PY{c+c1}{\PYZsh{} returns nothing because we drop nan values}
\end{Verbatim}


    \begin{Verbatim}[commandchars=\\\{\}]
{\color{incolor}In [{\color{incolor} }]:} \PY{c+c1}{\PYZsh{} \PYZsh{} With assert statement we can check a lot of thing. For example}
        \PY{c+c1}{\PYZsh{} assert data.columns[1] == \PYZsq{}Name\PYZsq{}}
        \PY{c+c1}{\PYZsh{} assert data.Speed.dtypes == np.int}
\end{Verbatim}


    In this part, you learn: * Diagnose data for cleaning * Explotary data
analysis * Visual exploratory data analysis * Tidy data * Pivoting data
* Concatenating data * Data types * Missing data and testing with assert

    \hypertarget{pandas-foundation}{%
\section{4. PANDAS FOUNDATION}\label{pandas-foundation}}

    \hypertarget{review-of-pandas}{%
\subsubsection{REVİEW of PANDAS}\label{review-of-pandas}}

As you notice, I do not give all idea in a same time. Although, we learn
some basics of pandas, we will go deeper in pandas. * single column =
series * NaN = not a number * dataframe.values = numpy

    \hypertarget{building-data-frames-from-scratch}{%
\subsubsection{BUILDING DATA FRAMES FROM
SCRATCH}\label{building-data-frames-from-scratch}}

\begin{itemize}
\tightlist
\item
  We can build data frames from csv as we did earlier.
\item
  Also we can build dataframe from dictionaries

  \begin{itemize}
  \tightlist
  \item
    zip() method: This function returns a list of tuples, where the i-th
    tuple contains the i-th element from each of the argument sequences
    or iterables.
  \end{itemize}
\item
  Adding new column
\item
  Broadcasting: Create new column and assign a value to entire column
\end{itemize}

    \begin{Verbatim}[commandchars=\\\{\}]
{\color{incolor}In [{\color{incolor}101}]:} \PY{c+c1}{\PYZsh{} data frames from dictionary}
          \PY{n}{country} \PY{o}{=} \PY{p}{[}\PY{l+s+s2}{\PYZdq{}}\PY{l+s+s2}{Spain}\PY{l+s+s2}{\PYZdq{}}\PY{p}{,}\PY{l+s+s2}{\PYZdq{}}\PY{l+s+s2}{France}\PY{l+s+s2}{\PYZdq{}}\PY{p}{]}
          \PY{n}{population} \PY{o}{=} \PY{p}{[}\PY{l+s+s2}{\PYZdq{}}\PY{l+s+s2}{11}\PY{l+s+s2}{\PYZdq{}}\PY{p}{,}\PY{l+s+s2}{\PYZdq{}}\PY{l+s+s2}{12}\PY{l+s+s2}{\PYZdq{}}\PY{p}{]}
          \PY{n}{list\PYZus{}label} \PY{o}{=} \PY{p}{[}\PY{l+s+s2}{\PYZdq{}}\PY{l+s+s2}{country}\PY{l+s+s2}{\PYZdq{}}\PY{p}{,}\PY{l+s+s2}{\PYZdq{}}\PY{l+s+s2}{population}\PY{l+s+s2}{\PYZdq{}}\PY{p}{]}
          \PY{n}{list\PYZus{}col} \PY{o}{=} \PY{p}{[}\PY{n}{country}\PY{p}{,}\PY{n}{population}\PY{p}{]}
          \PY{n}{zipped} \PY{o}{=} \PY{n+nb}{zip}\PY{p}{(}\PY{n}{list\PYZus{}label}\PY{p}{,}\PY{n}{list\PYZus{}col}\PY{p}{)}
          \PY{n+nb}{print}\PY{p}{(}\PY{p}{[}\PY{n}{z} \PY{k}{for} \PY{n}{z} \PY{o+ow}{in} \PY{n}{zipped}\PY{p}{]}\PY{p}{)}
          \PY{n+nb}{print}\PY{p}{(}\PY{n}{data\PYZus{}dict}\PY{p}{)}
          \PY{n}{data\PYZus{}dict} \PY{o}{=} \PY{n+nb}{dict}\PY{p}{(}\PY{n}{zipped}\PY{p}{)}
          \PY{n}{df} \PY{o}{=} \PY{n}{pd}\PY{o}{.}\PY{n}{DataFrame}\PY{p}{(}\PY{n}{data\PYZus{}dict}\PY{p}{)}
          \PY{n}{df}
\end{Verbatim}


    \begin{Verbatim}[commandchars=\\\{\}]
[('country', ['Spain', 'France']), ('population', ['11', '12'])]
\{'country': ['Spain', 'France'], 'population': ['11', '12']\}

    \end{Verbatim}

\begin{Verbatim}[commandchars=\\\{\}]
{\color{outcolor}Out[{\color{outcolor}101}]:} Empty DataFrame
          Columns: []
          Index: []
\end{Verbatim}
            
    \begin{Verbatim}[commandchars=\\\{\}]
{\color{incolor}In [{\color{incolor} }]:} \PY{c+c1}{\PYZsh{} Add new columns}
        \PY{n}{df}\PY{p}{[}\PY{l+s+s2}{\PYZdq{}}\PY{l+s+s2}{capital}\PY{l+s+s2}{\PYZdq{}}\PY{p}{]} \PY{o}{=} \PY{p}{[}\PY{l+s+s2}{\PYZdq{}}\PY{l+s+s2}{madrid}\PY{l+s+s2}{\PYZdq{}}\PY{p}{,}\PY{l+s+s2}{\PYZdq{}}\PY{l+s+s2}{paris}\PY{l+s+s2}{\PYZdq{}}\PY{p}{]}
        \PY{n}{df}
\end{Verbatim}


    \begin{Verbatim}[commandchars=\\\{\}]
{\color{incolor}In [{\color{incolor} }]:} \PY{c+c1}{\PYZsh{} Broadcasting}
        \PY{n}{df}\PY{p}{[}\PY{l+s+s2}{\PYZdq{}}\PY{l+s+s2}{income}\PY{l+s+s2}{\PYZdq{}}\PY{p}{]} \PY{o}{=} \PY{l+m+mi}{0} \PY{c+c1}{\PYZsh{}Broadcasting entire column}
        \PY{n}{df}
\end{Verbatim}


    \hypertarget{visual-exploratory-data-analysis}{%
\subsubsection{VISUAL EXPLORATORY DATA
ANALYSIS}\label{visual-exploratory-data-analysis}}

\begin{itemize}
\tightlist
\item
  Plot
\item
  Subplot
\item
  Histogram:

  \begin{itemize}
  \tightlist
  \item
    bins: number of bins
  \item
    range(tuble): min and max values of bins
  \item
    normed(boolean): normalize or not
  \item
    cumulative(boolean): compute cumulative distribution
  \end{itemize}
\end{itemize}

    \begin{Verbatim}[commandchars=\\\{\}]
{\color{incolor}In [{\color{incolor}102}]:} \PY{c+c1}{\PYZsh{} Plotting all data }
          \PY{n}{data1} \PY{o}{=} \PY{n}{data}\PY{o}{.}\PY{n}{loc}\PY{p}{[}\PY{p}{:}\PY{p}{,}\PY{p}{[}\PY{l+s+s2}{\PYZdq{}}\PY{l+s+s2}{Attack}\PY{l+s+s2}{\PYZdq{}}\PY{p}{,}\PY{l+s+s2}{\PYZdq{}}\PY{l+s+s2}{Defense}\PY{l+s+s2}{\PYZdq{}}\PY{p}{,}\PY{l+s+s2}{\PYZdq{}}\PY{l+s+s2}{Speed}\PY{l+s+s2}{\PYZdq{}}\PY{p}{]}\PY{p}{]}
          \PY{n}{data1}\PY{o}{.}\PY{n}{plot}\PY{p}{(}\PY{p}{)}
          \PY{c+c1}{\PYZsh{} it is confusing}
\end{Verbatim}


\begin{Verbatim}[commandchars=\\\{\}]
{\color{outcolor}Out[{\color{outcolor}102}]:} <matplotlib.axes.\_subplots.AxesSubplot at 0x10be62550>
\end{Verbatim}
            
    \begin{center}
    \adjustimage{max size={0.9\linewidth}{0.9\paperheight}}{output_102_1.png}
    \end{center}
    { \hspace*{\fill} \\}
    
    \begin{Verbatim}[commandchars=\\\{\}]
{\color{incolor}In [{\color{incolor}103}]:} \PY{c+c1}{\PYZsh{} subplots}
          \PY{n}{data1}\PY{o}{.}\PY{n}{plot}\PY{p}{(}\PY{n}{subplots} \PY{o}{=} \PY{k+kc}{True}\PY{p}{)}
\end{Verbatim}


\begin{Verbatim}[commandchars=\\\{\}]
{\color{outcolor}Out[{\color{outcolor}103}]:} array([<matplotlib.axes.\_subplots.AxesSubplot object at 0x10bf70358>,
                 <matplotlib.axes.\_subplots.AxesSubplot object at 0x10bf09898>,
                 <matplotlib.axes.\_subplots.AxesSubplot object at 0x10bf30a58>],
                dtype=object)
\end{Verbatim}
            
    \begin{center}
    \adjustimage{max size={0.9\linewidth}{0.9\paperheight}}{output_103_1.png}
    \end{center}
    { \hspace*{\fill} \\}
    
    \begin{Verbatim}[commandchars=\\\{\}]
{\color{incolor}In [{\color{incolor}104}]:} \PY{c+c1}{\PYZsh{} scatter plot  }
          \PY{n}{data1}\PY{o}{.}\PY{n}{plot}\PY{p}{(}\PY{n}{kind} \PY{o}{=} \PY{l+s+s2}{\PYZdq{}}\PY{l+s+s2}{scatter}\PY{l+s+s2}{\PYZdq{}}\PY{p}{,}\PY{n}{x}\PY{o}{=}\PY{l+s+s2}{\PYZdq{}}\PY{l+s+s2}{Attack}\PY{l+s+s2}{\PYZdq{}}\PY{p}{,}\PY{n}{y} \PY{o}{=} \PY{l+s+s2}{\PYZdq{}}\PY{l+s+s2}{Defense}\PY{l+s+s2}{\PYZdq{}}\PY{p}{)}
\end{Verbatim}


\begin{Verbatim}[commandchars=\\\{\}]
{\color{outcolor}Out[{\color{outcolor}104}]:} <matplotlib.axes.\_subplots.AxesSubplot at 0x10c3edda0>
\end{Verbatim}
            
    \begin{center}
    \adjustimage{max size={0.9\linewidth}{0.9\paperheight}}{output_104_1.png}
    \end{center}
    { \hspace*{\fill} \\}
    
    \begin{Verbatim}[commandchars=\\\{\}]
{\color{incolor}In [{\color{incolor}114}]:} \PY{c+c1}{\PYZsh{} hist plot  }
          \PY{c+c1}{\PYZsh{}data1.hist()}
          \PY{c+c1}{\PYZsh{}data1.plot(kind = \PYZdq{}hist\PYZdq{},y = \PYZdq{}Defense\PYZdq{},bins = 50,range= (0,250))}
          \PY{c+c1}{\PYZsh{}data1.plot(kind=\PYZdq{}hist\PYZdq{},stacked=True)}
          \PY{n}{data1}\PY{o}{.}\PY{n}{plot}\PY{p}{(}\PY{n}{kind}\PY{o}{=}\PY{l+s+s2}{\PYZdq{}}\PY{l+s+s2}{hist}\PY{l+s+s2}{\PYZdq{}}\PY{p}{,}\PY{n}{subplots}\PY{o}{=}\PY{k+kc}{True}\PY{p}{)}
\end{Verbatim}


\begin{Verbatim}[commandchars=\\\{\}]
{\color{outcolor}Out[{\color{outcolor}114}]:} array([<matplotlib.axes.\_subplots.AxesSubplot object at 0x10de2d048>,
                 <matplotlib.axes.\_subplots.AxesSubplot object at 0x10dd9d470>,
                 <matplotlib.axes.\_subplots.AxesSubplot object at 0x10ddc59e8>],
                dtype=object)
\end{Verbatim}
            
    \begin{center}
    \adjustimage{max size={0.9\linewidth}{0.9\paperheight}}{output_105_1.png}
    \end{center}
    { \hspace*{\fill} \\}
    
    \begin{Verbatim}[commandchars=\\\{\}]
{\color{incolor}In [{\color{incolor}115}]:} \PY{c+c1}{\PYZsh{} histogram subplot with non cumulative and cumulative}
          \PY{n}{fig}\PY{p}{,} \PY{n}{axes} \PY{o}{=} \PY{n}{plt}\PY{o}{.}\PY{n}{subplots}\PY{p}{(}\PY{n}{nrows}\PY{o}{=}\PY{l+m+mi}{2}\PY{p}{,}\PY{n}{ncols}\PY{o}{=}\PY{l+m+mi}{1}\PY{p}{)}
          \PY{n}{data1}\PY{o}{.}\PY{n}{plot}\PY{p}{(}\PY{n}{kind} \PY{o}{=} \PY{l+s+s2}{\PYZdq{}}\PY{l+s+s2}{hist}\PY{l+s+s2}{\PYZdq{}}\PY{p}{,}\PY{n}{y} \PY{o}{=} \PY{l+s+s2}{\PYZdq{}}\PY{l+s+s2}{Defense}\PY{l+s+s2}{\PYZdq{}}\PY{p}{,}\PY{n}{bins} \PY{o}{=} \PY{l+m+mi}{50}\PY{p}{,}\PY{n+nb}{range}\PY{o}{=} \PY{p}{(}\PY{l+m+mi}{0}\PY{p}{,}\PY{l+m+mi}{250}\PY{p}{)}\PY{p}{,}\PY{n}{normed} \PY{o}{=} \PY{k+kc}{True}\PY{p}{,}\PY{n}{ax} \PY{o}{=} \PY{n}{axes}\PY{p}{[}\PY{l+m+mi}{0}\PY{p}{]}\PY{p}{)}
          \PY{n}{data1}\PY{o}{.}\PY{n}{plot}\PY{p}{(}\PY{n}{kind} \PY{o}{=} \PY{l+s+s2}{\PYZdq{}}\PY{l+s+s2}{hist}\PY{l+s+s2}{\PYZdq{}}\PY{p}{,}\PY{n}{y} \PY{o}{=} \PY{l+s+s2}{\PYZdq{}}\PY{l+s+s2}{Defense}\PY{l+s+s2}{\PYZdq{}}\PY{p}{,}\PY{n}{bins} \PY{o}{=} \PY{l+m+mi}{50}\PY{p}{,}\PY{n+nb}{range}\PY{o}{=} \PY{p}{(}\PY{l+m+mi}{0}\PY{p}{,}\PY{l+m+mi}{250}\PY{p}{)}\PY{p}{,}\PY{n}{normed} \PY{o}{=} \PY{k+kc}{True}\PY{p}{,}\PY{n}{ax} \PY{o}{=} \PY{n}{axes}\PY{p}{[}\PY{l+m+mi}{1}\PY{p}{]}\PY{p}{,}\PY{n}{cumulative} \PY{o}{=} \PY{k+kc}{True}\PY{p}{)}
\end{Verbatim}


    \begin{Verbatim}[commandchars=\\\{\}]
/Users/ayim/Desktop/virenv/lib/python3.6/site-packages/matplotlib/axes/\_axes.py:6462: UserWarning: The 'normed' kwarg is deprecated, and has been replaced by the 'density' kwarg.
  warnings.warn("The 'normed' kwarg is deprecated, and has been "

    \end{Verbatim}

\begin{Verbatim}[commandchars=\\\{\}]
{\color{outcolor}Out[{\color{outcolor}115}]:} <matplotlib.axes.\_subplots.AxesSubplot at 0x10dfd4320>
\end{Verbatim}
            
    \begin{center}
    \adjustimage{max size={0.9\linewidth}{0.9\paperheight}}{output_106_2.png}
    \end{center}
    { \hspace*{\fill} \\}
    
    \hypertarget{statistical-exploratory-data-analysis}{%
\subsubsection{STATISTICAL EXPLORATORY DATA
ANALYSIS}\label{statistical-exploratory-data-analysis}}

I already explained it at previous parts. However lets look at one more
time. * count: number of entries * mean: average of entries * std:
standart deviation * min: minimum entry * 25\%: first quantile * 50\%:
median or second quantile * 75\%: third quantile * max: maximum entry

    \begin{Verbatim}[commandchars=\\\{\}]
{\color{incolor}In [{\color{incolor}116}]:} \PY{n}{data}\PY{o}{.}\PY{n}{describe}\PY{p}{(}\PY{p}{)}
\end{Verbatim}


\begin{Verbatim}[commandchars=\\\{\}]
{\color{outcolor}Out[{\color{outcolor}116}]:}               \#          HP      Attack     Defense     Sp. Atk     Sp. Def  \textbackslash{}
          count  800.0000  800.000000  800.000000  800.000000  800.000000  800.000000   
          mean   400.5000   69.258750   79.001250   73.842500   72.820000   71.902500   
          std    231.0844   25.534669   32.457366   31.183501   32.722294   27.828916   
          min      1.0000    1.000000    5.000000    5.000000   10.000000   20.000000   
          25\%    200.7500   50.000000   55.000000   50.000000   49.750000   50.000000   
          50\%    400.5000   65.000000   75.000000   70.000000   65.000000   70.000000   
          75\%    600.2500   80.000000  100.000000   90.000000   95.000000   90.000000   
          max    800.0000  255.000000  190.000000  230.000000  194.000000  230.000000   
          
                      Speed  Generation  
          count  800.000000   800.00000  
          mean    68.277500     3.32375  
          std     29.060474     1.66129  
          min      5.000000     1.00000  
          25\%     45.000000     2.00000  
          50\%     65.000000     3.00000  
          75\%     90.000000     5.00000  
          max    180.000000     6.00000  
\end{Verbatim}
            
    \hypertarget{indexing-pandas-time-series}{%
\subsubsection{INDEXING PANDAS TIME
SERIES}\label{indexing-pandas-time-series}}

\begin{itemize}
\tightlist
\item
  datetime = object
\item
  parse\_dates(boolean): Transform date to ISO 8601 (yyyy-mm-dd hh:mm:ss
  ) format
\end{itemize}

    \begin{Verbatim}[commandchars=\\\{\}]
{\color{incolor}In [{\color{incolor}118}]:} \PY{n}{time\PYZus{}list} \PY{o}{=} \PY{p}{[}\PY{l+s+s2}{\PYZdq{}}\PY{l+s+s2}{1992\PYZhy{}03\PYZhy{}08}\PY{l+s+s2}{\PYZdq{}}\PY{p}{,}\PY{l+s+s2}{\PYZdq{}}\PY{l+s+s2}{1992\PYZhy{}04\PYZhy{}12}\PY{l+s+s2}{\PYZdq{}}\PY{p}{]}
          \PY{n+nb}{print}\PY{p}{(}\PY{n+nb}{type}\PY{p}{(}\PY{n}{time\PYZus{}list}\PY{p}{[}\PY{l+m+mi}{1}\PY{p}{]}\PY{p}{)}\PY{p}{)} \PY{c+c1}{\PYZsh{} As you can see date is string}
          \PY{c+c1}{\PYZsh{} however we want it to be datetime object}
          \PY{n}{datetime\PYZus{}object} \PY{o}{=} \PY{n}{pd}\PY{o}{.}\PY{n}{to\PYZus{}datetime}\PY{p}{(}\PY{n}{time\PYZus{}list}\PY{p}{)}
          \PY{n+nb}{print}\PY{p}{(}\PY{n+nb}{type}\PY{p}{(}\PY{n}{datetime\PYZus{}object}\PY{p}{)}\PY{p}{)}
          \PY{n+nb}{print}\PY{p}{(}\PY{n}{datetime\PYZus{}object}\PY{p}{)}
\end{Verbatim}


    \begin{Verbatim}[commandchars=\\\{\}]
<class 'str'>
<class 'pandas.core.indexes.datetimes.DatetimeIndex'>
DatetimeIndex(['1992-03-08', '1992-04-12'], dtype='datetime64[ns]', freq=None)

    \end{Verbatim}

    \begin{Verbatim}[commandchars=\\\{\}]
{\color{incolor}In [{\color{incolor}126}]:} \PY{c+c1}{\PYZsh{} In order to practice lets take head of pokemon data and add it a time list}
          \PY{n}{data2} \PY{o}{=} \PY{n}{data}\PY{o}{.}\PY{n}{head}\PY{p}{(}\PY{p}{)}
          \PY{n}{date\PYZus{}list} \PY{o}{=} \PY{p}{[}\PY{l+s+s2}{\PYZdq{}}\PY{l+s+s2}{1992\PYZhy{}01\PYZhy{}10}\PY{l+s+s2}{\PYZdq{}}\PY{p}{,}\PY{l+s+s2}{\PYZdq{}}\PY{l+s+s2}{1992\PYZhy{}02\PYZhy{}10}\PY{l+s+s2}{\PYZdq{}}\PY{p}{,}\PY{l+s+s2}{\PYZdq{}}\PY{l+s+s2}{1992\PYZhy{}03\PYZhy{}10}\PY{l+s+s2}{\PYZdq{}}\PY{p}{,}\PY{l+s+s2}{\PYZdq{}}\PY{l+s+s2}{1993\PYZhy{}03\PYZhy{}15}\PY{l+s+s2}{\PYZdq{}}\PY{p}{,}\PY{l+s+s2}{\PYZdq{}}\PY{l+s+s2}{1993\PYZhy{}03\PYZhy{}16}\PY{l+s+s2}{\PYZdq{}}\PY{p}{]}
          \PY{n}{datetime\PYZus{}object} \PY{o}{=} \PY{n}{pd}\PY{o}{.}\PY{n}{to\PYZus{}datetime}\PY{p}{(}\PY{n}{date\PYZus{}list}\PY{p}{)}
          \PY{n}{data2}\PY{p}{[}\PY{l+s+s2}{\PYZdq{}}\PY{l+s+s2}{date}\PY{l+s+s2}{\PYZdq{}}\PY{p}{]} \PY{o}{=} \PY{n}{datetime\PYZus{}object}
          \PY{c+c1}{\PYZsh{} lets make date as index}
          \PY{n}{data2}\PY{o}{=} \PY{n}{data2}\PY{o}{.}\PY{n}{set\PYZus{}index}\PY{p}{(}\PY{l+s+s2}{\PYZdq{}}\PY{l+s+s2}{date}\PY{l+s+s2}{\PYZdq{}}\PY{p}{)}
          \PY{n}{data2} 
\end{Verbatim}


    \begin{Verbatim}[commandchars=\\\{\}]
/Users/ayim/Desktop/virenv/lib/python3.6/site-packages/ipykernel\_launcher.py:5: SettingWithCopyWarning: 
A value is trying to be set on a copy of a slice from a DataFrame.
Try using .loc[row\_indexer,col\_indexer] = value instead

See the caveats in the documentation: http://pandas.pydata.org/pandas-docs/stable/indexing.html\#indexing-view-versus-copy
  """

    \end{Verbatim}

\begin{Verbatim}[commandchars=\\\{\}]
{\color{outcolor}Out[{\color{outcolor}126}]:}             \#           Name Type 1  Type 2  HP  Attack  Defense  Sp. Atk  \textbackslash{}
          date                                                                        
          1992-01-10  1      Bulbasaur  Grass  Poison  45      49       49       65   
          1992-02-10  2        Ivysaur  Grass  Poison  60      62       63       80   
          1992-03-10  3       Venusaur  Grass  Poison  80      82       83      100   
          1993-03-15  4  Mega Venusaur  Grass  Poison  80     100      123      122   
          1993-03-16  5     Charmander   Fire     NaN  39      52       43       60   
          
                      Sp. Def  Speed  Generation  Legendary  
          date                                               
          1992-01-10       65     45           1      False  
          1992-02-10       80     60           1      False  
          1992-03-10      100     80           1      False  
          1993-03-15      120     80           1      False  
          1993-03-16       50     65           1      False  
\end{Verbatim}
            
    \begin{Verbatim}[commandchars=\\\{\}]
{\color{incolor}In [{\color{incolor}122}]:} \PY{c+c1}{\PYZsh{} Now we can select according to our date index}
          \PY{n+nb}{print}\PY{p}{(}\PY{n}{data2}\PY{o}{.}\PY{n}{loc}\PY{p}{[}\PY{l+s+s2}{\PYZdq{}}\PY{l+s+s2}{1993\PYZhy{}03\PYZhy{}16}\PY{l+s+s2}{\PYZdq{}}\PY{p}{]}\PY{p}{)}
          \PY{n+nb}{print}\PY{p}{(}\PY{n}{data2}\PY{o}{.}\PY{n}{loc}\PY{p}{[}\PY{l+s+s2}{\PYZdq{}}\PY{l+s+s2}{1992\PYZhy{}03\PYZhy{}10}\PY{l+s+s2}{\PYZdq{}}\PY{p}{:}\PY{l+s+s2}{\PYZdq{}}\PY{l+s+s2}{1993\PYZhy{}03\PYZhy{}16}\PY{l+s+s2}{\PYZdq{}}\PY{p}{]}\PY{p}{)}
\end{Verbatim}


    \begin{Verbatim}[commandchars=\\\{\}]
\#                      5
Name          Charmander
Type 1              Fire
Type 2               NaN
HP                    39
Attack                52
Defense               43
Sp. Atk               60
Sp. Def               50
Speed                 65
Generation             1
Legendary          False
Name: 1993-03-16 00:00:00, dtype: object
            \#           Name Type 1  Type 2  HP  Attack  Defense  Sp. Atk  \textbackslash{}
date                                                                        
1992-03-10  3       Venusaur  Grass  Poison  80      82       83      100   
1993-03-15  4  Mega Venusaur  Grass  Poison  80     100      123      122   
1993-03-16  5     Charmander   Fire     NaN  39      52       43       60   

            Sp. Def  Speed  Generation  Legendary  
date                                               
1992-03-10      100     80           1      False  
1993-03-15      120     80           1      False  
1993-03-16       50     65           1      False  

    \end{Verbatim}

    \hypertarget{resampling-pandas-time-series}{%
\subsubsection{RESAMPLING PANDAS TIME
SERIES}\label{resampling-pandas-time-series}}

\begin{itemize}
\tightlist
\item
  Resampling: statistical method over different time intervals

  \begin{itemize}
  \tightlist
  \item
    Needs string to specify frequency like ``M'' = month or ``A'' = year
  \end{itemize}
\item
  Downsampling: reduce date time rows to slower frequency like from
  daily to weekly
\item
  Upsampling: increase date time rows to faster frequency like from
  daily to hourly
\item
  Interpolate: Interpolate values according to different methods like
  `linear', `time' or index'

  \begin{itemize}
  \tightlist
  \item
    https://pandas.pydata.org/pandas-docs/stable/generated/pandas.Series.interpolate.html
  \end{itemize}
\end{itemize}

    \begin{Verbatim}[commandchars=\\\{\}]
{\color{incolor}In [{\color{incolor}130}]:} \PY{c+c1}{\PYZsh{} We will use data2 that we create at previous part}
          \PY{n}{data2}\PY{o}{.}\PY{n}{resample}\PY{p}{(}\PY{l+s+s2}{\PYZdq{}}\PY{l+s+s2}{A}\PY{l+s+s2}{\PYZdq{}}\PY{p}{)}\PY{o}{.}\PY{n}{mean}\PY{p}{(}\PY{p}{)}
\end{Verbatim}


\begin{Verbatim}[commandchars=\\\{\}]
{\color{outcolor}Out[{\color{outcolor}130}]:}               \#         HP     Attack  Defense    Sp. Atk    Sp. Def  \textbackslash{}
          date                                                                   
          1992-12-31  2.0  61.666667  64.333333     65.0  81.666667  81.666667   
          1993-12-31  4.5  59.500000  76.000000     83.0  91.000000  85.000000   
          
                          Speed  Generation  Legendary  
          date                                          
          1992-12-31  61.666667         1.0      False  
          1993-12-31  72.500000         1.0      False  
\end{Verbatim}
            
    \begin{Verbatim}[commandchars=\\\{\}]
{\color{incolor}In [{\color{incolor}128}]:} \PY{c+c1}{\PYZsh{} Lets resample with month}
          \PY{n}{data2}\PY{o}{.}\PY{n}{resample}\PY{p}{(}\PY{l+s+s2}{\PYZdq{}}\PY{l+s+s2}{M}\PY{l+s+s2}{\PYZdq{}}\PY{p}{)}\PY{o}{.}\PY{n}{mean}\PY{p}{(}\PY{p}{)}
          \PY{c+c1}{\PYZsh{} As you can see there are a lot of nan because data2 does not include all months}
\end{Verbatim}


\begin{Verbatim}[commandchars=\\\{\}]
{\color{outcolor}Out[{\color{outcolor}128}]:}               \#    HP  Attack  Defense  Sp. Atk  Sp. Def  Speed  Generation  \textbackslash{}
          date                                                                          
          1992-01-31  1.0  45.0    49.0     49.0     65.0     65.0   45.0         1.0   
          1992-02-29  2.0  60.0    62.0     63.0     80.0     80.0   60.0         1.0   
          1992-03-31  3.0  80.0    82.0     83.0    100.0    100.0   80.0         1.0   
          1992-04-30  NaN   NaN     NaN      NaN      NaN      NaN    NaN         NaN   
          1992-05-31  NaN   NaN     NaN      NaN      NaN      NaN    NaN         NaN   
          1992-06-30  NaN   NaN     NaN      NaN      NaN      NaN    NaN         NaN   
          1992-07-31  NaN   NaN     NaN      NaN      NaN      NaN    NaN         NaN   
          1992-08-31  NaN   NaN     NaN      NaN      NaN      NaN    NaN         NaN   
          1992-09-30  NaN   NaN     NaN      NaN      NaN      NaN    NaN         NaN   
          1992-10-31  NaN   NaN     NaN      NaN      NaN      NaN    NaN         NaN   
          1992-11-30  NaN   NaN     NaN      NaN      NaN      NaN    NaN         NaN   
          1992-12-31  NaN   NaN     NaN      NaN      NaN      NaN    NaN         NaN   
          1993-01-31  NaN   NaN     NaN      NaN      NaN      NaN    NaN         NaN   
          1993-02-28  NaN   NaN     NaN      NaN      NaN      NaN    NaN         NaN   
          1993-03-31  4.5  59.5    76.0     83.0     91.0     85.0   72.5         1.0   
          
                      Legendary  
          date                   
          1992-01-31        0.0  
          1992-02-29        0.0  
          1992-03-31        0.0  
          1992-04-30        NaN  
          1992-05-31        NaN  
          1992-06-30        NaN  
          1992-07-31        NaN  
          1992-08-31        NaN  
          1992-09-30        NaN  
          1992-10-31        NaN  
          1992-11-30        NaN  
          1992-12-31        NaN  
          1993-01-31        NaN  
          1993-02-28        NaN  
          1993-03-31        0.0  
\end{Verbatim}
            
    \begin{Verbatim}[commandchars=\\\{\}]
{\color{incolor}In [{\color{incolor}133}]:} \PY{c+c1}{\PYZsh{} In real life (data is real. Not created from us like data2) we can solve this problem with interpolate}
          \PY{c+c1}{\PYZsh{} We can interpolete from first value}
          \PY{n}{data2}\PY{o}{.}\PY{n}{resample}\PY{p}{(}\PY{l+s+s2}{\PYZdq{}}\PY{l+s+s2}{M}\PY{l+s+s2}{\PYZdq{}}\PY{p}{)}\PY{o}{.}\PY{n}{first}\PY{p}{(}\PY{p}{)}\PY{o}{.}\PY{n}{interpolate}\PY{p}{(}\PY{l+s+s2}{\PYZdq{}}\PY{l+s+s2}{linear}\PY{l+s+s2}{\PYZdq{}}\PY{p}{)}
\end{Verbatim}


\begin{Verbatim}[commandchars=\\\{\}]
{\color{outcolor}Out[{\color{outcolor}133}]:}                    \#           Name Type 1  Type 2    HP  Attack     Defense  \textbackslash{}
          date                                                                           
          1992-01-31  1.000000      Bulbasaur  Grass  Poison  45.0    49.0   49.000000   
          1992-02-29  2.000000        Ivysaur  Grass  Poison  60.0    62.0   63.000000   
          1992-03-31  3.000000       Venusaur  Grass  Poison  80.0    82.0   83.000000   
          1992-04-30  3.083333            NaN    NaN     NaN  80.0    83.5   86.333333   
          1992-05-31  3.166667            NaN    NaN     NaN  80.0    85.0   89.666667   
          1992-06-30  3.250000            NaN    NaN     NaN  80.0    86.5   93.000000   
          1992-07-31  3.333333            NaN    NaN     NaN  80.0    88.0   96.333333   
          1992-08-31  3.416667            NaN    NaN     NaN  80.0    89.5   99.666667   
          1992-09-30  3.500000            NaN    NaN     NaN  80.0    91.0  103.000000   
          1992-10-31  3.583333            NaN    NaN     NaN  80.0    92.5  106.333333   
          1992-11-30  3.666667            NaN    NaN     NaN  80.0    94.0  109.666667   
          1992-12-31  3.750000            NaN    NaN     NaN  80.0    95.5  113.000000   
          1993-01-31  3.833333            NaN    NaN     NaN  80.0    97.0  116.333333   
          1993-02-28  3.916667            NaN    NaN     NaN  80.0    98.5  119.666667   
          1993-03-31  4.000000  Mega Venusaur  Grass  Poison  80.0   100.0  123.000000   
          
                         Sp. Atk     Sp. Def  Speed  Generation Legendary  
          date                                                             
          1992-01-31   65.000000   65.000000   45.0         1.0     False  
          1992-02-29   80.000000   80.000000   60.0         1.0     False  
          1992-03-31  100.000000  100.000000   80.0         1.0     False  
          1992-04-30  101.833333  101.666667   80.0         1.0       NaN  
          1992-05-31  103.666667  103.333333   80.0         1.0       NaN  
          1992-06-30  105.500000  105.000000   80.0         1.0       NaN  
          1992-07-31  107.333333  106.666667   80.0         1.0       NaN  
          1992-08-31  109.166667  108.333333   80.0         1.0       NaN  
          1992-09-30  111.000000  110.000000   80.0         1.0       NaN  
          1992-10-31  112.833333  111.666667   80.0         1.0       NaN  
          1992-11-30  114.666667  113.333333   80.0         1.0       NaN  
          1992-12-31  116.500000  115.000000   80.0         1.0       NaN  
          1993-01-31  118.333333  116.666667   80.0         1.0       NaN  
          1993-02-28  120.166667  118.333333   80.0         1.0       NaN  
          1993-03-31  122.000000  120.000000   80.0         1.0     False  
\end{Verbatim}
            
    \begin{Verbatim}[commandchars=\\\{\}]
{\color{incolor}In [{\color{incolor}134}]:} \PY{c+c1}{\PYZsh{} Or we can interpolate with mean()}
          \PY{n}{data2}\PY{o}{.}\PY{n}{resample}\PY{p}{(}\PY{l+s+s2}{\PYZdq{}}\PY{l+s+s2}{M}\PY{l+s+s2}{\PYZdq{}}\PY{p}{)}\PY{o}{.}\PY{n}{mean}\PY{p}{(}\PY{p}{)}\PY{o}{.}\PY{n}{interpolate}\PY{p}{(}\PY{l+s+s2}{\PYZdq{}}\PY{l+s+s2}{linear}\PY{l+s+s2}{\PYZdq{}}\PY{p}{)}
\end{Verbatim}


\begin{Verbatim}[commandchars=\\\{\}]
{\color{outcolor}Out[{\color{outcolor}134}]:}                 \#         HP  Attack  Defense  Sp. Atk  Sp. Def   Speed  \textbackslash{}
          date                                                                      
          1992-01-31  1.000  45.000000    49.0     49.0    65.00    65.00  45.000   
          1992-02-29  2.000  60.000000    62.0     63.0    80.00    80.00  60.000   
          1992-03-31  3.000  80.000000    82.0     83.0   100.00   100.00  80.000   
          1992-04-30  3.125  78.291667    81.5     83.0    99.25    98.75  79.375   
          1992-05-31  3.250  76.583333    81.0     83.0    98.50    97.50  78.750   
          1992-06-30  3.375  74.875000    80.5     83.0    97.75    96.25  78.125   
          1992-07-31  3.500  73.166667    80.0     83.0    97.00    95.00  77.500   
          1992-08-31  3.625  71.458333    79.5     83.0    96.25    93.75  76.875   
          1992-09-30  3.750  69.750000    79.0     83.0    95.50    92.50  76.250   
          1992-10-31  3.875  68.041667    78.5     83.0    94.75    91.25  75.625   
          1992-11-30  4.000  66.333333    78.0     83.0    94.00    90.00  75.000   
          1992-12-31  4.125  64.625000    77.5     83.0    93.25    88.75  74.375   
          1993-01-31  4.250  62.916667    77.0     83.0    92.50    87.50  73.750   
          1993-02-28  4.375  61.208333    76.5     83.0    91.75    86.25  73.125   
          1993-03-31  4.500  59.500000    76.0     83.0    91.00    85.00  72.500   
          
                      Generation  Legendary  
          date                               
          1992-01-31         1.0        0.0  
          1992-02-29         1.0        0.0  
          1992-03-31         1.0        0.0  
          1992-04-30         1.0        0.0  
          1992-05-31         1.0        0.0  
          1992-06-30         1.0        0.0  
          1992-07-31         1.0        0.0  
          1992-08-31         1.0        0.0  
          1992-09-30         1.0        0.0  
          1992-10-31         1.0        0.0  
          1992-11-30         1.0        0.0  
          1992-12-31         1.0        0.0  
          1993-01-31         1.0        0.0  
          1993-02-28         1.0        0.0  
          1993-03-31         1.0        0.0  
\end{Verbatim}
            
    \hypertarget{manipulating-data-frames-with-pandas}{%
\section{MANIPULATING DATA FRAMES WITH
PANDAS}\label{manipulating-data-frames-with-pandas}}

    \hypertarget{indexing-data-frames}{%
\subsubsection{INDEXING DATA FRAMES}\label{indexing-data-frames}}

\begin{itemize}
\tightlist
\item
  Indexing using square brackets
\item
  Using column attribute and row label
\item
  Using loc accessor
\item
  Selecting only some columns
\end{itemize}

    \begin{Verbatim}[commandchars=\\\{\}]
{\color{incolor}In [{\color{incolor}135}]:} \PY{c+c1}{\PYZsh{} read data}
          \PY{n}{data} \PY{o}{=} \PY{n}{pd}\PY{o}{.}\PY{n}{read\PYZus{}csv}\PY{p}{(}\PY{l+s+s1}{\PYZsq{}}\PY{l+s+s1}{../input/pokemon.csv}\PY{l+s+s1}{\PYZsq{}}\PY{p}{)}
          \PY{n}{data}\PY{o}{=} \PY{n}{data}\PY{o}{.}\PY{n}{set\PYZus{}index}\PY{p}{(}\PY{l+s+s2}{\PYZdq{}}\PY{l+s+s2}{\PYZsh{}}\PY{l+s+s2}{\PYZdq{}}\PY{p}{)}
          \PY{n}{data}\PY{o}{.}\PY{n}{head}\PY{p}{(}\PY{p}{)}
\end{Verbatim}


\begin{Verbatim}[commandchars=\\\{\}]
{\color{outcolor}Out[{\color{outcolor}135}]:}             Name Type 1  Type 2  HP  Attack  Defense  Sp. Atk  Sp. Def  Speed  \textbackslash{}
          \#                                                                               
          1      Bulbasaur  Grass  Poison  45      49       49       65       65     45   
          2        Ivysaur  Grass  Poison  60      62       63       80       80     60   
          3       Venusaur  Grass  Poison  80      82       83      100      100     80   
          4  Mega Venusaur  Grass  Poison  80     100      123      122      120     80   
          5     Charmander   Fire     NaN  39      52       43       60       50     65   
          
             Generation  Legendary  
          \#                         
          1           1      False  
          2           1      False  
          3           1      False  
          4           1      False  
          5           1      False  
\end{Verbatim}
            
    \begin{Verbatim}[commandchars=\\\{\}]
{\color{incolor}In [{\color{incolor}136}]:} \PY{c+c1}{\PYZsh{} indexing using square brackets}
          \PY{n}{data}\PY{p}{[}\PY{l+s+s2}{\PYZdq{}}\PY{l+s+s2}{HP}\PY{l+s+s2}{\PYZdq{}}\PY{p}{]}\PY{p}{[}\PY{l+m+mi}{1}\PY{p}{]}
\end{Verbatim}


\begin{Verbatim}[commandchars=\\\{\}]
{\color{outcolor}Out[{\color{outcolor}136}]:} 45
\end{Verbatim}
            
    \begin{Verbatim}[commandchars=\\\{\}]
{\color{incolor}In [{\color{incolor}137}]:} \PY{c+c1}{\PYZsh{} using column attribute and row label}
          \PY{n}{data}\PY{o}{.}\PY{n}{HP}\PY{p}{[}\PY{l+m+mi}{1}\PY{p}{]}
\end{Verbatim}


\begin{Verbatim}[commandchars=\\\{\}]
{\color{outcolor}Out[{\color{outcolor}137}]:} 45
\end{Verbatim}
            
    \begin{Verbatim}[commandchars=\\\{\}]
{\color{incolor}In [{\color{incolor}138}]:} \PY{c+c1}{\PYZsh{} using loc accessor}
          \PY{n}{data}\PY{o}{.}\PY{n}{loc}\PY{p}{[}\PY{l+m+mi}{1}\PY{p}{,}\PY{p}{[}\PY{l+s+s2}{\PYZdq{}}\PY{l+s+s2}{HP}\PY{l+s+s2}{\PYZdq{}}\PY{p}{]}\PY{p}{]}
\end{Verbatim}


\begin{Verbatim}[commandchars=\\\{\}]
{\color{outcolor}Out[{\color{outcolor}138}]:} HP    45
          Name: 1, dtype: object
\end{Verbatim}
            
    \begin{Verbatim}[commandchars=\\\{\}]
{\color{incolor}In [{\color{incolor}139}]:} \PY{c+c1}{\PYZsh{} Selecting only some columns}
          \PY{n}{data}\PY{p}{[}\PY{p}{[}\PY{l+s+s2}{\PYZdq{}}\PY{l+s+s2}{HP}\PY{l+s+s2}{\PYZdq{}}\PY{p}{,}\PY{l+s+s2}{\PYZdq{}}\PY{l+s+s2}{Attack}\PY{l+s+s2}{\PYZdq{}}\PY{p}{]}\PY{p}{]}
\end{Verbatim}


\begin{Verbatim}[commandchars=\\\{\}]
{\color{outcolor}Out[{\color{outcolor}139}]:}       HP  Attack
          \#               
          1     45      49
          2     60      62
          3     80      82
          4     80     100
          5     39      52
          6     58      64
          7     78      84
          8     78     130
          9     78     104
          10    44      48
          11    59      63
          12    79      83
          13    79     103
          14    45      30
          15    50      20
          16    60      45
          17    40      35
          18    45      25
          19    65      90
          20    65     150
          21    40      45
          22    63      60
          23    83      80
          24    83      80
          25    30      56
          26    55      81
          27    40      60
          28    65      90
          29    35      60
          30    60      85
          ..   {\ldots}     {\ldots}
          771   95      65
          772   78      92
          773   67      58
          774   50      50
          775   45      50
          776   68      75
          777   90     100
          778   57      80
          779   43      70
          780   85     110
          781   49      66
          782   44      66
          783   54      66
          784   59      66
          785   65      90
          786   55      85
          787   75      95
          788   85     100
          789   55      69
          790   95     117
          791   40      30
          792   85      70
          793  126     131
          794  126     131
          795  108     100
          796   50     100
          797   50     160
          798   80     110
          799   80     160
          800   80     110
          
          [800 rows x 2 columns]
\end{Verbatim}
            
    \hypertarget{slicing-data-frame}{%
\subsubsection{SLICING DATA FRAME}\label{slicing-data-frame}}

\begin{itemize}
\tightlist
\item
  Difference between selecting columns

  \begin{itemize}
  \tightlist
  \item
    Series and data frames
  \end{itemize}
\item
  Slicing and indexing series
\item
  Reverse slicing
\item
  From something to end
\end{itemize}

    \begin{Verbatim}[commandchars=\\\{\}]
{\color{incolor}In [{\color{incolor} }]:} \PY{c+c1}{\PYZsh{} Difference between selecting columns: series and dataframes}
        \PY{n+nb}{print}\PY{p}{(}\PY{n+nb}{type}\PY{p}{(}\PY{n}{data}\PY{p}{[}\PY{l+s+s2}{\PYZdq{}}\PY{l+s+s2}{HP}\PY{l+s+s2}{\PYZdq{}}\PY{p}{]}\PY{p}{)}\PY{p}{)}     \PY{c+c1}{\PYZsh{} series}
        \PY{n+nb}{print}\PY{p}{(}\PY{n+nb}{type}\PY{p}{(}\PY{n}{data}\PY{p}{[}\PY{p}{[}\PY{l+s+s2}{\PYZdq{}}\PY{l+s+s2}{HP}\PY{l+s+s2}{\PYZdq{}}\PY{p}{]}\PY{p}{]}\PY{p}{)}\PY{p}{)}   \PY{c+c1}{\PYZsh{} data frames}
\end{Verbatim}


    \begin{Verbatim}[commandchars=\\\{\}]
{\color{incolor}In [{\color{incolor} }]:} \PY{c+c1}{\PYZsh{} Slicing and indexing series}
        \PY{n}{data}\PY{o}{.}\PY{n}{loc}\PY{p}{[}\PY{l+m+mi}{1}\PY{p}{:}\PY{l+m+mi}{10}\PY{p}{,}\PY{l+s+s2}{\PYZdq{}}\PY{l+s+s2}{HP}\PY{l+s+s2}{\PYZdq{}}\PY{p}{:}\PY{l+s+s2}{\PYZdq{}}\PY{l+s+s2}{Defense}\PY{l+s+s2}{\PYZdq{}}\PY{p}{]}   \PY{c+c1}{\PYZsh{} 10 and \PYZdq{}Defense\PYZdq{} are inclusive}
\end{Verbatim}


    \begin{Verbatim}[commandchars=\\\{\}]
{\color{incolor}In [{\color{incolor} }]:} \PY{c+c1}{\PYZsh{} Reverse slicing }
        \PY{n}{data}\PY{o}{.}\PY{n}{loc}\PY{p}{[}\PY{l+m+mi}{10}\PY{p}{:}\PY{l+m+mi}{1}\PY{p}{:}\PY{o}{\PYZhy{}}\PY{l+m+mi}{1}\PY{p}{,}\PY{l+s+s2}{\PYZdq{}}\PY{l+s+s2}{HP}\PY{l+s+s2}{\PYZdq{}}\PY{p}{:}\PY{l+s+s2}{\PYZdq{}}\PY{l+s+s2}{Defense}\PY{l+s+s2}{\PYZdq{}}\PY{p}{]} 
\end{Verbatim}


    \begin{Verbatim}[commandchars=\\\{\}]
{\color{incolor}In [{\color{incolor} }]:} \PY{c+c1}{\PYZsh{} From something to end}
        \PY{n}{data}\PY{o}{.}\PY{n}{loc}\PY{p}{[}\PY{l+m+mi}{1}\PY{p}{:}\PY{l+m+mi}{10}\PY{p}{,}\PY{l+s+s2}{\PYZdq{}}\PY{l+s+s2}{Speed}\PY{l+s+s2}{\PYZdq{}}\PY{p}{:}\PY{p}{]} 
\end{Verbatim}


    \hypertarget{filtering-data-frames}{%
\subsubsection{FILTERING DATA FRAMES}\label{filtering-data-frames}}

Creating boolean series Combining filters Filtering column based others

    \begin{Verbatim}[commandchars=\\\{\}]
{\color{incolor}In [{\color{incolor} }]:} \PY{c+c1}{\PYZsh{} Creating boolean series}
        \PY{n}{boolean} \PY{o}{=} \PY{n}{data}\PY{o}{.}\PY{n}{HP} \PY{o}{\PYZgt{}} \PY{l+m+mi}{200}
        \PY{n}{data}\PY{p}{[}\PY{n}{boolean}\PY{p}{]}
\end{Verbatim}


    \begin{Verbatim}[commandchars=\\\{\}]
{\color{incolor}In [{\color{incolor} }]:} \PY{c+c1}{\PYZsh{} Combining filters}
        \PY{n}{first\PYZus{}filter} \PY{o}{=} \PY{n}{data}\PY{o}{.}\PY{n}{HP} \PY{o}{\PYZgt{}} \PY{l+m+mi}{150}
        \PY{n}{second\PYZus{}filter} \PY{o}{=} \PY{n}{data}\PY{o}{.}\PY{n}{Speed} \PY{o}{\PYZgt{}} \PY{l+m+mi}{35}
        \PY{n}{data}\PY{p}{[}\PY{n}{first\PYZus{}filter} \PY{o}{\PYZam{}} \PY{n}{second\PYZus{}filter}\PY{p}{]}
\end{Verbatim}


    \begin{Verbatim}[commandchars=\\\{\}]
{\color{incolor}In [{\color{incolor} }]:} \PY{c+c1}{\PYZsh{} Filtering column based others}
        \PY{n}{data}\PY{o}{.}\PY{n}{HP}\PY{p}{[}\PY{n}{data}\PY{o}{.}\PY{n}{Speed}\PY{o}{\PYZlt{}}\PY{l+m+mi}{15}\PY{p}{]}
\end{Verbatim}


    \hypertarget{transforming-data}{%
\subsubsection{TRANSFORMING DATA}\label{transforming-data}}

\begin{itemize}
\tightlist
\item
  Plain python functions
\item
  Lambda function: to apply arbitrary python function to every element
\item
  Defining column using other columns
\end{itemize}

    \begin{Verbatim}[commandchars=\\\{\}]
{\color{incolor}In [{\color{incolor} }]:} \PY{c+c1}{\PYZsh{} Plain python functions}
        \PY{k}{def} \PY{n+nf}{div}\PY{p}{(}\PY{n}{n}\PY{p}{)}\PY{p}{:}
            \PY{k}{return} \PY{n}{n}\PY{o}{/}\PY{l+m+mi}{2}
        \PY{n}{data}\PY{o}{.}\PY{n}{HP}\PY{o}{.}\PY{n}{apply}\PY{p}{(}\PY{n}{div}\PY{p}{)}
\end{Verbatim}


    \begin{Verbatim}[commandchars=\\\{\}]
{\color{incolor}In [{\color{incolor} }]:} \PY{c+c1}{\PYZsh{} Or we can use lambda function}
        \PY{n}{data}\PY{o}{.}\PY{n}{HP}\PY{o}{.}\PY{n}{apply}\PY{p}{(}\PY{k}{lambda} \PY{n}{n} \PY{p}{:} \PY{n}{n}\PY{o}{/}\PY{l+m+mi}{2}\PY{p}{)}
\end{Verbatim}


    \begin{Verbatim}[commandchars=\\\{\}]
{\color{incolor}In [{\color{incolor} }]:} \PY{c+c1}{\PYZsh{} Defining column using other columns}
        \PY{n}{data}\PY{p}{[}\PY{l+s+s2}{\PYZdq{}}\PY{l+s+s2}{total\PYZus{}power}\PY{l+s+s2}{\PYZdq{}}\PY{p}{]} \PY{o}{=} \PY{n}{data}\PY{o}{.}\PY{n}{Attack} \PY{o}{+} \PY{n}{data}\PY{o}{.}\PY{n}{Defense}
        \PY{n}{data}\PY{o}{.}\PY{n}{head}\PY{p}{(}\PY{p}{)}
\end{Verbatim}


    \hypertarget{index-objects-and-labeled-data}{%
\subsubsection{INDEX OBJECTS AND LABELED
DATA}\label{index-objects-and-labeled-data}}

index: sequence of label

    \begin{Verbatim}[commandchars=\\\{\}]
{\color{incolor}In [{\color{incolor} }]:} \PY{c+c1}{\PYZsh{} our index name is this:}
        \PY{n+nb}{print}\PY{p}{(}\PY{n}{data}\PY{o}{.}\PY{n}{index}\PY{o}{.}\PY{n}{name}\PY{p}{)}
        \PY{c+c1}{\PYZsh{} lets change it}
        \PY{n}{data}\PY{o}{.}\PY{n}{index}\PY{o}{.}\PY{n}{name} \PY{o}{=} \PY{l+s+s2}{\PYZdq{}}\PY{l+s+s2}{index\PYZus{}name}\PY{l+s+s2}{\PYZdq{}}
        \PY{n}{data}\PY{o}{.}\PY{n}{head}\PY{p}{(}\PY{p}{)}
\end{Verbatim}


    \begin{Verbatim}[commandchars=\\\{\}]
{\color{incolor}In [{\color{incolor} }]:} \PY{c+c1}{\PYZsh{} Overwrite index}
        \PY{c+c1}{\PYZsh{} if we want to modify index we need to change all of them.}
        \PY{n}{data}\PY{o}{.}\PY{n}{head}\PY{p}{(}\PY{p}{)}
        \PY{c+c1}{\PYZsh{} first copy of our data to data3 then change index }
        \PY{n}{data3} \PY{o}{=} \PY{n}{data}\PY{o}{.}\PY{n}{copy}\PY{p}{(}\PY{p}{)}
        \PY{c+c1}{\PYZsh{} lets make index start from 100. It is not remarkable change but it is just example}
        \PY{n}{data3}\PY{o}{.}\PY{n}{index} \PY{o}{=} \PY{n+nb}{range}\PY{p}{(}\PY{l+m+mi}{100}\PY{p}{,}\PY{l+m+mi}{900}\PY{p}{,}\PY{l+m+mi}{1}\PY{p}{)}
        \PY{n}{data3}\PY{o}{.}\PY{n}{head}\PY{p}{(}\PY{p}{)}
\end{Verbatim}


    \begin{Verbatim}[commandchars=\\\{\}]
{\color{incolor}In [{\color{incolor} }]:} \PY{c+c1}{\PYZsh{} We can make one of the column as index. I actually did it at the beginning of manipulating data frames with pandas section}
        \PY{c+c1}{\PYZsh{} It was like this}
        \PY{c+c1}{\PYZsh{} data= data.set\PYZus{}index(\PYZdq{}\PYZsh{}\PYZdq{})}
        \PY{c+c1}{\PYZsh{} also you can use data.index = data[\PYZdq{}\PYZsh{}\PYZdq{}]}
\end{Verbatim}


    \hypertarget{hierarchical-indexing}{%
\subsubsection{HIERARCHICAL INDEXING}\label{hierarchical-indexing}}

\begin{itemize}
\tightlist
\item
  Setting indexing
\end{itemize}

    \begin{Verbatim}[commandchars=\\\{\}]
{\color{incolor}In [{\color{incolor} }]:} \PY{c+c1}{\PYZsh{} lets read data frame one more time to start from beginning}
        \PY{n}{data} \PY{o}{=} \PY{n}{pd}\PY{o}{.}\PY{n}{read\PYZus{}csv}\PY{p}{(}\PY{l+s+s1}{\PYZsq{}}\PY{l+s+s1}{../input/pokemon.csv}\PY{l+s+s1}{\PYZsq{}}\PY{p}{)}
        \PY{n}{data}\PY{o}{.}\PY{n}{head}\PY{p}{(}\PY{p}{)}
        \PY{c+c1}{\PYZsh{} As you can see there is index. However we want to set one or more column to be index}
\end{Verbatim}


    \begin{Verbatim}[commandchars=\\\{\}]
{\color{incolor}In [{\color{incolor} }]:} \PY{c+c1}{\PYZsh{} Setting index : type 1 is outer type 2 is inner index}
        \PY{n}{data1} \PY{o}{=} \PY{n}{data}\PY{o}{.}\PY{n}{set\PYZus{}index}\PY{p}{(}\PY{p}{[}\PY{l+s+s2}{\PYZdq{}}\PY{l+s+s2}{Type 1}\PY{l+s+s2}{\PYZdq{}}\PY{p}{,}\PY{l+s+s2}{\PYZdq{}}\PY{l+s+s2}{Type 2}\PY{l+s+s2}{\PYZdq{}}\PY{p}{]}\PY{p}{)} 
        \PY{n}{data1}\PY{o}{.}\PY{n}{head}\PY{p}{(}\PY{l+m+mi}{100}\PY{p}{)}
        \PY{c+c1}{\PYZsh{} data1.loc[\PYZdq{}Fire\PYZdq{},\PYZdq{}Flying\PYZdq{}] \PYZsh{} howw to use indexes}
\end{Verbatim}


    \hypertarget{pivoting-data-frames}{%
\subsubsection{PIVOTING DATA FRAMES}\label{pivoting-data-frames}}

\begin{itemize}
\tightlist
\item
  pivoting: reshape tool
\end{itemize}

    \begin{Verbatim}[commandchars=\\\{\}]
{\color{incolor}In [{\color{incolor}4}]:} \PY{n}{dic} \PY{o}{=} \PY{p}{\PYZob{}}\PY{l+s+s2}{\PYZdq{}}\PY{l+s+s2}{treatment}\PY{l+s+s2}{\PYZdq{}}\PY{p}{:}\PY{p}{[}\PY{l+s+s2}{\PYZdq{}}\PY{l+s+s2}{A}\PY{l+s+s2}{\PYZdq{}}\PY{p}{,}\PY{l+s+s2}{\PYZdq{}}\PY{l+s+s2}{A}\PY{l+s+s2}{\PYZdq{}}\PY{p}{,}\PY{l+s+s2}{\PYZdq{}}\PY{l+s+s2}{A}\PY{l+s+s2}{\PYZdq{}}\PY{p}{,}\PY{l+s+s2}{\PYZdq{}}\PY{l+s+s2}{B}\PY{l+s+s2}{\PYZdq{}}\PY{p}{]}\PY{p}{,}\PY{l+s+s2}{\PYZdq{}}\PY{l+s+s2}{gender}\PY{l+s+s2}{\PYZdq{}}\PY{p}{:}\PY{p}{[}\PY{l+s+s2}{\PYZdq{}}\PY{l+s+s2}{F}\PY{l+s+s2}{\PYZdq{}}\PY{p}{,}\PY{l+s+s2}{\PYZdq{}}\PY{l+s+s2}{M}\PY{l+s+s2}{\PYZdq{}}\PY{p}{,}\PY{l+s+s2}{\PYZdq{}}\PY{l+s+s2}{F}\PY{l+s+s2}{\PYZdq{}}\PY{p}{,}\PY{l+s+s2}{\PYZdq{}}\PY{l+s+s2}{M}\PY{l+s+s2}{\PYZdq{}}\PY{p}{]}\PY{p}{,}\PY{l+s+s2}{\PYZdq{}}\PY{l+s+s2}{response}\PY{l+s+s2}{\PYZdq{}}\PY{p}{:}\PY{p}{[}\PY{l+m+mi}{10}\PY{p}{,}\PY{l+m+mi}{45}\PY{p}{,}\PY{l+m+mi}{5}\PY{p}{,}\PY{l+m+mi}{9}\PY{p}{]}\PY{p}{,}\PY{l+s+s2}{\PYZdq{}}\PY{l+s+s2}{age}\PY{l+s+s2}{\PYZdq{}}\PY{p}{:}\PY{p}{[}\PY{l+m+mi}{15}\PY{p}{,}\PY{l+m+mi}{4}\PY{p}{,}\PY{l+m+mi}{72}\PY{p}{,}\PY{l+m+mi}{65}\PY{p}{]}\PY{p}{\PYZcb{}}
        \PY{n}{df} \PY{o}{=} \PY{n}{pd}\PY{o}{.}\PY{n}{DataFrame}\PY{p}{(}\PY{n}{dic}\PY{p}{)}
        \PY{n}{df}
\end{Verbatim}


\begin{Verbatim}[commandchars=\\\{\}]
{\color{outcolor}Out[{\color{outcolor}4}]:}    age gender  response treatment
        0   15      F        10         A
        1    4      M        45         A
        2   72      F         5         A
        3   65      M         9         B
\end{Verbatim}
            
    \begin{Verbatim}[commandchars=\\\{\}]
{\color{incolor}In [{\color{incolor}5}]:} \PY{c+c1}{\PYZsh{} pivoting}
        \PY{n}{df}\PY{o}{.}\PY{n}{pivot}\PY{p}{(}\PY{n}{index}\PY{o}{=}\PY{l+s+s2}{\PYZdq{}}\PY{l+s+s2}{treatment}\PY{l+s+s2}{\PYZdq{}}\PY{p}{,}\PY{n}{columns} \PY{o}{=} \PY{l+s+s2}{\PYZdq{}}\PY{l+s+s2}{gender}\PY{l+s+s2}{\PYZdq{}}\PY{p}{,}\PY{n}{values}\PY{o}{=}\PY{l+s+s2}{\PYZdq{}}\PY{l+s+s2}{response}\PY{l+s+s2}{\PYZdq{}}\PY{p}{)}
\end{Verbatim}


    \begin{Verbatim}[commandchars=\\\{\}]

        ---------------------------------------------------------------------------

        ValueError                                Traceback (most recent call last)

        <ipython-input-5-88760be3cc73> in <module>()
          1 \# pivoting
    ----> 2 df.pivot(index="treatment",columns = "gender",values="response")
    

        \textasciitilde{}/Desktop/virenv/lib/python3.6/site-packages/pandas/core/frame.py in pivot(self, index, columns, values)
       4380         """
       4381         from pandas.core.reshape.reshape import pivot
    -> 4382         return pivot(self, index=index, columns=columns, values=values)
       4383 
       4384     \_shared\_docs['pivot\_table'] = """


        \textasciitilde{}/Desktop/virenv/lib/python3.6/site-packages/pandas/core/reshape/reshape.py in pivot(self, index, columns, values)
        387         indexed = Series(self[values].values,
        388                          index=MultiIndex.from\_arrays([index, self[columns]]))
    --> 389         return indexed.unstack(columns)
        390 
        391 


        \textasciitilde{}/Desktop/virenv/lib/python3.6/site-packages/pandas/core/series.py in unstack(self, level, fill\_value)
       2222         """
       2223         from pandas.core.reshape.reshape import unstack
    -> 2224         return unstack(self, level, fill\_value)
       2225 
       2226     \# ----------------------------------------------------------------------


        \textasciitilde{}/Desktop/virenv/lib/python3.6/site-packages/pandas/core/reshape/reshape.py in unstack(obj, level, fill\_value)
        472     else:
        473         unstacker = \_Unstacker(obj.values, obj.index, level=level,
    --> 474                                fill\_value=fill\_value)
        475         return unstacker.get\_result()
        476 


        \textasciitilde{}/Desktop/virenv/lib/python3.6/site-packages/pandas/core/reshape/reshape.py in \_\_init\_\_(self, values, index, level, value\_columns, fill\_value)
        114 
        115         self.\_make\_sorted\_values\_labels()
    --> 116         self.\_make\_selectors()
        117 
        118     def \_make\_sorted\_values\_labels(self):


        \textasciitilde{}/Desktop/virenv/lib/python3.6/site-packages/pandas/core/reshape/reshape.py in \_make\_selectors(self)
        152 
        153         if mask.sum() < len(self.index):
    --> 154             raise ValueError('Index contains duplicate entries, '
        155                              'cannot reshape')
        156 


        ValueError: Index contains duplicate entries, cannot reshape

    \end{Verbatim}

    \hypertarget{stacking-and-unstacking-dataframe}{%
\subsubsection{STACKING and UNSTACKING
DATAFRAME}\label{stacking-and-unstacking-dataframe}}

\begin{itemize}
\tightlist
\item
  deal with multi label indexes
\item
  level: position of unstacked index
\item
  swaplevel: change inner and outer level index position
\end{itemize}

    \begin{Verbatim}[commandchars=\\\{\}]
{\color{incolor}In [{\color{incolor} }]:} \PY{n}{df1} \PY{o}{=} \PY{n}{df}\PY{o}{.}\PY{n}{set\PYZus{}index}\PY{p}{(}\PY{p}{[}\PY{l+s+s2}{\PYZdq{}}\PY{l+s+s2}{treatment}\PY{l+s+s2}{\PYZdq{}}\PY{p}{,}\PY{l+s+s2}{\PYZdq{}}\PY{l+s+s2}{gender}\PY{l+s+s2}{\PYZdq{}}\PY{p}{]}\PY{p}{)}
        \PY{n}{df1}
        \PY{c+c1}{\PYZsh{} lets unstack it}
\end{Verbatim}


    \begin{Verbatim}[commandchars=\\\{\}]
{\color{incolor}In [{\color{incolor} }]:} \PY{c+c1}{\PYZsh{} level determines indexes}
        \PY{n}{df1}\PY{o}{.}\PY{n}{unstack}\PY{p}{(}\PY{n}{level}\PY{o}{=}\PY{l+m+mi}{0}\PY{p}{)}
\end{Verbatim}


    \begin{Verbatim}[commandchars=\\\{\}]
{\color{incolor}In [{\color{incolor} }]:} \PY{n}{df1}\PY{o}{.}\PY{n}{unstack}\PY{p}{(}\PY{n}{level}\PY{o}{=}\PY{l+m+mi}{1}\PY{p}{)}
\end{Verbatim}


    \begin{Verbatim}[commandchars=\\\{\}]
{\color{incolor}In [{\color{incolor} }]:} \PY{c+c1}{\PYZsh{} change inner and outer level index position}
        \PY{n}{df2} \PY{o}{=} \PY{n}{df1}\PY{o}{.}\PY{n}{swaplevel}\PY{p}{(}\PY{l+m+mi}{0}\PY{p}{,}\PY{l+m+mi}{1}\PY{p}{)}
        \PY{n}{df2}
\end{Verbatim}


    \hypertarget{melting-data-frames}{%
\subsubsection{MELTING DATA FRAMES}\label{melting-data-frames}}

\begin{itemize}
\tightlist
\item
  Reverse of pivoting
\end{itemize}

    \begin{Verbatim}[commandchars=\\\{\}]
{\color{incolor}In [{\color{incolor} }]:} \PY{c+c1}{\PYZsh{} df.pivot(index=\PYZdq{}treatment\PYZdq{},columns = \PYZdq{}gender\PYZdq{},values=\PYZdq{}response\PYZdq{})}
        \PY{n}{pd}\PY{o}{.}\PY{n}{melt}\PY{p}{(}\PY{n}{df}\PY{p}{,}\PY{n}{id\PYZus{}vars}\PY{o}{=}\PY{l+s+s2}{\PYZdq{}}\PY{l+s+s2}{treatment}\PY{l+s+s2}{\PYZdq{}}\PY{p}{,}\PY{n}{value\PYZus{}vars}\PY{o}{=}\PY{p}{[}\PY{l+s+s2}{\PYZdq{}}\PY{l+s+s2}{age}\PY{l+s+s2}{\PYZdq{}}\PY{p}{,}\PY{l+s+s2}{\PYZdq{}}\PY{l+s+s2}{response}\PY{l+s+s2}{\PYZdq{}}\PY{p}{]}\PY{p}{)}
\end{Verbatim}


    \hypertarget{categoricals-and-groupby}{%
\subsubsection{CATEGORICALS AND
GROUPBY}\label{categoricals-and-groupby}}

    \begin{Verbatim}[commandchars=\\\{\}]
{\color{incolor}In [{\color{incolor} }]:} \PY{c+c1}{\PYZsh{} We will use df}
        \PY{n}{df}
\end{Verbatim}


    \begin{Verbatim}[commandchars=\\\{\}]
{\color{incolor}In [{\color{incolor} }]:} \PY{c+c1}{\PYZsh{} according to treatment take means of other features}
        \PY{n}{df}\PY{o}{.}\PY{n}{groupby}\PY{p}{(}\PY{l+s+s2}{\PYZdq{}}\PY{l+s+s2}{treatment}\PY{l+s+s2}{\PYZdq{}}\PY{p}{)}\PY{o}{.}\PY{n}{mean}\PY{p}{(}\PY{p}{)}   \PY{c+c1}{\PYZsh{} mean is aggregation / reduction method}
        \PY{c+c1}{\PYZsh{} there are other methods like sum, std,max or min}
\end{Verbatim}


    \begin{Verbatim}[commandchars=\\\{\}]
{\color{incolor}In [{\color{incolor} }]:} \PY{c+c1}{\PYZsh{} we can only choose one of the feature}
        \PY{n}{df}\PY{o}{.}\PY{n}{groupby}\PY{p}{(}\PY{l+s+s2}{\PYZdq{}}\PY{l+s+s2}{treatment}\PY{l+s+s2}{\PYZdq{}}\PY{p}{)}\PY{o}{.}\PY{n}{age}\PY{o}{.}\PY{n}{mean}\PY{p}{(}\PY{p}{)} 
\end{Verbatim}


    \begin{Verbatim}[commandchars=\\\{\}]
{\color{incolor}In [{\color{incolor} }]:} \PY{c+c1}{\PYZsh{} Or we can choose multiple features}
        \PY{n}{df}\PY{o}{.}\PY{n}{groupby}\PY{p}{(}\PY{l+s+s2}{\PYZdq{}}\PY{l+s+s2}{treatment}\PY{l+s+s2}{\PYZdq{}}\PY{p}{)}\PY{p}{[}\PY{p}{[}\PY{l+s+s2}{\PYZdq{}}\PY{l+s+s2}{age}\PY{l+s+s2}{\PYZdq{}}\PY{p}{,}\PY{l+s+s2}{\PYZdq{}}\PY{l+s+s2}{response}\PY{l+s+s2}{\PYZdq{}}\PY{p}{]}\PY{p}{]}\PY{o}{.}\PY{n}{mean}\PY{p}{(}\PY{p}{)} 
\end{Verbatim}


    \begin{Verbatim}[commandchars=\\\{\}]
{\color{incolor}In [{\color{incolor} }]:} \PY{n}{df}\PY{o}{.}\PY{n}{info}\PY{p}{(}\PY{p}{)}
        \PY{c+c1}{\PYZsh{} as you can see gender is object}
        \PY{c+c1}{\PYZsh{} However if we use groupby, we can convert it categorical data. }
        \PY{c+c1}{\PYZsh{} Because categorical data uses less memory, speed up operations like groupby}
        \PY{n}{df}\PY{p}{[}\PY{l+s+s2}{\PYZdq{}}\PY{l+s+s2}{gender}\PY{l+s+s2}{\PYZdq{}}\PY{p}{]} \PY{o}{=} \PY{n}{df}\PY{p}{[}\PY{l+s+s2}{\PYZdq{}}\PY{l+s+s2}{gender}\PY{l+s+s2}{\PYZdq{}}\PY{p}{]}\PY{o}{.}\PY{n}{astype}\PY{p}{(}\PY{l+s+s2}{\PYZdq{}}\PY{l+s+s2}{category}\PY{l+s+s2}{\PYZdq{}}\PY{p}{)}
        \PY{n}{df}\PY{p}{[}\PY{l+s+s2}{\PYZdq{}}\PY{l+s+s2}{treatment}\PY{l+s+s2}{\PYZdq{}}\PY{p}{]} \PY{o}{=} \PY{n}{df}\PY{p}{[}\PY{l+s+s2}{\PYZdq{}}\PY{l+s+s2}{treatment}\PY{l+s+s2}{\PYZdq{}}\PY{p}{]}\PY{o}{.}\PY{n}{astype}\PY{p}{(}\PY{l+s+s2}{\PYZdq{}}\PY{l+s+s2}{category}\PY{l+s+s2}{\PYZdq{}}\PY{p}{)}
        \PY{n}{df}\PY{o}{.}\PY{n}{info}\PY{p}{(}\PY{p}{)}
\end{Verbatim}


    \hypertarget{conclusion}{%
\section{CONCLUSION}\label{conclusion}}

Thank you for your votes and comments \textbf{MACHINE LEARNING }
https://www.kaggle.com/kanncaa1/machine-learning-tutorial-for-beginners/
\textbf{If you have any question or suggest, I will be happy to hear
it.}


    % Add a bibliography block to the postdoc
    
    
    
    \end{document}
